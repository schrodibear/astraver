\documentclass[a4paper]{llncs}
\usepackage{url}

\newcommand{\kw}[1]{\textsf{#1}}
\newcommand{\minelt}[1]{\ensuremath{\mathit{min\_elt}(#1)}}

\begin{document}

\title{Queens on a Chessboard: \\
       an Exercise in Program Verification}
\author{Jean-Christophe Filli\^atre}
\institute{CNRS}
\maketitle

\begin{abstract}
  This article details the formal verification of a 2-lines C program
  which computes the number of solutions to the $N$-queens problem.
\end{abstract}

\section{Introduction}

\begin{verbatim}
t(a,b,c){int d=0,e=a&~b&~c,f=1;if(a)for(f=0;d=(e-=d)&-e;f+=t(a-d,(b+d)*2,(
c+d)/2));return f;}main(q){scanf("%d",&q);printf("%d\n",t(~(~0<<q),0,0));}
\end{verbatim}

\url{http://www.iwriteiam.nl/SigProgC.html}

\section{Unobfuscation}

\begin{verbatim}
int t(int a, int b, int c) {
  int d=0, e=a&~b&~c, f=1;
  if (a)
    for (f=0; d=e&-e; e-=d)
      f += t(a-d,(b+d)*2,(c+d)/2));
  return f;
}
int queens(int q) {
  return t(~(~0<<q),0,0);
}
\end{verbatim}

{\begin{obeylines}
  \kw{int} $t$($a$, $b$, $c$) 
  ~~ \kw{if} $a \not= \emptyset$ 
  ~~~~ $e$ $\leftarrow$ $(a \backslash  b) \backslash c$ 
  ~~~~ $f$ $\leftarrow$ 0 
  ~~~~ \kw{while} $e \not=\emptyset$ 
  ~~~~~~ $d$ $\leftarrow$ $\minelt{e}$ 
  ~~~~~~ $f$ $\leftarrow$ $f$ $+$ $t$($a\backslash \{d\}$, $S(b\cup\{d\})$, $P(c\cup\{d\})$) 
  ~~~~~~ $e$ $\leftarrow$ $e \backslash  \{d\}$ 
  ~~~~ \kw{return} $f$ 
  ~~ \kw{else} 
  ~~~~ \kw{return} $1$ 
  ~~ 
  \kw{int} $f$($n$) 
  ~~ \kw{return} $t$($\{0,1,\dots,n-1\}$, $\emptyset$, $\emptyset$)
\end{obeylines}}


\section{Formal verification}

\subsection{The Caduceus tool}

\subsection{Termination}

\subsection{Correctness}

\section{Conclusion}

\nocite{*}
\bibliographystyle{plain}
\bibliography{./biblio}

\end{document}
