%\documentclass[handout,compress]{beamer}
\documentclass[compress]{beamer}

\usepackage{beamerjcf}

\title{Queens on a Chessboard}
\author{Jean-Christophe Filli\^atre}
%\institute{CNRS -- Universit\'e Paris Sud}
\date[Krakatoa/Caduceus WG]{Dec 15th, 2006\\
  Krakatoa/Caduceus working group}

\begin{document}

\begin{frame}
  \begin{center}
    {\Huge\emph{Queens on a Chessboard}} \\[1em]
    {\LARGE\emph{an exercise in program verification}} \\[3em]
    {\large Jean-Christophe Filli\^atre} \\[2em]
    Krakatoa/Caduceus working group \\[0.5em]
    December 15th, 2006
  \end{center}
\end{frame}
 
\begin{frame}[fragile]
  \frametitle{Introduction}

challenge for \emph{the verified program of the month}:

\bigskip

{\scriptsize
\begin{verbatim}
   t(a,b,c){int d=0,e=a&~b&~c,f=1;if(a)for(f=0;d=(e-=d)&-e;f+=t(a-d,(b+d)*2,(
   c+d)/2));return f;}main(q){scanf("%d",&q);printf("%d\n",t(~(~0<<q),0,0));}
\end{verbatim}}

\Pause

appears on a web page collecting C signature programs 

\bigskip

due to Marcel van Kervinck, \par
author of MSCP (Marcel's Simple Chess Program)
\end{frame}

\begin{frame}
  \frametitle{Unobfuscating...}
\begin{caduceus}
int t(int a, int b, int c) {
  int d=0, e=a&~b&~c, f=1;
  if (a)
    for (f=0; d=(e-=d)&-e;)
      f+=t(a-d,(b+d)*2,(c+d)/2);
  return f;
}
\end{caduceus}
\only<1>{%
\begin{caduceus}
int main(int q) {
  scanf("%d",&q);
  printf("%d\n",t(~(~0<<q),0,0));
}
\end{caduceus}
}%
\only<2>{%
\begin{caduceus}
int f(int n) {
  return t(~(~0<<n), 0, 0);
}
\end{caduceus}

\bigskip

we end up with a mysterious function 
\texttt{f} : $\mathbb{N} \rightarrow \mathbb{N}$
}
\end{frame}

\begin{frame}
  \frametitle{Queens on a chessboard}

given a number $n$ smaller than 32, 
$\mathtt{f}(n)$ is the number of ways to put $n$ queens on $n\times n$
chessboard so that they cannot beat each other

\Pause

let us prove that this program is \emph{correct}, that is:
\begin{itemize}
\item it does not crash
\item it terminates
\item it computes the right number
\end{itemize}
\end{frame}

\begin{frame}
  \frametitle{How does it work?}
\end{frame}

\end{document}

%%% Local Variables: 
%%% mode: latex
%%% TeX-master: t
%%% End: 
