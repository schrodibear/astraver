\documentclass{llncs}

\usepackage{amssymb}
\usepackage{alltt}
\usepackage{url}
\usepackage{pstricks}
\usepackage{pst-node}
\usepackage{comment}
\usepackage{times}
\usepackage{listings}


% common Why title page

\newcommand{\whytitlepage}[4]{%
\begin{titlepage}
\begin{center}
~\vfill
\rule\textwidth{0.1cm}\\[0.5cm]
\begin{Huge}\sffamily
#1 % title
\end{Huge}
\\[1cm]
\begin{Large}\sffamily
#2
\end{Large}
\\[0.1cm]
\rule\textwidth{0.1cm}\\[1cm]
Version #3\\[3cm]
#4
\vfill
\today\\
INRIA Team-Project \emph{Proval} \url{http://proval.lri.fr} \\
INRIA Futurs \& LRI, CNRS UMR 8623\\ 
4, rue Jacques Monod, 91893 Orsay cedex, France
\end{center}
\end{titlepage}}

\newcommand{\why}{\textsf{Why}}
\newcommand{\Why}{\why}
\newcommand{\java}{\textsc{Java}\index{Java@\textsf{Java}}}
\newcommand{\Java}{\java}
\newcommand{\krakatoa}{\textsf{Krakatoa}\index{Krakatoa@\textsf{Krakatoa}}}
\newcommand{\Krakatoa}{\krakatoa}
\newcommand{\caduceus}{\textsf{Caduceus}\index{Caduceus@\textsf{Caduceus}}}
\newcommand{\Caduceus}{\caduceus}
\newcommand{\coq}{\textsf{Coq}\index{Coq@\textsf{Coq}}}
\newcommand{\Coq}{\coq}
\newcommand{\pvs}{\textsf{PVS}\index{PVS@\textsf{PVS}}}

%

\newcommand{\kw}[1]{\ensuremath{\mathsf{#1}}}

% types
\newcommand{\bool}{\kw{bool}}
\newcommand{\unit}{\kw{unit}}
%\newcommand{\tref}[1]{\ensuremath{#1~\kw{ref}}}
\newcommand{\tref}[1]{\ensuremath{#1~\mathsf{ref}}}
\newcommand{\tarray}[2]{\ensuremath{\kw{array}~#1~\kw{of}~#2}}

% constructs
\newcommand{\prepost}[3]{\ensuremath{\{#1\}\,#2\,\{#3\}}}
\newcommand{\result}{\ensuremath{\mathit{result}}}

\newcommand{\void}{\kw{void}}
\newcommand{\access}[1]{\ensuremath{!#1}}
\newcommand{\assign}[2]{\ensuremath{#1~:=~#2}}
\newcommand{\pref}[1]{\ensuremath{\kw{ref}~#1}}
\newcommand{\taccess}[2]{\ensuremath{#1\texttt{[}#2\texttt{]}}}
\newcommand{\tassign}[3]{\ensuremath{#1\texttt{[}#2\texttt{]}~\texttt{:=}~#3}}
\newcommand{\faccess}[2]{\ensuremath{(\mathit{access}~#1~#2)}}
\newcommand{\fupdate}[3]{\ensuremath{(\mathit{update}~#1~#2~#3)}}
%\newcommand{\taccess}[2]{\ensuremath{#1[#2]}}
%\newcommand{\tassign}[3]{\ensuremath{#1[#2]~:=~#3}}
%\newcommand{\faccess}[2]{\ensuremath{(\mathit{access}~#1~#2)}}
%\newcommand{\fupdate}[3]{\ensuremath{(\mathit{update}~#1~#2~#3)}}
% \newcommand{\block}[1]{\ensuremath{\kw{begin}~#1~\kw{end}}}
\newcommand{\seq}[2]{\ensuremath{#1;~#2}}
%\newcommand{\plabel}[2]{\ensuremath{#1:#2}}
\newcommand{\plabel}[2]{\ensuremath{#1\texttt{:}#2}}
\newcommand{\assert}[2]{\ensuremath{\kw{assert}~\{#1\};~#2}}
\newcommand{\while}[4]{\ensuremath{\kw{while}~#1~\kw{do}~\{\kw{invariant}~#2~\kw{variant}~#3\}~#4~\kw{done}}}
\newcommand{\ite}[3]{\ensuremath{\kw{if}~#1~\kw{then}~#2~\kw{else}~#3}}
\newcommand{\fun}[3]{\ensuremath{\kw{fun}~#1:#2\rightarrow#3}}
\newcommand{\app}[2]{\ensuremath{(#1~#2)}}
\newcommand{\rec}[4]{\ensuremath{\kw{rec}~#1:#2~\{\kw{variant}~#3\}=#4}}
\newcommand{\letin}[3]{\ensuremath{\kw{let}~#1=#2~\kw{in}~#3}}
\newcommand{\raisex}[2]{\ensuremath{\kw{raise}~(#1~#2)}}
\newcommand{\exn}[1]{\ensuremath{\kw{Exn}~#1}}
\newcommand{\try}[2]{\ensuremath{\kw{try}~#1~\kw{with}~#2~\kw{end}}}
\newcommand{\coerce}[2]{\ensuremath{(#1:#2)}}

\newcommand{\statement}{\textit{statement}}
\newcommand{\program}{\textit{program}}
\newcommand{\expression}{\textit{expression}}
\newcommand{\predicate}{\textit{predicate}}

% inference rules
\newcommand{\espacev}{\rule{0in}{1em}}
\newcommand{\espacevn}{\rule[-0.4em]{0in}{1em}}
\newcommand{\irule}[2]
  {\frac{\espacevn\displaystyle#1}{\espacev\displaystyle#2}}
\newcommand{\typage}[3]{#1 \, \vdash \, #2 : #3}
\newcommand{\iname}[1]{\textsf{#1}}

\newcommand{\emptyef}{\bot}
\newcommand{\wf}[1]{#1~\kw{wf}}
\newcommand{\pur}[1]{#1~\kw{pure}}
\newcommand{\variant}[1]{#1~\kw{variant}}

\newcommand{\wpre}[2]{\ensuremath{\mathit{wp}(#1,#2)}}
\newcommand{\wprx}[3]{\ensuremath{\mathit{wp}(#1,#2,#3)}}

\newcommand{\barre}[1]{\ensuremath{\overline{#1}}}

%%% Local Variables: 
%%% mode: latex
%%% TeX-master: "doc"
%%% End: 


% LLNCS hack: too many space around subsection titles
\makeatletter
\def\subsection{\@startsection{subsection}{2}{\z@}{-8pt plus-3pt minus
 -3pt}{5pt plus 3pt minus 0pt}{\normalsize\bf\boldmath
\pretolerance=10000\relax\rightskip=0pt plus8em}}
\makeatother

\pagestyle{plain}

\begin{document}

\title{A case study of C source code verification: the Schorr-Waite algorithm}
\author{Thierry Hubert and Claude March\'e}
\institute{PCRI --- LRI (CNRS UMR 8623) --- INRIA Futurs ---
  Universit\'e Paris 11 \\
B\^at 490, Universit\'e Paris-sud, 91405 Orsay cedex, France \\
\{\texttt{hubert},\texttt{marche}\}\texttt{@lri.fr} 
}
\maketitle

\begin{abstract}
  We describe an experiment done using the \caduceus{} tool for formal
  verification of C programs. We performed a full formal proof of the
  classical Schorr-Waite graph-marking algorithm, which has already
  been used several times as a case study for formal reasoning
  computer systems. Our study is original with respect to previous
  experiments for several reasons. First, we use a general-purpose
  tool for C source code: we start from a real source code written in C,
  specified using an annotation language for arbitrary C programs.
  (BOF Second, we are not bound to a specific prover.) Third, we indeed
  formally established more properties of the algorithm than previous
  works, in particular a formal proof of termination is made.

  \noindent{\bf Keywords:} C programming language, Hoare logic,
  pointer programs, formal verification and proof.
\end{abstract}

\footnotetext{This research was partly supported by the ``projet
  GECCOO de l'ACI S\'ecurit\'e Informatique'', CNRS \& INRIA,
  \url{http://gecco.lri.fr} + Dassault ? Averroes ?} 




\hfill\begin{minipage}{0.5\textwidth}
\begin{slshape}
The Schorr-Waite algorithm is the first moutain that any formalism for
pointer aliasing should climb.
\end{slshape}

~\hfill --- Richard Bornat (\cite{bornat00mpc}, page 121)
\end{minipage}

\section{Introduction}

Using formal methods for verifying properties of programs at their
source code level has gained more interest with the increased use of
embedded programs, which are short programs where a high-level of
confidence is required. Such embedded programs are no more written in
assembly language but in C (plane command control, cars, etc.) or in
JavaCard~\cite{JavaCard} (mobile phones, smart cards, etc.).  To
perform formal verification of C or Java programs, one faces the
general issue of verification of pointer programs: \emph{aliasing},
that is referencing a memory location by several ways, must be taken
into account. 

The Schorr-Waite algorithm~\cite{schorr67cacm} is a graph-marking
algorithm, intended to be used in garbage collectors. It performs a
depth-first traversal of an arbitrary graph structure (hence a
structure where aliasing may occur), without using additional memory,
but using the pointers in the graph structure itself as a backtracking
stack. A first (non computer-aided) proof of correctness of this
algorithm has been proposed by Topor~\cite{topor79acta}. {\huge citer
  aussi Morris \cite{morris82} Reynolds }

In 2000, Bornat~\cite{bornat00mpc} has been able to perform a
computer-aided formal proof of the Schorr-Waite algorithm using the
Jape system. To make the proof tractable, an old idea by
Burstall~\cite{burstall72} for reasoning on pointer programs as been
reused: the heap memory is not modelled by single large array, but by
several ones, separating the memory into parts where it is statically
known that no aliasing can occur between two parts. {\huge Bornat dit
  component-as-array trick} These parts correspond to differents
fields of records (i.e. structures in C, or instance variables in
Java).

In 2003, the Schorr-Waite algorithm has been used again as a case
study by Mehta and Nipkow~\cite{mehta03cade}, this time for
verification of pointer programs in the higher-logic system
Isabelle/HOL. PARLER aussi de la preuve en B

In 2004, we proposed a new verification method for ANSI
C source code~\cite{filliatre04icfem}.  A prototype tool has been
implemented, called \caduceus{}, freely available for
experimentation~\cite{Caduceus}. This tool is currently under
experimentation at Axalto (smart cards) and Dassault Aviation
(aeronautics). Unlike formerly mentioned systems, \caduceus{} takes as
input real source code in the C programming language, where
specifications are given inside regular C comments, in the spirit of
the Java Modeling Language~\cite{leavens00jml}. 

To provide an evidence that \caduceus{} is a powerful approach (and
tool) to formal verification of C source, we decided to perform again
the verification of the Schorr-Waite algorithm {\huge faire reference
  a la citation de Bornat}, and this article report
on this successful experiment.

It has to be noticed that with \caduceus{}, we indeed formally proved
more properties of the Schorr-Waite algorithm than previous studies.
The first additional property we show is that if the graph structure
is regularly allocated at the beginning, then no invalid pointer
deferencing occur during execution of the algorithm. Even this is not
hard to prove, this is a very important thing to prove on C programs,
and it is indeed mandatory with the \caduceus{} approach. The second
additional property we show is about the remaining of the memory, that
is the part of the memory that is not accessible from the root node
given as a starting point of the algorithm. We are able to show that
in this remaining part, nothing is changed. The third additional
property is the termination: although informal termination arguments
were known since Topor~\cite{topor79acta} (A VERIFIER), no formal
verification has been made before. This is now done.

This paper is organized as follows.  Section~\ref{sec:overview} gives
an overview of the use of \caduceus{}.  In Section~\ref{sec:algo}, we
describe the Schorr-Waite algorithm and its functional specification.
We then describe the formal proof of it in Section~\ref{sec:proof},
including the additional annotations needed, mainly the loop
invariant.  We conclude in Section~\ref{sec:conclusion} with a
comparison to related works.





%%% Local Variables: 
%%% mode: latex
%%% TeX-master: "rapport"
%%% End: 

\section{Overview of the \caduceus{} approach}
\label{sec:overview}

In the following, we assume the reader fairly familiar with ANSI
C~\cite{KR88}.  In the \caduceus{} setting, the properties that may be
checked are of two kinds. First, the program may be checked to be free
of \emph{threats}, that are operations which may lead to abnormal
program termination: division by zero and such, including the crucial
case of dereferencing a pointer that does not point to a valid memory
block (i.e. on the current stack or allocated on the heap by a
\texttt{malloc}-like call). In our approach, out-of-bounds array
access is a particular case of pointer dereferencing threat, because
we support full pointer arithmetic.  The second kind of properties
which can be proved are user-defined behavioral properties given as
functions's \emph{postconditions}.

These kinds of properties are not independent since they both usually
need insertion of appropriate annotations (functions's
\emph{preconditions}, \emph{global invariants}, \emph{loop
  invariants}, etc.) as usual in a Hoare logic framework.  In
practice, these annotations are inserted in the source code as
comments of a specific shape \verb!/*@...*/!. The specification
language we use in those comments is largely inspired by the Java
Modeling Language (JML)~\cite{leavens00jml}. It has however
significant differences, mainly due to the fact that unlike JML we do
not seek runtime assertion checking. Description of this annotation
language is not the purpose of this article ; we refer
to~\cite{Caduceus,filliatre04icfem} for more information. We simply
introduce here what we need for the case study.  One feature of this
annotation language, shared with JML, is that annotations follow
a syntax similar to C, so that a C programmer may learn it
quite easily.
 
Once a C program is annotated, verification is done by running
\caduceus{} on its source code in a way similar to a classical
compiler, but resulting in the generation of so-called
\emph{verification conditions}: first-order predicate logic formulas
whose validity implies the soundness of the program with respect to
the absence of threat and to the behavioral  properties given as
annotations. At this point, a general purpose theorem prover must be
used to establish those verification conditions.  An original
\caduceus{} feature is its independence with respect to the prover. It
currently supports Coq~\cite{CoqProofAssistant} and PVS~\cite{PVS}
interactive proof assistants, and Simplify~\cite{simplify},
haRVey~\cite{ranise03harvey} and CVC-lite~\cite{barrett04cav}
automatic provers. New provers may easily be added via a suitable
pretty-printer.

To perform proof of the verification conditions with an
interactive prover, the user needs to understand the
modeling of the memory heap provided by \caduceus{}. The
`component-as-array' trick of Burstall and Bornat is re-used, extended
to arrays and pointer arithmetic in
general~\cite{filliatre04icfem}. For this case study, no pointer
arithmetic occurs in the considered source code, and fortunately the
generated verification conditions are exactly what they could be
expected with the original `component-as-array' modeling. That is, the
extension of this trick to pointer arithmetic does not add any
overhead when the source code makes no use of pointer arithmetic.

%%% Local Variables: 
%%% mode: latex
%%% TeX-master: "rapport"
%%% End: 


\section{The Schorr-Waite algorithm and its formal specification}
\label{sec:algo}

The Schorr-Waite algorithm performs a depth-first traversal of a
directed graph, starting from a specific node of the graph called the
root. Its main characteristic is that it directly uses the pointers of
the graph to implement backtracking. This is why its soundness is
quite difficult to establish. 

\subsection{The C source code}

\begin{figure*}[t]
\begin{alltt}
void schorr_waite(node root) \{
  node t = root; node p = NULL;
  while (p != NULL || (t != NULL && ! t->m)) \{
    if (t == NULL || t->m) \{
      if (p->c) \{ \begin{slshape}/* \textbf{pop} */\end{slshape}
        node q = t; t = p; p = p->r; t->r = q;
      \} 
      else \{ \begin{slshape}/* \textbf{swing} */\end{slshape}
        node q = t; t = p->r; p->r = p->l; p->l = q; p->c = 1;
      \}
    \} 
    else \{ \begin{slshape}/* \textbf{push} */ \end{slshape}
      node q = p; p = t; t = t->l; p->l = q; p->m = 1; p->c = 0;
    \}
  \}
\}
\end{alltt}
\caption{C version of the Schorr-Waite algorithm}
\label{fig:code}
\end{figure*}

\begin{figure*}
\vspace*{2mm}
\begin{center}
  \unitlength=0.5mm
  \psset{unit=\unitlength,arrows=->}
%\fbox{
\begin{picture}(180,65)(-80,115)
\put(-80,180){\rnode{null}{\makebox(0,0)[l]{\tiny NULL}}}
\put(-80,160){\unode{n0}}
\put(-70,140){\unode{n1}}
\put(-90,140){\unode{n2}}
\put(-100,180){\rnode{p}{\makebox(0,0)[r]{$p$}}}
\put(-100,160){\rnode{t}{\makebox(0,0)[r]{$t$}}}
\put(-100,120){\rnode{null2}{\makebox(0,0)[c]{\tiny NULL}}}
\put(-80,120){\rnode{null3}{\makebox(0,0)[c]{\tiny NULL}}}
\put(-60,160){\rnode{fig1}}
\put(-75,130){\rnode{n11}}
\put(-65,130){\rnode{n12}}
\ncline[linestyle=dotted]{-}{n1}{n11}
\ncline[linestyle=dotted]{-}{n1}{n12}
\ncline{n0}{n1}
\ncline{n0}{n2}
\ncline[nodesepB=1mm]{n2}{null2}
\ncline[nodesepB=1mm]{n2}{null3}
\ncline[nodesep=1mm]{p}{null}
\ncline[nodesepA=1mm]{t}{n0}
\put(-30,160){\rnode{fig2}}
\put(0,180){\rnode{null}{\makebox(0,0)[b]{\tiny NULL}}}
\put(0,160){\mnode{n0}}
\put(10,140){\unode{n1}}
\put(-10,140){\unode{n2}}
\put(-20,155){\rnode{p}{\makebox(0,0)[r]{$p$}}}
\put(-30,140){\rnode{t}{\makebox(0,0)[r]{$t$}}}
\put(-20,120){\rnode{null2}{\makebox(0,0)[c]{\tiny NULL}}}
\put(0,120){\rnode{null3}{\makebox(0,0)[c]{\tiny NULL}}}
\put(5,130){\rnode{n11}}
\put(15,130){\rnode{n12}}
\ncline[linestyle=dotted]{-}{n1}{n11}
\ncline[linestyle=dotted]{-}{n1}{n12}
\ncarc[nodesepB=1mm,arcangleA=130,linewidth=1]{n0}{null}
\ncline{n0}{n1}
\ncline[nodesepB=1mm]{n2}{null2}
\ncline[nodesepB=1mm]{n2}{null3}
\ncarc[nodesepA=1mm,arcangleA=-10,arcangleB=-50]{p}{n0}
\ncline[nodesepA=1mm]{t}{n2}
\ncline[linewidth=2mm,linecolor=gray]{fig1}{fig2}\Aput{push}
\put(20,160){\rnode{fig2}}
\put(50,160){\rnode{fig3}}
\put(80,180){\rnode{null}{\makebox(0,0)[b]{\tiny NULL}}}
\put(80,160){\mnode{n0}}
\put(90,140){\unode{n1}}
\put(70,140){\mnode{n2}}
\put(50,135){\rnode{p}{\makebox(0,0)[r]{$p$}}}
\put(40,120){\rnode{t}{\makebox(0,0)[r]{$t$}}}
\put(60,120){\rnode{null2}{\makebox(0,0)[c]{\tiny NULL}}}
\put(80,120){\rnode{null3}{\makebox(0,0)[c]{\tiny NULL}}}
\put(95,130){\rnode{n11}}
\put(85,130){\rnode{n12}}
\ncline[linestyle=dotted]{-}{n1}{n11}
\ncline[linestyle=dotted]{-}{n1}{n12}
\ncarc[nodesepB=1mm,arcangleA=130,linewidth=1]{n0}{null}
\ncarc[arcangleA=140,linewidth=1]{n2}{n0}
\ncline{n0}{n1}
\ncline[nodesepB=1mm]{n2}{null3}
\ncarc[nodesepA=1mm,arcangleA=-10,arcangleB=-40]{p}{n2}
\ncline[nodesepA=1mm,nodesepB=3mm]{t}{null2}
\ncline[linewidth=2mm,linecolor=gray]{fig2}{fig3}\Aput{push}
\put(100,160){\rnode{fig3}}
\put(130,160){\rnode{fig4}}
\ncline[linewidth=2mm,linecolor=gray]{fig3}{fig4}\Aput{swing}
\end{picture}
%}
%\\\hrulefill
%\vspace*{5mm}
%\fbox{
\begin{picture}(180,65)(-80,30)
\put(-80,80){\rnode{null}{\makebox(0,0)[b]{\tiny NULL}}}
\put(-80,60){\mnode{n0}}
\put(-70,40){\unode{n1}}
\put(-90,40){\mnode{n2}}
\put(-110,50){\rnode{p}{\makebox(0,0)[r]{$p$}}}
\put(-60,20){\rnode{t}{\makebox(0,0)[l]{$t$}}}
\put(-100,20){\rnode{null2}{\makebox(0,0)[c]{\tiny NULL}}}
\put(-80,20){\rnode{null3}{\makebox(0,0)[c]{\tiny NULL}}}
\put(-75,30){\rnode{n11}}
\put(-65,30){\rnode{n12}}
\ncline[linestyle=dotted]{-}{n1}{n11}
\ncline[linestyle=dotted]{-}{n1}{n12}
\ncarc[nodesepB=1mm,arcangleA=120,linewidth=1]{n0}{null}
\ncarc[arcangleA=-120,linewidth=1]{n2}{n0}
\ncline{n0}{n1}
\ncline[nodesepA=1mm]{p}{n2}
\ncline[nodesepA=1mm,nodesepB=3mm]{t}{null3}
\ncline[nodesepB=1mm]{n2}{null2}
\put(-60,60){\rnode{fig4}}
\put(-30,60){\rnode{fig5}}
\put(0,80){\rnode{null}{\makebox(0,0)[b]{\tiny NULL}}}
\put(0,60){\mnode{n0}}
\put(10,40){\unode{n1}}
\put(-10,40){\mnode{n2}}
\put(-20,50){\rnode{p}{\makebox(0,0)[r]{$p$}}}
\put(20,20){\rnode{t}{\makebox(0,0)[l]{$t$}}}
\put(-20,20){\rnode{null2}{\makebox(0,0)[t]{\tiny NULL}}}
\put(0,20){\rnode{null3}{\makebox(0,0)[t]{\tiny NULL}}}
\put(5,30){\rnode{n11}}
\put(15,30){\rnode{n12}}
\ncline[linestyle=dotted]{-}{n1}{n11}
\ncline[linestyle=dotted]{-}{n1}{n12}
\ncline{n0}{n1}
\ncline[nodesepB=1mm]{n2}{null2}
\ncline[nodesepB=1mm]{n2}{null3}
\ncarc[nodesepA=1mm,arcangleA=-10,arcangleB=-40]{p}{n0}
\ncarc[nodesepA=1mm]{t}{n2}
\ncarc[nodesepB=1mm,arcangleA=120,linewidth=1]{n0}{null}
\ncline[linewidth=2mm,linecolor=gray]{fig4}{fig5}\Aput{pop}
\put(20,60){\rnode{fig5}}
\put(50,60){\rnode{fig6}}
\put(80,80){\rnode{null}{\makebox(0,0)[b]{\tiny NULL}}}
\put(80,60){\mnode{n0}}
\put(90,40){\unode{n1}}
\put(70,40){\mnode{n2}}
\put(60,50){\rnode{p}{\makebox(0,0)[r]{$p$}}}
\put(110,30){\rnode{t}{\makebox(0,0)[l]{$t$}}}
\put(60,20){\rnode{null2}{\makebox(0,0)[t]{\tiny NULL}}}
\put(80,20){\rnode{null3}{\makebox(0,0)[t]{\tiny NULL}}}
\put(95,30){\rnode{n11}}
\put(85,30){\rnode{n12}}
\ncline[linestyle=dotted]{-}{n1}{n11}
\ncline[linestyle=dotted]{-}{n1}{n12}
\put(90,60){\rnode{fig1}}
\ncline{n0}{n2}
\ncline[nodesepB=1mm]{n2}{null2}
\ncline[nodesepB=1mm]{n2}{null3}
\ncline[nodesepA=1mm]{p}{n0}
\ncline[nodesepA=1mm]{t}{n1}
\ncarc[nodesepB=1mm,arcangleA=-120,linewidth=1]{n0}{null}
\ncline[linewidth=2mm,linecolor=gray]{fig5}{fig6}\Aput{swing}
\put(110,60){\rnode{fig6}}
\put(140,60){\rnode{fig7}}
\ncline[linewidth=2mm,linecolor=gray]{fig6}{fig7}\Aput{...}
\end{picture}
%}
\end{center}
\vspace*{5mm}
\caption{Schorr-Waite algorithm: sample execution}
\label{fig:algo}
\end{figure*}

In the version we consider here, we assume that the nodes have
at most two child, which is the situation for example in a garbage
collector of a pure Lisp interpreter, where the only memory allocated
value is the \textsf{cons} of lists. In C, these \textsf{cons}
structures can be defined as
\begin{alltt}
typedef struct struct_node \{
  unsigned int m:1, c:1; /* booleans */
  struct struct_node *l, *r;
\} * node;
\end{alltt}
where $l$ and $r$ are respectively pointers to the left and to the
right child (they may be set to NULL). Fields $m$ and $c$ are integers
over 1 bit, intended to represent booleans. The field $m$ is a
mark: initially, all nodes of the graph will be assumed unmarked, and
at the end of the traversal, they will have to be all marked. The
field $c$ is used internally by the algorithm, to denote which
of the child is currently been explored.
 
\begin{figure*}
\begin{alltt}\begin{slshape}
/*@ \textbf{requires}
  @  \bs{}forall node x; 
  @     x != \bs{}null && reachable(root,x) => \valid(x) && ! x->m  
  @ \textbf{ensures}
  @  (\bs{}forall node x; \old(x->l) == x->l && \old(x->r) == x->r) 
  @ &&
  @  (\bs{}forall node x; x != \bs{}null && reachable(root,x) => x->m) 
  @ &&
  @  (\bs{}forall node x; !reachable(root,x) => x->m == \old(x->m))
  @*/\end{slshape}
void schorr_waite(node root) \{
  ...
\}
\end{alltt}
\caption{Specification of Schorr-Waite algorithm in the \caduceus{}
  syntax}
\label{fig:spec}
\end{figure*}

The Schorr-Waite graph-marking algorithm, written directly in true C
source code, is given in Figure~\ref{fig:code}. 
%This algorithm is a non-recursive graph marking. 
The beginning of a sample execution is illustrated in
Figure~\ref{fig:algo}, with each possible move `push', `swing' and
`pop'. The black nodes represent marked nodes, whereas white nodes are
unmarked. The sample graph shown is a tree for readibility, but of
course any graph could be given. $t$ is the next node to 
be explored and $p$ is the head of the backtracking stack. `push' marks a
new node and then explores the left child. `swing' occurs when the left
child has been explored: the search continues to the right child. `pop' is used
when the right child is been explored: the search goes back up.        


\subsection{Formal specification of the algorithm}
\label{sec:spec}

The first step in the formal verification process is to give a formal
specification to the \verb|schorr_waite| function. The informal
specification says that every node in the graph, reachable from the
root node, must be marked. Moreover the graph structure must be
restored to its initial state.

So it appears immediately that to formally specify the algorithm, one
needs to talk about reachability of some node from another in the
graph. Indeed reasoning about reachability is the main part of all the
verification, as already noticed by previous studies of the
Schorr-Waite algorithm.

At this point, the \caduceus{} methodology is quite different from the
JML one: in JML, one would need to provide a so-called \emph{pure}
(i.e. side-effect free) method defining reachability and using it in
annotations. This is convenient for runtime assertion checking, but
for formal verification it is not, because we need to reason (e.g.
by induction) about reachability, not only compute with it. In the
\caduceus{} methodology, such a logical notion must be declared, by a
\verb|predicate| annotation, and supposed to be defined or axiomatized
in the back-end prover. This is very similar to a forward declaration
of a C function, whose implementation will be given later: such
predicates have to be `implemented' later, either by giving
additional \caduceus{} annotations (such as \verb|axioms|) or in the
back-end prover. For the reachability predicate, we write
%\pagebreak
\begin{alltt}\begin{slshape}
/*@ predicate 
  @   reachable(node p1, node p2) 
  @   reads p1->l,p1->r */
\end{slshape}\end{alltt}
This declares a binary predicate on nodes, but nothing yet is
said about its semantics. The \texttt{reads} clause that follows the predicate
declaration above gives information on which data the predicate
depends on: in this case, it does depend on \verb|p1| and
\verb|p2| but also on the child of \verb|p1|. This may be confusing
at first, because in fact this predicate depends not only on the
child of \verb|p1|, but also the grand child and so on. To
really understand the meaning of the \texttt{reads} clause, one needs
to remember that we use the component-as-array modeling: this
declaration indeed means that this predicate will have, on the prover
side, additional `array' arguments \verb|l| and \verb|r| respectively
representing the left and right child of all nodes. We will come back
to this in Section~\ref{sec:coq}.


Once this \texttt{reachable} predicate is declared, it is possible to
give a formal specification to the algorithm, even if we did not give
the semantics of \texttt{reachable} yet. The specification, in the
\caduceus{} syntax, is given in Figure~\ref{fig:spec}. It is made of
two clauses: the \texttt{requires} clause specifies the pre-condition
whereas the \texttt{ensures} clause specifies the post-condition. Both
are followed by a logical formula, where logical connectives follow a
C-like syntax: C operator \verb|&&| denotes conjunction, \verb|!|
denotes negation, etc. Additional syntax is introduced to denote
implication by \verb|=>|, and universal quantification by
$\bs{}\verb|forall|~\textsl{type}~x ; \textsl{formula}$. C expressions 
can be used in logical formulas as long as they are side-effect free. On this example we use equality and inequality (\verb|==| and
\verb|!=|) and field accesses like \verb|x->l|. Additional predicates
exist, all prefixed by a backslash, such as \verb|\null| which 
denotes the null pointer. Notice that our annotation language
inherits from C the fact that booleans are not clearly distinguished
from the integers: when an integer expression $e$ is used as a logical
atom, then it should be understood as the boolean $e\neq 0$, so that
one can equivalently write \verb|! x->m| or \verb|x->m == 0|. 
Notice that we use a standard first-order logic: every function is
total (like in JML, i.e. no 3-valued logic) so that an expression
\verb|p->f| is meaningfull even if \verb|p| is \verb|null|, but no
property of it is known (very similarly to division by zero for
example). 

In the pre-condition, the built-in predicate $\valid(p)$ means that
the pointer $p$ points to a correctly allocated memory block, i.e. that
dereferencing $p$ is safe. So the pre-condition specifies that for any
node $x$ that is not null and reachable from the root, it should be
regularly allocated and its $m$ field must be false (that is $x$ is
unmarked).

The post-condition is a conjunct of three assertions. The built-in
construct $\old(e)$ in \caduceus{}, inspired from the JML one, denotes
the value of expression $e$ before the execution of the function. So
the first assertion means that for all nodes (even the non-reachable
ones), the child are the same after the run of the algorithm, as
they were before. This is of course very important to specify, because
during the execution the child are modified, so one wants to prove
that the initial graph structure is restored. The second
assertion specifies that any non-null node $x$ reachable from the
root is now marked. So this specifies that the whole 
graph has been traversed. The third assertion says that for the nodes
which were not reachable, their mark is not changed by the algorithm,
that is the unreachable nodes are not traversed by the algorithm.

With respect to the formal specification given by
Bornat~\cite{bornat00mpc} or the similar one in
Isabelle/HOL~\cite{mehta03cade}, we added two things: first the
pre-condition that all pointers of the graph are regularly
allocated (which is the meaning of \verb|\valid|) at the beginning, so
that we are now able to prove that no 
wrong pointer dereferencing can occur. Second, we also talk about the
non-reachable nodes of the graph: we are also able to show that they
are not traversed and not modified. In Section~\ref{sec:termination},
we will add more annotations in order to establish the termination of the
\verb|schorr_waite| function.


%%% Local Variables: 
%%% mode: latex
%%% TeX-master: "rapport"
%%% End: 


\section{Verification}
\label{sec:proof}

The function is formally specified, now the hard work begins, we need
to prove that the implementation of Figure~\ref{fig:code} indeed
satisfies this specification. If one runs the \caduceus{} tool on that
annotated code now, it will generate verification conditions that
are indeed not provable: as usual with a Hoare-like system, when there
are loops like the \verb|while| of Schorr-Waite, it is mandatory
(except in simple cases) to manually add a loop invariant. The design
of a suitable loop invariant is the most difficult part of
the verification process, more difficult than performing the proof
themselves. Fortunately for us, suitable loop invariants were proposed
by previous works on Schorr-Waite algorithm. However, some
difficulties remain. The main one is that we need to express this loop
invariant in the \caduceus{} syntax, which is quite poor, for example
in~\cite{mehta03cade} they use a higher-order syntax that is simply
not allowed by \caduceus. Also, since we proved a little bit more than
previous works, namely the absence of invalid pointer dereferencing,
we needed to add new assertions in the loop invariant.

\subsection{Designing the loop invariant}

\begin{figure}[t]
\begin{center}
  \unitlength=0.5mm
  \psset{unit=\unitlength,arrows=->}
\begin{picture}(100,100)
\put(50,100){\rnode{null}{\makebox(0,0)[b]{\tiny NULL}}}
\put(50,80){\mnode{n0}}\put(58,83){\makebox(0,0)[b] {c=false}}
\put(30,60){\mnode{n1}}\put(23,63){\makebox(0,0)[b] {c=true}}
\put(60,60){\unode{n2}}

\put(20,40){\mnode{n11}}
\put(40,40){\mnode{n12}}\put(50,43){\makebox(0,0)[b] {c=false}}
\put(50,20){\unode{n122}}
\put(30,20){\unode{n121}}
\ncline[linestyle=dashed]{n0}{n1}
\ncarc[arcangleA=120,linewidth=1]{n0}{null}
\ncline{n0}{n2}
\ncline{n1}{n11}
\ncline{n12}{n122}
\ncline[linestyle=dashed]{n1}{n12}
\ncline[linestyle=dashed]{n12}{n121}
\ncarc[arcangleA=120,linewidth=1]{n12}{n1}
\ncarc[arcangleA=-100,arcangleB=-30,linewidth=1]{n1}{n0}
\end{picture}\label{figure}
\end{center}
\caption{The backtracking stack inside the graph structure}
\label{fig:stack}
\end{figure}

The key idea for the loop invariant is that even if the graph
structure is temporarily modified, the reachability of nodes is
preserved. More precisely, each node that is initially reachable from
the root will be always reachable either from p or from t. We also
need to describe as a loop invariant how the backtracking stack is
implemented using the pointers of the graph. On
Figure~\ref{fig:stack}, we display some state of the graph as it may
occur at an arbitrary iteration of the loop. The dashed arrows
corresponds to the children pointers as they were at the beginning,
whereas the thick arrows shows their current values. the normal arrows
corresponds to unmodified links. The backtracking stack is indeed the
list of pointers, starting from p ({\huge t ?}), following either the left
children when c is 0, or the right children when c is 1.


\subsection{Writing the loop invariant}

\begin{figure}[t]
\begin{alltt}
void schorr_waite(node root) \{
  node t = root;
  node p = NULL;\begin{slshape}
  /*@ invariant
    @  (I1 :: \bs{}forall node x; 
    @     \old(reachable(root,x)) => 
    @        reachable(t,x) || reachable(p,x))
    @ &&
    @  (I2 :: \bs{}forall node x; x != \null => 
    @     ((reachable(t,x) || reachable(p,x)) => 
    @        \old(reachable(root,x))))
    @ &&
    @  (I3 :: \bs{}forall node x; ! \old(reachable(root,x)) => 
    @     x->m == \old(x->m)) 
    @ &&
    @  \bs{}exists plist stack;
    @   (I4a :: clr_list (p,stack)) 
    @   &&
    @   (I4b :: \bs{}forall node p; in_list (p,stack) => p->m) 
    @   &&
    @   (I4c :: \bs{}forall node x; \valid(x) && \old(reachable(root,x)) && !x->m =>
    @      unmarked_reachable(t,x) || 
    @      (\bs{}exists node y; in_list(y,stack) && unmarked_reachable(y->r,x))) 
    @   &&
    @   (I4d :: \bs{}forall node x; !in_list(x,stack) =>  
    @      (x->r == \old(x->r) && x->l == \old(x->l))) 
    @   &&
    @   (I4e :: \bs{}forall node p1; \bs{}forall node p2;
    @          pair_in_list(p1,p2,cons(t,stack)) => 
    @         (p2->c => \old(p2->l) == p2->l && \old(p2->r) == p1)
    @         &&
    @        (!p2->c => \old(p2->l) == p1 && \old(p2->r) ==
    @              p2->r)) 
    @*/\end{slshape}
  while (p != NULL || (t != NULL && ! t->m)) \{
    ...
\end{alltt}
\caption{The loop invariant}
\label{fig:loopinv}
\end{figure}

The loop invariant as it is writing in the \caduceus{} annotation
language is displayed on Figure~\ref{fig:loopinv}.  For specifying this
loop invariant, in particular the stack, we need to talk about lists
of pointers, in annotations. With \caduceus{}, such a datatype from
the logical side can be imported, implicitly. Then new predicates and
logical function symbols can be declared, as the \verb|reachable|
predicate already introduced, which can take as argument either C
types or logical types. Here we declare
\begin{alltt}
\begin{slshape}
/*@ logic plist cons(node p, plist l) */
/*@ predicate in_list(node p,plist l) */
/*@ predicate pair_in_list(node p1,node p2, plist l) */
\end{slshape}
\end{alltt}
which first introduce the logical type \verb|plist| for finite lists
of pointers, then a logical function symbol \verb|cons| which adds an
element to a list, a predicate \verb|in_list| which tests of a
pointer belongs to a list, and finally a predicate \verb|pair_in_list|,
saying that p1 and p2 are two consecutive nodes of a list.

The loop invariant finally requires the declaration of two predicates:
\begin{alltt}
\begin{slshape}
/*@ predicate unmarked_reachable (node p1, node p2) 
  @   reads p1->r,p1->l,p1->m 
  @*/
\end{slshape}
\end{alltt}
specifies that p2 is reachable from p1, by traversing only unmarked
nodes; and
\begin{alltt}
\begin{slshape}
/*@ predicate clr_list (node p, plist stack) 
  @   reads p->c,p->l,p->r 
  @*/
\end{slshape}
\end{alltt}
specifies that \verb|stack| is the list of pointers obtained from p as
described on Figure~\ref{fig:stack}.

On Figure~\ref{fig:loopinv}, we used the \caduceus{} notation to name
formulas, of the form \textsl{name}\verb|::|\textsl{formula}. The loop
invariant has 9 assertions:
\begin {enumerate}
\item {I1} : means that all nodes which were initially reachable from
  root, are reachable from t or p
\item {I2} : means that conversely, all (non-null) nodes reachable by t or
  p were initially reachable from root. This is true only for non-null
  node, because null may not be reachable from root is the graph is cyclic.
\item {I3} : means that for nodes initially unreachable, the mark
  remains unchanged.
\item {I4a} : defines the backtracking stack as on Figure~\ref{fig:stack}.
\item {I4b} : means that all nodes in the \texttt{stack} are marked.
\item {I4c} : the trickest annotation, due to {\Huge ?}, means that
  any valid {\Huge non null ?} node, initially reachable from root, 
  and not yet marked, is reachable by traversing only
  unmarked nodes, either from t or from the right children of a node
  in the stack.
\item {I4d} : means all nodes out of \texttt{stack} have
  the same children as before the function. 
\item {I4e} : a trick one also, that is essential for being able to
  `recover' the correct children of nodes of the stack, as it can be
  seen again on Figure~\ref{fig:stack}: if p1 is in
  the stack, followed by p2 (or if p1 is t and p2 is the head of the
  stack), then : if p2->c is
  1 then its left child is the same as its was initially
     and its right child is p1 ; and if p2->c is 0 then its left child
   is p1 and its right child is the same as it was initially.
\end{enumerate}


\subsection{Performing the verification}

12 VCG

. invariant vrai init
. post cond
. 3 fois invariant preserve resp. pop swing push
. validit� des acc�s
. po1 = WF order
. po2 = validity of access to t->m in the condition of while 
. po3 = validity of access to t->m in the condition of the first if 
. po4 = validity of access to p->c in the condition of the second if 
. po5 = validity of t in 't->r = q' in the first branch of if 
. po6 = preservation of loop invariant for "pop" 
. po7 = validity of p in "p->r = p->l" in second branch of if (swing) 
. po8 = preservation of loop invariant for "swing" 
. po9 = validity of acces to t in the "push" branch
. po10 = validity of acces to p in the "push" branch
. po11 = preservation of loop invariant for "push" 
. po12 = dummy po due to variant
. po13 = post-condition of the function
. po14 = loop invariant true at the beginning 

with Simplify: 

.***.*.*..*.** (6/8/0)

-> 4/12 (po5,7,9 et 10)

ajout axiome reachable(p,p) : 

.....*.*..*.** (9/5/0)   

-> 7/12 (po2,3,4 en plus)


passage a Coq:

def de plist avec list de la lib standard de Coq

def de llist (le modele component-as-array exposed to the user)

def clr\_list

def des path a la prolog: 3 clauses

def de reachable, unmarked\_reachable
 
lemme intermediaires

Coq: 12/12

chiffres avec coqwc

\subsubsection{Intermediate lemmas}

For define in coq the reachability two definitions:
Firstly a inductive definition  \texttt{path} which define a path between two pointers.
\begin{alltt}
\begin{slshape}
Inductive path (a: alloc_table) (l: memory pointer)(r: memory pointer) : 
pointer ->  pointer -> list pointer -> Prop :=
  | Path_null : forall p:pointer, path a l r p  p nil
  | Path_left :
      forall p1 p2:pointer,
      forall lp : list pointer,
        valid a p1 ->
          path a l r (acc l p1) p2 lp-> path a l r p1  p2 (p1::lp)
  | Path_right :
      	forall p1 p2:pointer,
      forall lp : list pointer ,
        	valid a p1 ->
        	  path a l r (acc r p1)  p2 lp -> path a l r p1 p2 (p1::lp). 
\end{slshape}
\end{alltt}
Secondly a definition  \texttt{reachable} which define if two pointers are
reachable i.e. there exist a path between this two pointers.
\begin{alltt}
\begin{slshape}
Definition reachable (a: alloc_table) 
  (l: memory pointer)(r: memory pointer) 
  (p1 :pointer) (p2:pointer) : Prop :=
  exists lp : list pointer, path a l r p1 p2 lp
\end{slshape}
\end{alltt}

We supplement this definition with the lemma
\texttt{path\_no\_cycle} which affirm if
there exist a path between this two pointers then there exist a path
between this two pointers without cycle.
\begin{alltt}
\begin{slshape}
Lemma path_no_cycle : forall (a : alloc_table) 
  (p1 p2 : pointer) (l r : memory pointer) 
  (pa : list pointer), path a l r p1 p2 pa -> 
  exists pa' :list pointer, 
  incl pa' pa /\ no_rep pa' /\ path a l r p1 p2 pa'. 
\end{slshape}
\end{alltt}

And we supplement this definition with four lemmas
\texttt{path\_upd\_left}, \texttt{path\_upd\_right},
\texttt{path\_inv\_upd\_left} and \texttt{path\_inv\_upd\_right} which
affirm if we have un path between the two pointers p1 p2 and if we
modify the graphe without touch the pointers in this path then this
path exists always. 
\begin{alltt}
\begin{slshape}
Lemma path_upd_left : forall (alloc : alloc_table) 
  (l r : memory pointer) (p:pointer)
  (lp : list pointer) (p1 p0 p2 : pointer), 
  ~ In p lp -> path alloc l r p1 p0 lp -> 
  path alloc (upd l p p2) r p1 p0 lp.
\end{slshape}
\end{alltt}



An other definition on reachability is neccessary, the definition
 \texttt{unmarked\_reachable} which define if there exist a path
 between this two pointers with all the pointers in this path are not marked.
\begin{alltt}
\begin{slshape}
Definition unmarked_reachable (a: alloc_table) 
  (m:memory Z) (l r: memory pointer) (p1 p2:pointer) 
  : Prop :=
  exists lp : list pointer, 
    (forall x : pointer, In x lp -> (acc m x) = 0 ) 
    /\ path a l r p1 p2 lp.
\end{slshape}
\end{alltt}


An other important definition for this proof is \texttt{clr\_list}
which define the pile. For this it use a inductive definition llist
which define a list form a initial pointer and un function
\texttt{next} which give us a follower of a pointer.  
\begin{alltt}
\begin{slshape}
Definition clr_list (a: alloc_table)  (c:memory Z) (l: memory pointer)
(r: memory pointer) : pointer ->list pointer-> Prop :=
let next t := if Z_eq_dec (acc c t) 0 then (acc l t) else (acc r t) in
llist a next .
\end{slshape}
\end{alltt}

\section{Termination}
\label{sec:termination}

\begin{alltt}
/*@ logic Length weight(node p , node t) reads p->m,p->c,p->l,p->r*/

/*@ requires \bs{}exists plist l; reachable_elements(root,root,l) */
void schorr_waite(node root) \{
  node t = root;
  node p = NULL;
  /*@variant weight(p,t) for order_mark_m_and_c_and_stack 
    @*/
  while (p != NULL || (t != NULL && ! t->m)) \{
   ...
  \}
\}
\end{alltt}

To proof the termination of the while we need an addition in the
precondition. This addition specifies there are a list of all the
graph's elements.

The clause variant specifies the order who decrease for all step of
while. This order is the function \verb|order_mark_m_and_c_stack|, this
function is an order lexicographic of the number of graph's elements
not mark with m, the number of graph's elements not mark with c and the number
of element in the \texttt{stack}. 



2 VCG en plus: 


%%% Local Variables: 
%%% mode: latex
%%% TeX-master: "rapport"
%%% End: 

\section{Conclusion}
\label{sec:conclusion}

We presented a machine-checked full verification of the Schorr-Waite
algorithm, directly on a real C source version of it. We have not only
proved the behavioral properties of it, like previous
works~\cite{bornat00mpc,mehta03cade}, but we have also proved that no
invalid pointer dereferencing may occur, and the
termination. Moreover we used a tool for arbitrary C programs, without
anything specialized for this case study, which additionally gives the
choice of the back-end prover.

We claim this is a very successful case study, which provides a clear
evidence that the \caduceus{} approach is powerful. Anyway, there are
still many potential improvements. There is a clear need for
cooperation between automatic provers and interactive
ones. Interactive provers are nice because they are very expressive
and allow to prove manually difficult goals, but their automatic
tactics are less powerful than automatic provers. This is especially
true for Coq, because any decision procedure in it needs to provide a
checkable proof trace, making its integration very difficult.

Several improvements to the \caduceus{} tool are in progress. There are
scaling up issues for dealing with code of large size. We are
currently investigating solutions, in particular by integration of
advanced static analysis techniques.

An interesting question is what would be the next ``mountain to
climb'' for pointer program verification? multi-procedure programs,
separation {\Huge TODO}

{\Huge comparer ACSL avec JML}

%%% Local Variables: 
%%% mode: latex
%%% TeX-master: "rapport"
%%% End: 


\bigskip
\paragraph{Acknowledgments.} 
We thank Jean-Christophe Filli�tre for its fruitful help, advices and
remarks about this case study.

\bibliographystyle{abbrv}
\input{biblio-demons}


\end{document}

%%% Local Variables: 
%%% mode: latex
%%% TeX-master: t
%%% End: 
