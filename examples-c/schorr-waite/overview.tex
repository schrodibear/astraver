\section{Overview of the caduceus approach}
\label{sec:overview}

In the following, we assume the reader fairly familiar with ANSI
C~\cite{KR88}.  In our setting, the properties that may be checked are
of two kinds: first, the program may be checked to be free of
\emph{threats} (null pointer dereferencing or out-of-bounds array
access) and second, it may be proved satisfying functional properties
given as functions's \emph{postconditions}. These kinds of properties
are not independent since they both usually need insertion of
appropriate annotations (functions's \emph{preconditions},
\emph{global invariants}, \emph{loop invariants}, etc.) as usual in a
Hoare logic framework.  In practice, these annotations are inserted in
the source code as comments of a specific shape \verb!/*@...*/!. The
specification language we use in those comments is largely inspired by
the Java Modeling Language (JML)~\cite{leavens00jml}. It has however
significant differences, mainly due to the fact that unlike JML we do
not seek runtime assertion checking.
 
Once a C program is annotated, verification is done by running 
\caduceus{} on the sources in a way similar to a classical compiler
but resulting in the generation of the so-called \emph{verification
  conditions}: these are first-order predicate logic formulas whose validity
implies the 
soundness of the program with respect to the absence of threat and
to the functional properties given as annotations. At this point, a general
purpose theorem prover must be used to establish those verification conditions.
An original \caduceus{} feature is its independence with respect to
the prover. It currently supports the Coq interactive proof
assistant~\cite{CoqProofAssistant} and the Simplify automatic
prover~\cite{simplify}. New 
provers may easily be added provided a suitable
pretty-printer; support for PVS~\cite{PVS} and haRVey~\cite{ranise03harvey}
are planned in a near future.


PARLER du modele: on utilise aussi un mod�le Burstall-Bornat, mais
�tendu aux tableaux, et � l'arithm�tique de pointeurs.

%%% Local Variables: 
%%% mode: latex
%%% TeX-master: "rapport"
%%% End: 
