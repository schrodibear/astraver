\section{Conclusion}
\label{sec:conclusion}

We presented a machine-checked full verification of the Schorr-Waite
algorithm, directly on a real C source version of it. We have not only
proved the behavioral properties of it, like previous
works~\cite{bornat00mpc,mehta03cade}, but we have also proved that no
invalid pointer dereferencing may occur, and the
termination. Moreover we used a tool for arbitrary C programs, without
anything specialized for this case study, which additionally gives the
choice of the back-end prover.

We claim this is a very successful case study, which provides a clear
evidence that the \caduceus{} approach is powerful. We think that the
ability of using predicates in annotations which are only defined on
the prover side is fundamental. Consequently, it is probably a
weakness of the JML annotation language for Java, that only Java
pure methods can be used in annotations.

There are still many potential improvements. There is a clear
need for cooperation between automatic provers and interactive ones.
Interactive provers are nice because they are very expressive and
allow to prove manually difficult goals, but their automatic tactics
are less powerful than automatic provers. This is especially true for
Coq, because any decision procedure in it needs to provide a checkable
proof trace, making its integration very difficult.

Several improvements to the \caduceus{} tool are in progress. There are
scaling up issues for dealing with code of large size. Properties of
separation of data appears to be crucial for large programs. We are
currently investigating solutions, in particular by integration of
advanced static analysis techniques, and ideas from separation
logic~\cite{reynolds02lics}.

An interesting question is what would be the next ``mountain to
climb'' for pointer program verification? We think that
memory allocation and multi-procedure programs (where specification of
separation is crucial) are important issues. Significant challenges
include advanced graph algorithms (such as planarity test) and
Ukkonen's algorithm~\cite{ukkonen95}.


%%% Local Variables: 
%%% mode: latex
%%% TeX-master: "rapport"
%%% End: 
