\section{Conclusion}
\label{sec:conclusion}

We presented a machine-checked full verification of the Schorr-Waite
algorithm, directly on a real C source version of it. We have not only
proved the behavioral properties of it, like previous
works~\cite{bornat00mpc,mehta03cade,abrial03fme}, but we have also
proved that no invalid pointer dereferencing may occur, and the
termination.

The termination does not use any unique feature of \caduceus{}:
termination is indeed established in~\cite{abrial03fme} as a condition
for refinement, and we believe that it could probably also proved by
others as well. The important point is that termination is smoothly
integrated in our methodology, thanks to the powerful clause
\texttt{variant} we have, allowing to use arbitrary well-founded
relations.

Regarding the absence of invalid pointer dereferencing, it has to be
said that advanced static analysis techniques are able to establish it
also. The Schorr-Waite algorithm is one of the examples automatically
handled by the TVLA approach~\cite{sagiv02toplas,reps03esop}, where a powerful
analysis of reachability in pointer structures is done. But as far as
we know, only our approach is able to handle both the pointer
dereferencing checks and the validity of behavioral properties given
by the user. Notice that with \caduceus, as shown in
Section~\ref{sec:simplify}, absence of invalid pointer dereferencing
is established automatically, even if \caduceus{} does not do any kind of
reachability analysis by itself.

Finally, we emphasize again that we use a generic tool for arbitrary C
programs, without anything specialized for this case study, which
additionally gives the choice of the back-end prover. At the
beginning, it was not clear whether the \caduceus{} specification
language would be enough, in particular because it is first-order, but
we finally succeeded.  We claim this is a very successful case study,
which provides a clear evidence that the \caduceus{} approach is
powerful. We think that the ability of using predicates in annotations
which are only defining on the prover side is fundamental. As said
before, our specification language is inspired from JML annotation
language for Java. But with JML, as far as we know, it is not possible
to declare predicates without also defined them as Java methods: these
are the so-called \emph{pure} methods. The latter choice is natural
for the main goal of JML, that is runtime assertion checking:
specifications must be executable. But for the goal of formal
verification, we believe that this is not a good choice, in other
words we do not agree with the claim in~\cite{leavens03a} that JML is
suitable both for runtime assertime checking and formal verification.
Of course, it should be easy to simply relax the JML constraint that
pure methods must have an implementation. But then runtime assertion
checking is not possible anymore: achieving both runtime assertion
checking and formal verification with exactly the same JML annotations
remains, to us, an open problem. This would require to define
predicates in a way which allows both computation and reasoning with
them, which is possible as far as we know, only in higher-order
logical frameworks.

There are still many potential improvements in our approach. One
improvement clearly emphasized by this case study is the ability to
perform proofs with \emph{cooperation} between automatic provers and
interactive ones. Interactive provers are nice because they are very
expressive and allow to prove manually difficult goals, but their
automatic tactics are less powerful than those of automatic provers. This is
especially true for Coq, because any decision procedure in it needs to
provide a checkable proof trace, making its integration very
difficult. So one of the major goals of our research is now to develop
a platform for verification where automatic theorem proving will be
used as far as possible, and only when this fails the user would be asked
for manual interaction.

This case study was motivated by Bornat's assertion given at the
beginning. But the Schorr-Waite program is not representative of the
kind of C program we are really interested in. We focus in particular
on embedded industrial C programs, where a high level of confidence is
required: \caduceus{} is currently under experimentation at Axalto
(smart cards) and Dassault Aviation (aeronautics). For this goal,
several improvements to the \caduceus{} tool are in progress. There
are scaling up issues for dealing with code of large size. Automatic
annotation of programs is an important need (\`a la ESC/Java for
example). Properties of separation of data appear to be crucial for
large programs. We are currently investigating solutions, in
particular by integration of advanced static analysis techniques, and
ideas from separation logic~\cite{reynolds02lics}. Generally speaking
adding static analysis techniques to our setting is a major goal.

Among more academic future work, we wonder what would be the next
``mountain to climb'' for pointer program verification? We think that
memory allocation and multi-procedure programs (where specification of
separation is crucial) are important issues. Significant challenges
include advanced graph algorithms (such as planarity test) and
Ukkonen's algorithm~\cite{ukkonen95}.



%%% Local Variables: 
%%% mode: latex
%%% TeX-master: "rapport"
%%% End: 
