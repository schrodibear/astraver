
\section{Verification}
\label{sec:proof}

The \verb|schorr_waite| C function is now formally specified. The hard
work begins: we need to prove that the implementation of
Figure~\ref{fig:code} indeed satisfies this specification. If one runs
the \caduceus{} tool on that annotated code, it generates verification
conditions that are actually not provable: as usual with a Hoare-like
system, when there are loops like the \verb|while| loop, it
is mandatory (except in simple cases) to manually add a \emph{loop
  invariant}, that is an assertion preserved by each iteration of the
loop. The design of a suitable loop invariant is the most difficult
part of this verification process, but suitable loop invariants were
already proposed.  However, some difficulties remain, the main one is
that we need to express this loop invariant in the \caduceus{} syntax,
which is limited to a first-order logic. Hence, the higher-order
specifications of~\cite{mehta03cade} are not allowed by \caduceus.
% Also, since we proved a little bit more than
% previous works, namely the absence of invalid pointer dereferencing,
% we needed to add new assertions in the loop invariant.

\subsection{Designing the loop invariant}
\label{sec:invariant}

\begin{figure}[t]
\vspace*{2mm}
\begin{center}
  \unitlength=0.5mm
  \psset{unit=\unitlength,arrows=->}
\begin{picture}(100,100)
\put(50,100){\rnode{null}{\makebox(0,0)[b]{\tiny NULL}}}
\put(80,83){\rnode{root}{\makebox(0,0)[lb]{root}}}
\put(50,80){\mnode{n0}}\put(58,83){\makebox(0,0)[b] {$c=0$}}
\put(30,60){\mnode{n1}}\put(23,63){\makebox(0,0)[b] {$c=1$}}
\put(60,60){\unode{n2}}

\put(20,40){\mnode{n11}}
\put(40,40){\mnode{n12}}\put(50,43){\makebox(0,0)[b] {$c=0$}}
\put(50,20){\unode{n122}}
\put(30,20){\unode{n121}}
\put(40,0){\unode{n1211}}
\put(20,0){\unode{n1212}}
\put(70,43){\rnode{p}{\makebox(0,0)[lb]{$p$}}}
\put(0,30){\rnode{t}{\makebox(0,0)[rb]{$t$}}}

\ncline[linestyle=dotted]{n0}{n1}
\ncarc[arcangleA=120,linewidth=1]{n0}{null}
\ncline{n0}{n2}
\ncline{n1}{n11}
\ncline{n12}{n122}
\ncline[linestyle=dotted]{n1}{n12}
\ncline[linestyle=dotted]{n12}{n121}
\ncarc[arcangleA=120,arcangleB=30,linewidth=1]{n12}{n1}
\ncarc[arcangleA=-100,arcangleB=-30,linewidth=1]{n1}{n0}
\ncline[nodesepA=1mm]{root}{n0}
\ncline{n121}{n1211}
\ncline{n121}{n1212}
\ncline[nodesepA=1mm,linewidth=1]{p}{n12}
\ncline[nodesepA=1mm,linewidth=1]{t}{n121}
\end{picture}\label{figure}
\end{center}
\caption{The backtracking stack inside the graph structure}
\label{fig:stack}
\end{figure}

The key idea for the loop invariant is that even if the graph
structure is temporarily modified, the reachability of nodes is
preserved. More precisely, each node that is initially reachable from
the root will be always reachable either from $p$ or from $t$. We also
need to describe as a loop invariant how the backtracking stack is
implemented using the pointers of the graph. On
Figure~\ref{fig:stack}, we display some state of the graph as it may
occur at an arbitrary iteration of the loop. The dotted arrows
correspond to the children pointers as they were at the beginning,
whereas the thick arrows show their current values. The thin arrows
correspond to unmodified links. The backtracking stack is indeed the
list of pointers, starting from $p$, following either the left
child when $c$ is 0, or the right child when $c$ is 1.



\begin{figure*}[tp]
\begin{alltt}
void schorr_waite(node root) \{
  node t = root; node p = NULL;\begin{slshape}
  /*@ invariant
    @  (I1 :: \bs{}forall node x; 
    @    \old(reachable(root,x)) => reachable(t,x) || reachable(p,x))
    @ &&
    @  (I2 :: \bs{}forall node x; x != \bs{}null => 
    @         (reachable(t,x) || reachable(p,x)) => \old(reachable(root,x)))
    @ &&
    @  (I3 :: \bs{}forall node x; 
    @       ! \old(reachable(root,x)) => x->m == \old(x->m)) 
    @ &&
    @  \bs{}exists plist stack;
    @    (I4a :: clr_list (p,stack)) 
    @  &&
    @    (I4b :: \bs{}forall node p; in_list (p,stack) => p->m) 
    @  &&
    @    (I4c :: \bs{}forall node x; \valid(x) && \old(reachable(root,x)) && !x->m =>
    @          unmarked_reachable(t,x) || 
    @          (\bs{}exists node y; in_list(y,stack) && unmarked_reachable(y->r,x))) 
    @  &&
    @    (I4d :: \bs{}forall node x; 
    @             !in_list(x,stack) => (x->r == \old(x->r) && x->l == \old(x->l))) 
    @  &&
    @    (I4e :: 
    @      \bs{}forall node p1; \bs{}forall node p2; 
    @        pair_in_list(p1,p2,cons(t,stack)) =>
    @            (p2->c => \old(p2->l) == p2->l && \old(p2->r) == p1)
    @          &&
    @            (!p2->c => \old(p2->l) == p1 && \old(p2->r) == p2->r)) 
    @*/\end{slshape}
  while (p != NULL || (t != NULL && ! t->m)) \{
    ...
\end{alltt}
\caption{The loop invariant}
\label{fig:loopinv}
\end{figure*}

The loop invariant as it is written in the \caduceus{} annotation
language is displayed on Figure~\ref{fig:loopinv}.  For specifying this
loop invariant, in particular the stack, we need to talk about lists
of pointers in annotations. With \caduceus{}, such a datatype from
the logical side can be imported, implicitly. Then new predicates and
logical function symbols can be declared, as the \verb|reachable|
predicate already introduced, which can take as arguments either C
types or logical types. Here we declare
\begin{alltt}\begin{slshape}
//@ logic plist cons(node p, plist l) 
//@ predicate in_list(node p, plist l)
/*@ predicate 
  @   pair_in_list(node p1, node p2,
  @                plist l) */\end{slshape}
\end{alltt}
which first introduces the logical type \verb|plist| for finite lists
of pointers, then declares a logical function symbol \verb|cons| which
is intended to add an
element to a list, a predicate \verb|in_list| which is supposed to
test whether a pointer belongs to a list, and finally a predicate
$\verb|pair_in_list|(p_1,p_2,l)$, saying that $p_1$ and $p_2$ are two
consecutive elements in a list $l$. As for \verb|reachable|, these are
only declarations, true definitions or axiomatizations of these
functions and predicates will be given later.

The loop invariant finally requires the declaration of two predicates:
\begin{alltt}\begin{slshape}
/*@ predicate 
  @   unmarked_reachable(node p1,node p2) 
  @   reads p1->r,p1->l,p1->m */\end{slshape}
\end{alltt}
specifies that $p_2$ is reachable from $p_1$ by traversing only
unmarked nodes; and
\begin{alltt}\begin{slshape}
/*@ predicate 
  @   clr_list(node p,plist stack) 
  @   reads p->c,p->l,p->r */\end{slshape}
\end{alltt}
specifies that \verb|stack| is the list of pointers obtained from $p$ as
described on Figure~\ref{fig:stack}. The precise definition of it will be given in Section~\ref{sec:coq}.

In Figure~\ref{fig:loopinv}, we use the \caduceus{} notation to name
formulas with double-colons. The loop
invariant is made of eight assertions:
\begin {enumerate}
\item[I1] means that each node which was initially reachable from
  root, is reachable from $t$ or $p$.
\item[I2] means that conversely, each (non-null) node reachable from $t$
  or $p$ was initially reachable from \verb|root|. This is true only
  for non-null nodes, because \verb|null| may be not reachable from
  \verb|root| if the graph is cyclic.
\item[I3] means that for initially unreachable nodes, the mark
  remains unchanged.
\item[I4a] defines the backtracking stack as on
  Figure~\ref{fig:stack}, using the \verb|clr_list| predicate declared
  before. 
\item[I4b] means that all nodes in the \texttt{stack} are marked.
\item[I4c] is the trickiest annotation: it means that
  any valid node, initially reachable from root, 
  and not yet marked, is reachable by traversing only
  unmarked nodes, either from $t$ or from the right child of a node
  in the stack.
\item[I4d] means that all nodes out of \texttt{stack} have
  the same children as initially. 
\item[I4e] is essential for being able to
  `recover' the correct children of nodes of the stack, as it can be
  seen again on Figure~\ref{fig:stack}: if $p_1$ is in
  the stack, followed by $p_2$ (or if $p_1$ is $t$ and $p_2$ is the head of the
  stack), then: if $p_2\verb|->c|$ is
  1 then its left child is the same as it was initially
     and its right child is $p_1$; and if $p_2\verb|->c|$ is 0 then
     its left child is $p_1$ and its right child is the same as it was
     initially. 
\end{enumerate}


\subsection{Verification using automatic provers}
\label{sec:simplify}

\begin{comment}

. invariant vrai init
. post cond
. 3 fois invariant preserve resp. pop swing push
. validit� des acc�s
. po1 = WF order
. po2 = validity of access to t->m in the condition of while 
. po3 = validity of access to t->m in the condition of the first if 
. po4 = validity of access to p->c in the condition of the second if 
. po5 = validity of t in 't->r = q' in the first branch of if 
. po6 = preservation of loop invariant for "pop" 
. po7 = validity of p in "p->r = p->l" in second branch of if (swing) 
. po8 = preservation of loop invariant for "swing" 
. po9 = validity of acces to t in the "push" branch
. po10 = validity of acces to p in the "push" branch
. po11 = preservation of loop invariant for "push" 
. po12 = dummy po due to variant
. po13 = post-condition of the function
. po14 = loop invariant true at the beginning 

with Simplify: 

.***.*.*..*.** (6/8/0)

-> 4/12 (po5,7,9 et 10)

ajout axiome reachable(p,p) : 

.....*.*..*.** (9/5/0)   

-> 7/12 (po2,3,4 en plus)

\end{comment}

Running \caduceus{} on the annotated source code leads to twelve
verification conditions. Five of them relate to the behavioral
specification of the algorithm: loop invariant true when entering the
loop, preservation of the loop invariant for each of the branches
`pop', `swing' and `push', and validity of the post-condition. The
seven others are required to establish the absence of invalid pointer
dereferencing: the first one asks for validity of \verb|t| for the
field access \verb|t->m| in the condition of the \verb|while|, and the
six others similarly ask for validity of dereferencing \verb|t->m| in
the first \verb|if|, \verb|p->c| in the second \verb|if|, \verb|t->r|
in the `pop' branch, \verb|p->l| in the `swing' branch, \verb|t->l|
and \verb|p->l| in the `push' branch. Notice that some accesses do not
require a validity check, because they are trivial enough to be
automatically discharged, for example \verb|p->r| in the `pop` branch
because it was preceeded by \verb|p->c| and \verb|p| did not change.

A first attempt can be made to solve these obligations with an
automatic prover. Here we present the results with Simplify, which appears
to be the best in that case. Four of the twelve obligations are
solved automatically, which are the validity of the dereferencings
occuring in the `pop', `swing' and `push' branches. Indeed, a quick
look at those obligations shows that they can be solved by equality
reasoning: for example the \verb|t->r| access in the `pop' branch is
valid because \verb|t| is equal to the old value of \verb|p| in the
condition of the if, which is already assumed valid. It is not
surprising at all that Simplify does not solve the other obligations,
because until now we did not give any definition or axiomatization for
the introduced predicates, in particular reachability. Indeed, to
establish for example validity of \verb|t->m| in the condition of the
\verb|while|, one needs to use the function's pre-condition saying
that any non-null initially reachable node is valid. To establish
that \verb|t| is initially reachable, one can use the \verb|I1|
assertion of the loop invariant, as soon as one knows the simple fact
that \verb|t| is reachable from itself! So one thing that can be done
is to add in the source an axiom for \verb|reachable|:
\begin{alltt}\begin{slshape}
/*@ axiom reachable_refl : 
  @   \bs{}forall node p ; reachable(p,p) */
\end{slshape}\end{alltt}
Indeed with that new axiom, Simplify automatically solves the three
remaining obligations for validity of pointer dereferencing: we have
certified the absence of threat. 

At this point, any of our attempts to add axioms for solving the
remaining obligations failed. We also tried our other supported
automatic provers haRVey and CVC-lite, but they are even worse than
Simplify. These automated provers are of course incomplete, and the
obligations we have are apparently too difficult for them, so we
switched to an interactive prover: the Coq proof assistant.

\subsection{Verification with the Coq interactive prover}
\label{sec:coq}

\begin{figure*}[t]
\begin{alltt}\begin{slshape}
/*@ logic Length weight(node p, node t) 
                   reads p->m,p->c,p->l,p->r  */
/*@ predicate reachable_elements(node root, plist s) 
                   reads root->l,root->r */ 
/*@ requires 
       \bs{}exists plist s; reachable_elements(root,s) && 
     ... */\end{slshape}
void schorr_waite(node root) \{
  node t = root; node p = NULL;\begin{slshape}
  /*@ invariant ...
    @ variant weight(p,t) for order_mark_m_and_c_and_stack 
    @*/\end{slshape}
  while (p != NULL || (t != NULL && ! t->m)) \{
   ...
\end{alltt}
\vspace*{-2mm}
\caption{Additional annotations for termination}
\label{fig:term}
\end{figure*}

Unlike with Simplify, we have now all the power of the Coq
specification language; in particular we do not need to axiomatize
predicates in the C source file, but we can \emph{define} them, and in
particular for reachability, it is natural to use an inductive
definition.

Due to lack of space, we do not present any Coq code here, but only
approximations to ease reading. The whole development can be found
from the \caduceus{} web page~\cite{Caduceus}. The first step is to
define the reachability predicate in Coq. We proceed in a slightly
different way than Mehta and Nipkow~\cite{mehta03cade}, who use a
set-theoretic axiomatization: we use inductively defined predicates,
which are more suitable for reasoning in Coq.  We first introduce a
more general predicate \verb|path| which also makes precise the path
between nodes. It is defined by three clauses:
\[
\begin{array}{l}
\verb|path|(a,l,r, p, p, nil).  \\
\verb|path|(a,l,r,p_1,p_2,cons(p_1,s)) \verb|:-| \\
  \quad\verb|valid|(a,p_1), \verb|path|(a,select(l,p_1),p_2,s) \\
\verb|path|(a,l,r,p_1,p_2,cons(p_1,s)) \verb|:-| \\
  \quad\verb|valid|(a,p_1), \verb|path|(a,select(r,p_1),p_2,s)
\end{array}
\]
the first one says that one may go from a node to itself by an empty
path, the second (resp. the third) say that if there is a path $s$
from the left (resp. right) child of $p_1$ to $p_2$ then $cons(p_1,s)$
is a path from $p_1$ to $p_2$. Then \verb|reachable| is defined by the
existence of a path (resp. \verb|unmarked_reachable| by a path with
unmarked nodes):
\[
\begin{array}{l}
\verb|reachable|(a,l,r, p_1, p_2) := \exists s,~\verb|path|(a,l,r, p_1,
p_2, s) \\
\verb|unmarked_reachable|(a,m,l,r, p_1, p_2) := \\
\qquad \exists s,~\verb|path|(a,l,r, p_1,p_2, s) \land \\
\qquad\qquad \forall p, p \in s \rightarrow select(m,p) = 0 \\
\end{array} 
\]
Notice that it is now easy to establish the validity of the axiom
\verb|reachable_refl|. It may be surprising at first that the
\verb|reachable| predicate 
defined here has five arguments instead of two, as declared in
Section~\ref{sec:spec}. This is the effect of the \texttt{reads}
clause: as already said,
this is due to the component-as-array modeling: the
\verb|reachable| predicate has two arguments in C, but in the modeling
it is given extras arguments $a$, $l$, $r$ which are arrays
indexed by pointers (with $select$ as their access function). $a$
corresponds to an allocation table which tells for each pointer
whether it is allocated or not, and $l$, $r$ correspond to
fields of the \verb|node| structure. It is clear that to perform the
verification of a C program with \caduceus{} and an interactive prover
as back-end, one needs to learn more on this
modeling~\cite{Caduceus,filliatre04icfem}.

The next step is to deal with the type \verb|plist| used to model the
stack: we simply use the finite list datatype of the Coq standard
library. The \verb|clr_list| predicate can be defined inductively with
the Prolog-like clauses:
\[
\begin{array}{l}
\verb|clr_list|(a,c,l,r, p, nil).  \\
\verb|clr_list|(a,c,l,r,p,cons(p,s)) \verb|:-| \\
\qquad\qquad \verb|valid|(a,p), \verb|clr_list|(a,next(c,l,r,p),s).
\end{array}
\]
where 
\[
\begin{array}{l}
next(c,l,r,p) := \verb|if|~select(c,p) = 0 ~\verb|then| ~ select(l,p)\\
\qquad\qquad\qquad\qquad \verb|else|~select(r,p)
\end{array}
\]
A few lemmas are proved before proving the obligations, the main one,
reused several times in the proofs, is to show that when $p_2$ is
reachable from $p_1$, there is a path which has no cycle:
\[
\begin{array}{l}
\forall a, l, r, p_1, p_2, s,~\verb|path|(a,l,r,p_1,p_2,s) \rightarrow
 \\\qquad
  \exists s', s' \subseteq s \land \verb|no_rep|(s') \land
  \verb|path|(a,l,r,p_1, p_2, s') 
\end{array}
\] 
where \verb|no_rep| is a simple predicate on a list which says that no
element occurs twice. 

With those definitions and lemmas, we are able to manually
complete all the proof obligations in Coq. We give a few numbers
at the end of the next section.
% after talking about the last point: termination.





\subsection{Termination}
\label{sec:termination}

Proving the termination of the algorithm is proving the
termination of the while loop. In \caduceus{}, one can annotate a
loop by a \emph{variant}: an expression which decreases with respect
to some well-founded ordering. It is not hard to find such a
measure for the Schorr-Waite algorithm. There is a triple of natural
numbers which decreases lexicographically at each iteration:
%\begin{enumerate}
%\item 
the number of unmarked reachable nodes;
%\item 
the number of reachable nodes which have $c=0$;
%\item 
the length of the stack.
%\end{enumerate}
There is a minor issue: in our modeling, there is no assumption on
the finiteness of the memory, so indeed nothing says that the set of
reachable nodes is finite. So this is an extra assumption we have
to insert in the pre-condition. The additional annotations for
termination are given in Figure~\ref{fig:term}, where in Coq we define
\[
\begin{array}{l}
\verb|reachable_elements|(a,l,r, root, s) := \\ 
\qquad\qquad\qquad \forall p, p \in s \leftrightarrow \verb|reachable|(a,l,r,root,p) 
\end{array}
\]
The \verb|weight| function and the \url{order_mark_m_and_c_and_stack}
relation are also defined in Coq according to the informal definition of
the measure above. 

With the completed annotated code, \caduceus{} generates two
additional verification conditions. The first is to show that the ordering is
well-founded, whereas the second is a specificity of the
\caduceus{} verification condition generator, trivially solved by
equality reasoning. The verification that the measure decreases at
each iteration is added to the obligations corresponding to
preservation of the loop invariant. 

The well-foundedness obligation is not passed to automatic provers
like Simplify because it is a second-order formula. So finally, the
Simplify prover solves 8 of the 13 obligations, which is a quite good
score but of course the 5 remaining obligations are by far the most
difficult.

With Coq, we completed all the obligations including the
well-foundedness. Here are a few numbers about the whole Coq proof: 
definition of predicates on lists, reachability and such, and
corresponding lemmas amount to 317 lines of definitions and 589 lines
of proof tactics. The generated verification conditions amount to 985
lines, and 2411 lines of tactics were inserted manually to prove
them. Approximately 40\% of the work was devoted to the termination proof,
both in terms of lines of Coq text and of time spent: proving the
obligations without termination required around 3 weeks, and the
termination proof required 2 more weeks. These times include several
iterations of the process of modifying the annotations, regenerating
the verification conditions and modifications of the proof scripts.

These numbers may seem quite big for a program of only 10 lines of C
code. But of course this algorithm is especially clever, and it should
not be surprising that its justification is hard. Another reason is
that we had to define in Coq many preliminary definitions and lemmas
(regarding reachability) from scratch: we could have avoided this if
Coq had an existing rich library for graphs. Indeed, some of our
preliminaries will be included in the next version of Coq. Compared to
the similar proof in Isabelle/HOL~\cite{mehta03cade}, we have roughly
three times more lines of proofs: we believe that this is due to the
weakness of automatic tactics of Coq (but also remember that we have
the additional proofs of termination and absence of threats).


%%% Local Variables: 
%%% mode: latex
%%% TeX-master: "rapport"
%%% End: 
