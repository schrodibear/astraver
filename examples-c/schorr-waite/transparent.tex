\documentclass[pdf,proval,slideColor,colorBG,total]{prosper}

\usepackage{alltt}
\usepackage[T1]{fontenc}
\usepackage[latin1]{inputenc}
\usepackage{pstricks}
\usepackage{pst-node}
\usepackage{color}
\usepackage{url}


%%% Local Variables: 
%%% mode: latex
%%% TeX-master: "rapport"
%%% End: 


\pagestyle{empty}

\newrgbcolor{lightred}{1 0.8 0.8}
\newrgbcolor{lightblue}{0.8 0.8 1}
\newrgbcolor{whycolor}{0 0.5 1}
\newrgbcolor{coqcolor}{1 0.5 0}


\begin{document}
%\sffamily
%\parindent=0mm

\title{Certification de l'algorithme de
  Schorr-Waite avec Caduceus}
\author{Thierry Hubert}
\institution{Projet PROVAL de l'INRIA-equipe DEMONS
  du LRI - Orsay}

\slideCaption{S�minaire LogiCal-ProVal --- 11 mars 2005}

\maketitle





\begin{slide}{Plan de cet expos�}

\begin{enumerate}
\item Motivations
\item L'algorithme de Schorr-Waite
\item Sa sp�cification formelle en Caduceus
\item Annotation du code 
\item la preuve Simplify et en Coq
\item Terminaison
\end{enumerate}

\end{slide}


\begin{slide}{Motivations}

\begin{slshape}
The Schorr-Waite algorithm is the first moutain that any formalism for
pointer aliasing should climb.
\end{slshape}

--- Richard Bornat (Proving pointer programs in {H}oare logic, In {\em
  Mathematics of Program Construction}, 2000, page 121)

\begin{itemize}
\item 2000, Bornat, syst�me Jape
\item 2003, Mehta \& Nipkow, syst�me Isabelle/HOL
\item 2004, Abrial, syst�me Atelier B
\end{itemize}

\end{slide}

\begin{slide}{Originalit� de la preuve avec Caduceus}

\begin{itemize}
\item Directement sur du code source en C
\item Preuve de l'absence de ``segmentation fault'' 
\item Partie inaccessible montr�e inchang�e
\item Preuve formelle de terminaison
\end{itemize}

\end{slide}


\begin{slide}{L'algorithme de Schorr-Waite}

Parcours d'un graphe orient� (binaire) 
\begin{itemize}
\item en profondeur
\item sans utiliser d'allocation memoire
\end{itemize}

Noeuds du graphe cod�s par structure C :
\fontsize{10}{10pt}
\begin{alltt}
typedef struct struct_node \{
  unsigned int m :1;         /* marquage */
  unsigned int c :1;         /* usage interne */
  struct struct_node *l, *r; /* fils gauche et droit */
\} * node;

void schorr_waite(node root);
\end{alltt}
\end{slide}




\begin{slide}{L'algorithme de Schorr-Waite en C}
\fontsize{10}{10pt}
\begin{alltt}
void schorr_waite(node root) \{
  node t = root; node p = NULL;
  while (p != NULL || (t != NULL && ! t->m)) \{
    if (t == NULL || t->m) \{
      if (p->c) \{ \begin{slshape}/* \textbf{pop} */\end{slshape}
        node q = t; t = p; p = p->r; t->r = q;
      \} 
      else \{ \begin{slshape}/* \textbf{swing} */\end{slshape}
        node q = t; t = p->r; p->r = p->l; 
        p->l = q; p->c = 1;
      \}
    \} 
    else \{ \begin{slshape}/* \textbf{push} */ \end{slshape}
      node q = p; p = t; t = t->l; p->l = q; 
      p->m = 1; p->c = 0;
\}\}\}
\end{alltt}
\end{slide}

\overlays{6}{
\begin{slide}{Exemple d'ex�cution}

\begin{center}
  \unitlength=0.5mm
  \psset{unit=\unitlength,arrows=->}
%\fbox{
\begin{picture}(100,100)
\put(50,100){\rnode{null}{\makebox(0,0)[l]{\tiny NULL}}}
\untilSlide*{1}{\put(50,80){\unode{n0}}}
\fromSlide*{2}{\put(50,80){\mnode{n0}}}
\untilSlide*{2}{\put(20,50){\unode{n1}}}
\fromSlide*{3}{\put(20,50){\mnode{n1}}}
\put(80,50){\unode{n2}}
\untilSlide*{1}{\put(30,100){\rnode{p}{\makebox(0,0)[r]{$p$}}}}
\untilSlide*{1}{\put(30,80){\rnode{t}{\makebox(0,0)[r]{$t$}}}}
\put(0,20){\rnode{null2}{\makebox(0,0)[c]{\tiny NULL}}}
\put(40,20){\rnode{null3}{\makebox(0,0)[c]{\tiny NULL}}}
\put(60,20){\rnode{n11}}
\put(100,20){\rnode{n12}}
\ncline[linestyle=dotted]{-}{n2}{n11}
\ncline[linestyle=dotted]{-}{n2}{n12}
\untilSlide*{1}{\ncline{n0}{n1}}
\untilSlide*{5}{\ncline{n0}{n2}}
\untilSlide*{2}{\ncline[nodesepB=1mm]{n1}{null2}}
\untilSlide*{3}{\ncline[nodesepB=1mm]{n1}{null3}}
\onlySlide*{1}{\ncline[nodesep=1mm]{p}{null}}
\onlySlide*{1}{\ncline[nodesepA=1mm]{t}{n0}}
\onlySlide*{2}{\put(80,100){\rnode{p}{\makebox(0,0)[r]{$p$}}}}
\onlySlide*{2}{\put(0,50){\rnode{t}{\makebox(0,0)[r]{$t$}}}}
\onlySlide*{2}{\ncline[nodesepA=3mm]{p}{n0}}
\onlySlide*{2}{\ncline[nodesepA=1mm]{t}{n1}}
\onlySlide*{2}{\ncarc[nodesepB=1mm,arcangleA=130,linewidth=1]{n0}{null}}
\onlySlide*{3}{\put(0,70){\rnode{p}{\makebox(0,0)[r]{$p$}}}}
\onlySlide*{3}{\put(-10,30){\rnode{t}{\makebox(0,0)[r]{$t$}}}}
\onlySlide*{3}{\ncline[nodesepA=3mm]{p}{n1}}
\onlySlide*{3}{\ncline[nodesepB=1mm]{t}{null2}}
\onlySlide*{3}{\ncarc[nodesepB=1mm,arcangleA=130,linewidth=1]{n0}{null}}
\onlySlide*{3}{\ncarc[arcangleA=150,arcangleB=-20,linewidth=1]{n1}{n0}}
\onlySlide*{4}{\put(0,70){\rnode{p}{\makebox(0,0)[r]{$p$}}}}
\onlySlide*{4}{\put(20,-10){\rnode{t}{\makebox(0,0)[r]{$t$}}}}
\onlySlide*{4}{\ncline[nodesepA=3mm]{p}{n1}}
\onlySlide*{4}{\ncline[nodesepB=1mm]{t}{null3}}
\onlySlide*{4}{\ncarc[nodesepB=1mm,arcangleA=130,linewidth=1]{n0}{null}}
\onlySlide*{4}{\ncarc[arcangleA=-120,arcangleB=-40,linewidth=1]{n1}{n0}}
\fromSlide*{4}{\ncline[nodesepB=1mm]{n1}{null2}}
\onlySlide*{5}{\put(60,100){\rnode{p}{\makebox(0,0)[r]{$p$}}}}
\onlySlide*{5}{\put(-50,70){\rnode{t}{\makebox(0,0)[r]{$t$}}}}
\onlySlide*{5}{\ncline[nodesepA=3mm]{p}{n0}}
\onlySlide*{5}{\ncline[nodesepB=1mm]{t}{n1}}
\onlySlide*{5}{\ncarc[nodesepB=1mm,arcangleA=130,linewidth=1]{n0}{null}}
\fromSlide*{5}{\ncline[nodesepB=1mm]{n1}{null3}}
\onlySlide*{6}{\put(-50,70){\rnode{p}{\makebox(0,0)[r]{$p$}}}}
\onlySlide*{6}{\put(70,70){\rnode{t}{\makebox(0,0)[r]{$t$}}}}
\onlySlide*{6}{\ncline[nodesepA=1mm]{p}{n0}}
\onlySlide*{6}{\ncline[nodesepA=2mm]{t}{n2}}
\onlySlide*{6}{\ncarc[nodesepB=1mm,arcangleA=-120,arcangleB=-20,linewidth=1]{n0}{null}}
\fromSlide*{6}{\ncline[nodesepB=1mm]{n0}{n1}}
\end{picture}
%}

\end{center}

\end{slide}}


\begin{slide}{Sp�cification informelle}

\begin{alltt}
void schorr_waite(node root);
\end{alltt}

\begin{itemize}
\item Les noeuds accessibles � partir de root sont non-marqu�s
initialement (n->m == 0) et marqu�s � la fin (n->m == 1)

\item La structure du graphe est restaur�e (pointeurs des fils gauche et
droit restaur�s � leurs valeurs initiales)
\end{itemize}
\end{slide}

\begin{slide}{Sp�cification formelle en Caduceus}

Il faut exprimer l'accessibilit�. On introduit
\fontsize{10}{10pt}
\begin{alltt}\begin{slshape}\begin{blue}
/*@ predicate reachable (node p1, node p2) 
  @   reads p1->l,p1->r */
\end{blue}\end{slshape}\end{alltt}
\begin{itemize}
\item d�clar� en Caduceus
\item suppos� d�fini (ou du moins axiomatis�) dans le prouveur.
\item declaration \verb|reads|: permet de faire dependre ce predicat
  de p1 (->l \cup ->r)
\end{itemize}

\end{slide}

\begin{slide}{Annotation de la fonction}

\fontsize{10}{10pt}
\begin{alltt}\begin{slshape}\begin{blue}
/*@ \textbf{requires}
  @  \bs{}forall node x; 
  @     x != \bs{}null && reachable(root,x) => \valid(x) && ! x->m  
  @ \textbf{ensures}
  @  (\bs{}forall node x; \old(x->l) == x->l && \old(x->r) == x->r) 
  @ &&
  @  (\bs{}forall node x; x != \bs{}null && reachable(root,x) => x->m) 
  @ &&
  @  (\bs{}forall node x; ! reachable(root,x) => x->m == \old(x->m))
  @*/\end{blue}\end{slshape}
void schorr_waite(node root);
\end{alltt}
\end{slide}


\begin{slide}{Annotation du code}

cad trouver le bon invariant de boucle.

idee essentielle: codage d'une pile de backtracking dans le graphe

\end{slide}

\begin{slide}{La pile de backtrack}
\begin{center}
  \unitlength=0.5mm
  \psset{unit=\unitlength,arrows=->}
\begin{picture}(100,100)
\put(50,100){\rnode{null}{\makebox(0,0)[b]{\tiny NULL}}}
\put(80,83){\rnode{root}{\makebox(0,0)[lb]{root}}}
\put(50,80){\mnode{n0}}\put(58,83){\makebox(0,0)[b] {$c=0$}}
\put(30,60){\mnode{n1}}\put(23,63){\makebox(0,0)[b] {$c=1$}}
\put(60,60){\unode{n2}}

\put(20,40){\mnode{n11}}
\put(40,40){\mnode{n12}}\put(50,43){\makebox(0,0)[b] {$c=0$}}
\put(50,20){\unode{n122}}
\put(30,20){\unode{n121}}
\put(40,0){\unode{n1211}}
\put(20,0){\unode{n1212}}
\put(70,43){\rnode{p}{\makebox(0,0)[lb]{$p$}}}
\put(0,30){\rnode{t}{\makebox(0,0)[rb]{$t$}}}

\ncline[linestyle=dotted]{n0}{n1}
\ncarc[arcangleA=120,linewidth=1]{n0}{null}
\ncline{n0}{n2}
\ncline{n1}{n11}
\ncline{n12}{n122}
\ncline[linestyle=dotted]{n1}{n12}
\ncline[linestyle=dotted]{n12}{n121}
\ncarc[arcangleA=120,arcangleB=30,linewidth=1]{n12}{n1}
\ncarc[arcangleA=-100,arcangleB=-30,linewidth=1]{n1}{n0}
\ncline[nodesepA=1mm]{root}{n0}
\ncline{n121}{n1211}
\ncline{n121}{n1212}
\ncline[nodesepA=1mm]{p}{n12}
\ncline[nodesepA=1mm]{t}{n121}
\end{picture}\label{figure}
\end{center}

\end{slide}
\begin{slide}{Invariant de la boucle}
\fontsize{10pt}{9pt}
\begin{alltt}
/*@ invariant
    @  (I1 :: \bs{}forall node x; \old(reachable(root,x)) => 
    @    reachable(t,x) || reachable(p,x))
    @ &&

\end{alltt}
Ce premier invariant indique que tous les noeuds dans le graphe au
debut de la fonction sont atteignable par t ou p.
\fontsize{10pt}{9pt}
\begin{alltt}
(I2 :: \bs{}forall node x; x != \bs{}null => 
    @         (reachable(t,x) || reachable(p,x)) => 
    @         \old(reachable(root,x)))
    @ &&
\end{alltt}
Ce deuxi�me invariant indique que tous les noeuds non null atteignable
par t ou p etaient dans le graphe avant la fonction.
\end{slide}
\begin{slide}{Invariant de la boucle}
\fontsize{10pt}{9pt}
\begin{alltt}
(I3 :: \bs{}forall node x; ! \old(reachable(root,x)) => 
    @         x->m == \old(x->m)) 
    @ &&
\end{alltt}
Cet invariant indique que tous les noeuds qui n'etait pas dans le
graphe avant la fonction, ont leur marquage inchanger.
\fontsize{10pt}{9pt}
\begin{alltt}
 \bs{}exists plist stack;
    @  (I4a :: clr_list (p,stack))
    &&
\end{alltt}
Cet invariant permet de definir la pile de backtrack.
La declaration de clr_list dans caduceus est :
\fontsize{10pt}{9pt}
\begin{alltt}
/*@ predicate clr_list(node p, plist stack)
  @   reads p->c,p->l,p->r
  @*/
\end{alltt}
\end{slide}
\begin{slide}{Invariant de la boucle}
\fontsize{10pt}{9pt}
\begin{alltt}
(I4b :: \bs{}forall node p; in_list (p,stack) => p->m)  &&
\end{alltt}
Cet invariant precise que tous les noeuds dans la pile sont marqu�s.
La declaration de in_list dans caduceus est :
\fontsize{10pt}{9pt}
\begin{alltt}
/*@ predicate in_list(node p,plist stack) */
\end{alltt}

\fontsize{10pt}{9pt}
\begin{alltt}
(I4c :: \bs{}forall node x; 
    @          \valid(x) && \old(reachable(root,x)) && 
    @          !x->m =>
    @          unmarked_reachable(t,x) || 
    @          (\bs{}exists node y; 
    @            in_list(y,stack) && 
    @          unmarked_reachable(y->r,x))) 
    @  &&
\end{alltt}
Cet invariant precise que tous les noeuds non marqu�s qui etaient
valid et dans le graphe au debut de la fonction, sont atteignable par
un chemin non marqu�s par t ou atteignable par un chemin non marqu�s
par un des fils droit des noeuds de la pile de backtrack.
La declaration de unmarked_reachable dans caduceus est :
\fontsize{10pt}{9pt}
\begin{alltt}
/*@ predicate unmarked_reachable(node p1, node p2)
  @   reads p1->r,p1->l,p1->m
  @*/
\end{alltt}

\end{slide}
\begin{slide}{Invariant de la boucle}
\fontsize{10pt}{9pt}
\begin{alltt}
(I4d ::\bs{}forall node x; !in_list(x,stack) =>  
    @        (x->r == \old(x->r) && x->l == \old(x->l))) 
    @  &&
\end{alltt}
Cet invariant precise que tous les noeuds hors de la pile de backtrack
ont leurs fils inchang�s.
\fontsize{10pt}{9pt}
\begin{alltt}
    @ (I4e :: \bs{}forall node p1; \bs{}forall node p2;
    @          pair_in_list(p1,p2,cons(t,stack)) => 
    @          (p2->c => \old(p2->l) == p2->l 
    @                    && \old(p2->r) == p1)
    @          &&(!p2->c => \old(p2->l) == p1 
    @                    && \old(p2->r) == p2->r)) */
\end{alltt}
Enfin cet invariant precise que pour tous les elements de la pile de
backtrack, suivant la valeur de c soit son fils gauche soit son fils
droit est inchang� et leur fils changer est maintenant la pointeur
qui les prec�de dans la pile ou bien t si ce pointer est en t�te de la pile.
La declaration de pair_in_list dans caduceus est :
\fontsize{10pt}{9pt}
\begin{alltt}
/*@ predicate pair_in_list(node p1, node p2, plist stack) */
\end{alltt}

\end{slide}
\begin{slide}{La preuves Simplify}
\begin{comment}
. invariant vrai init
. poste condition
. 3 fois invariant preserve resp. pop swing push
. validite des acces
. po1 = WF order
. po2 = validite de l'acces a t->m dans la boucle while 
. po3 = validite de l'acces a t->m dans la premiere condition du if 
. po4 = validite de l'acces a  p->c dans la seconde condition du if 
. po5 = validite de t dans 't->r = q' dans la premiere branche du if 
. po6 = preservation de l'invariant de boucle pour le ``pop'' 
. po7 = validite de p dans ``p->r = p->l'' dans la seconde branche du
  if (swing) 
. po8 = preservation de l'invariant de boucle pour le ``swing'' 
. po9 = validite de l'acces a  t dans la branche ``push'' 
. po10 = validite de l'acces a p dans la branche ``push'' 
. po11 = preservation de l'invariant de boucle pour le ``push'' 
. po12 = po du a la terminaison
. po13 = post-condition de la fonction
. po14 = l'invariant vrai au debut de la boucle 

avec Simplify: 

.***.*.*..*.** (6/8/0)

-> 4/12 (po5,7,9 et 10)

ajout axiome reachable(p,p) : 

.....*.*..*.** (9/5/0)   

-> 7/12 (po2,3,4 en plus)

\end{comment}

. invariant vrai init
. post cond
. 3 fois invariant preserve resp. pop swing push
. validit	 des acc
\end{slide}
\begin{slide}{La preuve Coq}
Nous devons d''abord definir les predicates utilis�s par caduceus.
\fontsize{10pt}{9pt}
\begin{alltt}
  Definition llist (a: alloc_table) (next : pointer -> pointer)
  (p:pointer) (l:plist) :=
  lpath a next p l null.

  Definition clr_list (a: alloc_table)  (c:memory Z) 
  (l: memory pointer)(r: memory pointer) : 
  pointer ->list pointer-> Prop :=
  let next t := if Z_eq_dec (acc c t) 0 then (acc l t) 
  else (acc r t) in
  llist a next .
\end{alltt}
Voici donc la d	finition Coq de la pile de backtrack.
\end{slide}
\begin{slide}{La preuve Coq}
\fontsize{10pt}{9pt}
\begin{alltt}
  Inductive path (a: alloc_table) (l: memory pointer)
  (r: memory pointer) : 
  pointer ->  pointer -> list pointer -> Prop :=
    | Path_null : forall p:pointer, path a l r p  p nil
    | Path_left :
      forall p1 p2:pointer,
        forall lp : list pointer,
          valid a p1 ->
            path a l r (acc l p1) p2 lp-> 
               path a l r p1  p2 (p1::lp)
    | Path_right :
      	forall p1 p2:pointer,
          forall lp : list pointer ,
        	valid a p1 ->
        	  path a l r (acc r p1)  p2 lp -> 
                     path a l r p1 p2 (p1::lp). 
\end{alltt}
Ce predicat represente le chemin entre 2 noeuds du graphe.
\end{slide}
\begin{slide}{La preuve Coq}
\fontsize{10pt}{9pt}
\begin{alltt}     
  Definition reachable (a: alloc_table) 
    (l: memory pointer)(r: memory pointer) 
    (p1 :pointer) (p2:pointer) : Prop :=
    exists lp : list pointer, path a l r p1 p2 lp.
\end{alltt}
Ce predicat represente l'accesibilite d'un noeuds a un autre dans le graphe.
\end{slide}
\begin{slide}{La preuve Coq}
\fontsize{10pt}{9pt}
\begin{alltt}
Definition unmarked_reachable (a: alloc_table) 
  (m:memory Z) (l r: memory pointer) (p1 p2:pointer) 
  : Prop :=
  exists lp : list pointer, 
    (forall x : pointer, In x lp -> (acc m x) = 0 ) 
    /\ path a l r p1 p2 lp.
\end{alltt}
Ce predicat represente l'accesibilite par des noeuds non marquer d'un noeuds a un autre dans le graphe.
\end{slide}
\begin{slide}{La preuve Coq}
\fontsize{10pt}{9pt}
\begin{alltt}
Lemma path_no_cycle : forall (a : alloc_table) 
  (p1 p2 : pointer) (l r : memory pointer) 
  (pa : list pointer), path a l r p1 p2 pa -> 
  exists pa' :list pointer, 
  incl pa' pa /\ no_rep pa' /\ path a l r p1 p2 pa'.   
\end{alltt}
Ce lemme affirme que s'y il existe un chemin de p1 a p2, alors il
existe un chemin de p1 a p2 sans repetition.
\end{slide}
\begin{slide}{La preuve Coq}
\fontsize{10pt}{9pt}
\begin{alltt}
Lemma path_upd_left : forall (alloc : alloc_table) 
  (l r : memory pointer) (p:pointer)
  (lp : list pointer) (p1 p0 p2 : pointer), 
  ~ In p lp -> path alloc l r p1 p0 lp -> 
  path alloc (upd l p p2) r p1 p0 lp.
\end{alltt}
Les quatres lemmes \texttt{path\_upd\_left}, \texttt{path\_upd\_right},
\texttt{path\_inv\_upd\_left} et \texttt{path\_inv\_upd\_right}
affirme que si il y a un chemin entre p1 et p2 et que l'ont modifie le
graphe en dehors de ce chemin alors ce chemin existe toujours. 
\end{slide}
\begin{slide}{La preuve de terminaison}
Pour prouver la terminaison il faut monter la terminaison de la boucle
while. Pour cela en caduceus on annote la boucle avec \emph{variant}:
une expression qui decroit a chaque tour de la boucle.
La mesure qui decroit strictement pour cette boucle est: 
\begin{enumerate}
\item le nombre de noeud non marquer.
\item le nombre de noeuds qui ont $c=0$
\item la taille de la pile
\end{enumerate}
\end{slide}
\begin{slide}{La preuve de terminaison}
Pour la preuve de terminaison on doit rajouter les elements suivant a
l'annotation du code.
\fontsize{10pt}{9pt}
\begin{alltt}
/*@ logic Length weight(node p, node t) 
            reads p->m,p->c,p->l,p->r  */
/*@ predicate reachable_elements(node root, plist s) 
            reads root->l,root->r */ 
/*@ requires 
       \bs{}exists plist s; reachable_elements(root,s) && 
     ... */\end{slshape}
void schorr_waite(node root) \{
  node t = root; node p = NULL;\begin{slshape}
  /*@ invariant ...
    @ variant weight(p,t) for order_mark_m_and_c_and_stack 
    @*/\end{slshape}
  while (p != NULL || (t != NULL && ! t->m)) \{
   ...
\end{alltt}
ou en Coq on definit :
\[
\begin{array}{l}
\verb|reachable_elements|(a,l,r, root, s) = %\\\qquad\qquad\qquad 
\forall p, p \in s \leftrightarrow \verb|reachable|(a,l,r,root,p) 
\end{array}
\]
Les fontions \verb|weight| et \verb|order_mark_m_and_c_and_stack| sont
la definition de la mesure precedente.

\end{slide}
\begin{slide}{Conclusion}
On a prouver different aspect de l'algorithme de schorrWaite : 
\begin{enumerate}
\item cette algorithme est bien un aprcours de graphe
\item il ne peu y avoir de ``segmentation fault''
\item que les parties hors du graphes ne sont pas changees
\item la terminaison
\end{enumerate}

De plus contrairement aux autres travaus, cette preuve a ete fait sur un vrai programme C.


\end{slide}
\end{document}




%%% Local Variables: 
%%% mode: latex
%%% TeX-master: t
%%% End: