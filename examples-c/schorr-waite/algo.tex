
\section{The Schorr-Waite algorithm and its formal specification}
\label{sec:algo}

The Schorr-Waite algorithm performs a depth-first traversal of a
directed graph, starting from a specific node of the graph called the
root. 

\subsection{The C source code}

In the version we consider here, we assume that the nodes have
at most two children, which is situation for example for a garbage
collector for a pure lisp interpreter, where the only memory allocated value
is the \textsf{cons} of lists. In C, these \textsf{cons} structures
can be defined as
\begin{alltt}
typedef struct struct_node \{
  unsigned int m :1;
  unsigned int c :1;
  struct struct_node *l, *r;
\} * node;
\end{alltt}
where $l$ and $r$ are respectively pointers to the left and to the
right child (they may be assigned to NULL). The boolean field $m$ is a
mark: initially, all nodes of the graph will be assumed unmarked, and
at the end of the traversal, they will have to be all marked. The
boolean field $c$ is used internally by the algorithm, to denote which
of the children is currently explored.
 
\begin{figure}[t]
\begin{alltt}
void schorr_waite(node root) \{
  node t = root;
  node p = NULL;
  while (p != NULL || (t != NULL && ! t->m)) \{
    if (t == NULL || t->m) \{
      if (p->c) \{ \begin{slshape}/* pop */\end{slshape}
        node q = t;
        t = p;
        p = p->r;
        t->r = q;
        \} else \{ \begin{slshape}/* swing */\end{slshape}
        node q = t;
        t = p->r;
        p->r = p->l;
        p->l = q;
        p->c = 1;
      \}
    \} else \{ \begin{slshape}/* push */ \end{slshape}
      node q = p;
      p = t;
      t = t->l;
      p->l = q;
      p->m = 1;
      p->c = 0;
    \}
  \}
\}
\end{alltt}
\caption{C version of the Schorr-Waite algorithm}
\label{fig:code}
\end{figure}

The Schorr-Waite graph-marking algorithm, written in C source code, is
then given in Figure~\ref{fig:code}

\subsection{Formal specification of the algorithm}

The informal specification given above talks about nodes of the graph
which are reachable from the root node: all of them have to be
marked. So it appears immediately that to formally specify the
algorithm, one needs to talk about reachability of some node from
another in the graph. In the \caduceus{} methodology, such an
``external'' notion can be declared by a predicate annotation. For the
reachability predicate, this is written has
\begin{alltt}
\begin{slshape}
/*@ predicate isreachable (node p1, node p2) reads p1->r,p1->l */
\end{slshape}
\end{alltt}
It is important to notice that this annotation is simply a declaration
of a logical predicate, but nothing yet is sais about its semantics:
we only declare a binary predicate that is intended to relate two
nodes. The \texttt{reads} clause that follows the predicate
declaration is for giving information on which data the predicate
depends on: in this case, we say that the value of the children of
\texttt{p1}. There is a small trick here: TODO

Once the \texttt{isreachable} predicate is declared, it is possible to
give a specification to the algorithm, even if we did not give the
semantics of \texttt{isreachable} yet. The specification is as follows:
\begin{alltt}\begin{slshape}
/*@ requires 
  @   \bs{}forall node x; 
  @      x != \null && isreachable(root,x) => \valid(x) && ! x->m  
  @ ensures 
  @   (\bs{}forall node x; \old(x->l) == x->l && \old(x->r) == x->r) 
  @   &&
  @   (\bs{}forall node x; \valid(x) && isreachable(root,x) => x->m) 
  @   &&
  @   (\bs{}forall node x; ! isreachable(root,x) => x->m == \old(x->m))
  @*/\end{slshape}
void schorr_waite(node root);
\end{alltt}

The clause \texttt{requires} specifies
the pre-condition. The built-in predicate $\valid(p)$ in \caduceus{}
means that the pointer $p$ points to a correctly allocated memory
block, so that dereferencing $p$ is safe. So the pre-condition is that
for any node $x$ that is not null and reachable 
from the root, it should be valid, and its $m$ field must be false,
that is $x$ is unmarked. 

The clause \texttt{ensures} specifies
the post-condition: it is a conjunct of three assertions. The built-in
construct $\old(e)$ in \caduceus{} denotes the value of expression $e$
before the execution of the function. So the first assertion means
that for all nodes (even the non-reachable ones), the children are the
same after the run of the algorithm, as they were before. This is of
course very important to specify, because during the execution the
children are modified, so one wants to prove that the initial
structure of the graph is restored. The second assertion specifies
that any node $x$ that is regularly allocated {\Huge Pourquoi pas non
  null seulement ?} and reachable 
from the root, is now marked. So this specifies that all the graph has
been traversed. The third assertion says that for the nodes that were
not reachable, their mark is not changed by the algorithm, that is the
unreachable node are not traversed by the algorithm.

With respect to the most recent formal specification given in
Isabelle/HOL \cite{mehta03cade}, we added two things: first the
pre-condition that of all pointers of the graph are regularly
allocated at the beginning, so that we are now able to prove that no
wrong pointer dereferencing can occur. Second, we also talk about the
non-reachable nodes of the graph: we are also able to show that they
are not affected.


%%% Local Variables: 
%%% mode: latex
%%% TeX-master: "rapport"
%%% End: 
