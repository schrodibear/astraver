\documentclass[a4paper,12pt]{article}

\usepackage[a4paper=true,pdftex,colorlinks=true,urlcolor=blue,pdfstartview=FitH]{hyperref}

\usepackage[T1]{fontenc}

\usepackage{times}
\usepackage{graphicx}
\textwidth=160mm
\oddsidemargin=0mm
\evensidemargin=0mm
\textheight=270mm
\topmargin=-25mm

\renewcommand{\textfraction}{0.01}
\renewcommand{\topfraction}{0.99}
\renewcommand{\bottomfraction}{0.99}

\pagestyle{empty}

\begin{document}
\sloppy

\title{Case study: new features of specification languages}
\author{Claude March\'e}
\date{February 2008}
\maketitle

\thispagestyle{empty}



The following program is not a challenge to prove (although not easy
for automatic provers). It is rather an illustration of new features
we added to our specification languages ACSL (for C programs) and KSL
(Krakatoa Specification Language for Java programs).

The new features are:
\begin{itemize}
\item named behaviors in contracts
\item ghost code using pure arithmetic types
\item predefined labels
\item user lemmas (which differ from user axioms) 
\end{itemize}

We illustrate them on a Java version.

\section{The case study}

The following Java program computes the gcd (greatest common divisor)
of two non-negative integers, using Euclide's algorithm.
\input{Gcd-nospec.pp}

The goal is to prove not only that the result is the gcd, by also the
classical \emph{B\'ezout identity}: there exists $a$ and $b$ such that
$ax+by=gcd(x,y)$. These properties can be expressed easily within
different \emph{named behaviors}, leading to the contract given on
Figure~\ref{fig:spec}.

\begin{figure}[t]
  \input{Gcd-spec.pp}
  \caption{Specification contract for \texttt{gcd} method}
\label{fig:spec}
\hrulefill
\end{figure}

For this contract, we first introduce the binary predicate
$\texttt{divides}(x,y)$ true whenever $x$ divides $y$, and the ternary
predicate $\texttt{isGcd}(a,b,d)$ true whenever $d$ is a gcd of $a$
and $b$. The latter is introduced by a predicate instead of a
function, for simplicity (we could turn it into a total function if we
require that the result is non-negative, nevertheless it would remain
non-trivial to prove existence of unicity of it).

The difficulty is how to prove such a contract without actually
extend the source code into the so-called \emph{extended Euclide's
algorithm} to indeed compute $a$ and $b$. 

\section{Solution using Krakatoa}

A solution is to instrument the code using so-called \emph{ghost
  code}, to compute $a$ and $b$ in parallel with the normal execution
of the code. Interestingly, in the ghost code we do not want to
compute with machine integers: we do intend to actually run the
code. Thus, the ghost code indeed operate on mathematical integers,
whose type is denoted as \texttt{integer} in the specification
language. The resulting fully annotated code is given Figure~\ref{fig:code}.

\begin{figure}[t]
  \input{Gcd.pp}
  \caption{Annotated code}
\label{fig:code}
\hrulefill
\end{figure}

Another novelty to remark is in the loop invariant, where we use the
notation $\verb|\at|(e,\texttt{Pre})$ to denote the value of $e$ in
the pre-state of the method. \texttt{Pre} is indeed a \emph{predefined
  label} visible in the code, to refer to the pre-state. It is
intended to be a safer replacement of the use \verb|\old| which is now
forbidden: it might be confusing for the users, one may think it
denotes the state just before entering the loop, or at the beginning
of the current iteration, etc.

\section{Proof and comments}

The proof can be performed almost fully automatically. Nevertheless,
it requires the cooperation of several theorem provers, and needs to
add a few \emph{lemmas}: these are added as hihnts for provers, and
unlike tohe former \emph{axioms} that were allowed by Caduceus, these
are supposed to be proven too.

\begin{figure}[p]
  \input{Gcd-lemmas.pp}
  \caption{Lemmas added as hints}
\label{fig:lemmas}
\end{figure}

For this example, the lemmas of Figure~\ref{fig:lemmas} are
added. These are commmonly known, and indeed there are proven by at
least one automatic prover (Z3 in particular), excepted for
\verb|div_mod_property|, \verb|gcd_property|, \verb|gcd_zero|.

All the verification conditions generared from method \verb|Gcd| are
proven automatically, including the absence of arithmetic
overflow. Indeed they are all proven by Simplify, except the one to
show that \verb|x%y| is greater than \verb|min_int|. 

The complete ``proof card'' is given below. The first screenshot
displays VCs for the user lemmas and for the safety if the \verb|gcd|
method. The focus is on the goal for proving that the operation
\verb|x%y| do not overflow, only Z3 is able to discharge it. The
second screenshot displays VCs for the two given behaviors. The focus
is on the preservation of the invariant
\verb|c*\at(x,Pre)+d*\at(y,Pre) == y|, only Simplify is able to
discharge it.

\clearpage

\begin{center}
  \vspace*{-0.07\textheight}
  \hspace*{-0.04\textwidth}\includegraphics[width=1.08\textwidth]{Gcd1.png}
  \vfill
  \hspace*{-0.04\textwidth}\includegraphics[width=1.08\textwidth]{Gcd2.png}
  \vspace*{-0.07\textheight}
\end{center}


\end{document}

%%% Local Variables: 
%%% mode: latex
%%% TeX-master: t
%%% End: 
