
\section{Syntax}

\subsection{Abstract syntax}

\begin{displaymath}
  \begin{array}{rrl}
    \epsilon
      & ::= & (\{x,\ldots,x\}, \{x,\ldots,x\}) \\
    \tau
      & ::= & \beta \\
      &   | & \tref{\tau} \\
      &   | & \tarray{t}{\tau} \\
      &   | & (x:\tau)\rightarrow\kappa \\
    \kappa
      & ::= & (x:\tau, \epsilon, p, p) \\[1em]

    e
      & ::= & \prepost{p}{s}{p} \\[1em]
    s
      & ::= & E \\
      &   | & \access{x} \\
      &   | & \assign{x}{e} \\
      &   | & \pref{e} \\
      &   | & \taccess{x}{e} \\ % sugar?
      &   | & \tassign{x}{e}{e} \\ % sugar?
      &   | & \seq{e}{e} \\
      &   | & \plabel{x}{e} \\
      &   | & \assert{p}{e} \\
      &   | & \while{e}{p}{v}{e} \\
      &   | & \ite{e}{e}{e} \\
      &   | & \letin{x}{e}{e} \\
      &   | & \fun{x}{\tau}{e} \\
      &   | & \app{e}{e} \\
      &   | & \rec{x}{\tau}{v}{e} \\
      &   | & \coerce{e}{\kappa} \\[1em]
    E & ::= & x \\
      &   | & \mathit{constant} \\
      &   | & f(E,\dots,E) \\
  \end{array}
\end{displaymath}

\subsection{Syntactic sugar}

\section{Typing}

\begin{displaymath}
\begin{array}{c}

  % variable
  \irule{\mathit{Type}(c)=\tau}
        {\typage \Gamma c {(\tau,\emptyef)} }
  \iname{CONST}

  \qquad

  \irule{x:\tau \in \Gamma}
        {\typage \Gamma x {(\tau,\emptyef)} }
  \iname{VAR}

  \\[1.5em]

  \irule{\typage {\Gamma,x:\tau} e {\kappa}
         \qquad
         \Gamma \vdash_a \wf{\tau}}
        {\typage \Gamma {\fun{x}{\tau}{e}} 
                         {((x:\tau)\rightarrow\kappa,\emptyef)}}
  \iname{FUN}

  \\[1.5em]

  % application � un terme
  \irule{\typage \Gamma {e_1} {(\tau_2\rightarrow(\tau_1,\epsilon),
                               \epsilon_1)}
         \qquad
         \typage \Gamma {e_2} {(\tau_2,\epsilon_2)}
         \qquad
         \pur{\tau_2}}
        {\typage \Gamma {\app{e_1}{e_2}} 
                         {(\tau_1,\epsilon_1\sqcup\epsilon_2\sqcup\epsilon)}}
  \iname{APP} 

  \\[1.5em]

  % application � une r�f�rence
  \irule{\typage \Gamma e 
               {((x:\tref{\tau_1})\rightarrow(\tau,\epsilon),
                 \epsilon_2)}
         \qquad
         r:\tref{\tau_1}\in\Gamma
         \qquad
         r\notin(\tau,\epsilon)}
        {\typage \Gamma {\app{e}{r}} 
                         {(\tau[x\leftarrow r],
                           \epsilon_2\sqcup\epsilon[x\leftarrow r])}}
  \iname{APPREF} 

  \\[1.5em]

  % fonction r�cursive

  \irule{\begin{array}{c}
         \Gamma\vdash_a\wf{\vec{\tau}}
         \qquad
         \Gamma,\vec{x}:\vec{\tau} \vdash_a \wf{\kappa}
         \\[0.3em]
         \typage {\Gamma,f:(\vec{x}:\vec{\tau})\rightarrow\kappa,
                          \vec{x}:\vec{\tau}}
                  e {\kappa}
         \qquad
         \typage {Pre(\Gamma,\vec{x}:\vec{\tau})} \nu {\variant{A}}
        \end{array}}
        {\typage
            \Gamma
            {\rec{f}{(\vec{x}:\vec{\tau}):\kappa}{\nu}{e}}
            {((\vec{x}:\vec{\tau})\rightarrow\kappa,\emptyef)}}
  \iname{REC}

  \\[1.5em]

  % conditionnelle
  \irule{\typage \Gamma {e_1} {(\bool,\epsilon_1)}
         \qquad  
         \typage \Gamma {e_2} {(\tau,\epsilon_2)}
         \qquad
         \typage \Gamma {e_3} {(\tau,\epsilon_3)}}
        {\typage \Gamma {\ite{e_1}{e_2}{e_3}}
                         {(\tau,\epsilon_1\sqcup\epsilon_2\sqcup\epsilon_3)}}
  \iname{COND}
  
  \\[1.5em]
  
  % let in

  \irule{\typage \Gamma {e_1} {(\tau_1,\epsilon_1)}
         \qquad
         \pur{\tau_1}
         \qquad
         \typage {\Gamma,x:\tau_1} {e_2} (\tau,\epsilon)}
        {\typage \Gamma {\letin{x}{e_1}{e_2}} 
                         {(\tau,\epsilon_1\sqcup\epsilon)}}
  \iname{LET}

  \\[1.5em]

  % let ref in

  \irule{\typage \Gamma {e_1} {(\tref{\tau_1},\epsilon_1)}
         \qquad
         \typage {\Gamma,x:\tref{\tau_1}} {e_2} {(\tau_2,\epsilon_2)}
         \qquad
         x\notin\tau_2}
        {\typage \Gamma {\letin{x}{e_1}{e_2}} 
                         {(\tau_2,\epsilon_1\sqcup\epsilon_2\backslash x)}}
  \iname{LETREF}

  \\[1.5em]

  % sequence 
  \irule{\typage \Gamma {e_1} {(\unit,\epsilon_1)}
         \qquad 
         \typage \Gamma {e_2} {(\tau,\epsilon_2)}}
        {\typage \Gamma {\seq{e_1}{e_2}} {(\tau,\epsilon_1\sqcup\epsilon_2)}}
  \iname{SEQ}

  \\[1.5em]

  % boucle
  \irule{\begin{array}{c}
         \typage \Gamma {e_1} {(\bool,\epsilon_1)}
         \qquad
         \typage \Gamma {e_2} {(\unit,\epsilon_2)}
         \\[0.3em]
         \typage{Pre(\Gamma)}{P}{\kw{Prop}}
         \qquad
         \typage{Pre(\Gamma)}{\nu}{\variant{A}}
         \end{array}}
        {\typage \Gamma {\while{e_1}{P}{\nu}{e_2}}
                         {(\unit,\epsilon_1\sqcup\epsilon_2)}}
   \iname{LOOP}

  \\[1.5em]

  % cr�ation
  \irule{\typage \Gamma e {(\tau,\epsilon)}
         \qquad
         \pur{\tau}}
        {\typage \Gamma {\pref{e}} {(\tref{\tau},\epsilon)}}
  \iname{REF}

  \qquad

  % acc�s
  \irule{x:\tref{\tau} \in \Gamma}
        {\typage \Gamma {!x} {(\tau,(\{x\},\emptyset))}}
  \iname{DEREF}

  \\[1.5em]

  % affectation
  \irule{x:\tref{\tau} \in \Gamma
         \qquad
         \typage \Gamma e {(\tau,(\rho,\omega))}}
        {\typage \Gamma {\assign{x}{e}} 
                         {(\unit,(\{x\}\cup\rho,\{x\}\cup\omega))}}
  \iname{AFF}

\end{array}
\end{displaymath}

\section{Weakest preconditions}

\begin{displaymath}
  \begin{array}{rl}
    \wpre{E}{q} & = 
       q[\result \leftarrow E] \\
    \wpre{\access{x}}{q} & = 
       q[\result \leftarrow x] \\
    \wpre{\assign{x}{e}}{q} & = 
       \wpre{e}{q[\result\leftarrow\void; x\leftarrow\result]} \\
    \wpre{\pref{e}}{q} & =
       \mathit{meaningless} \\ % cf typing
    \wpre{\taccess{t}{e}}{q} & =
       \wpre{e}{q[\result\leftarrow t[\result]]} \\
    \wpre{\tassign{t}{e_1}{e_2}}{q} & =
       ??? \\
    \wpre{\seq{e_1}{e_2}}{q} & =
       \wpre{e_1}{\wpre{e_2}{q}} \\
    \wpre{\plabel{l}{e}}{q} & =
       \wpre{e}{q}[x_l\leftarrow x] \\
    \wpre{\assert{p}{e}}{q} & =
       \wpre{e}{q} \\ % ???
    \wpre{\ite{e_1}{e_2}{e_3}}{q} & =
       \wpre{e_1}{\ite{\result}{\wpre{e_2}{q}}{\wpre{e_3}{q}}} \\
    \wpre{\letin{x}{e_1}{e_2}}{q} & =
       \wpre{e_1}{\wpre{e_2}{q}[x\leftarrow\result]} \\
    \wpre{\fun{x}{\tau}{e}}{q} & =
       \\
    \wpre{\app{e_1}{e_2}}{q} & =
       \\
    \wpre{\rec{x}{\tau}{\phi}{e}}{q} & =
       \\
    \wpre{\coerce{e}{\kappa}}{q} & =
       \wpre{e}{q} \\ % ???
  \end{array}
\end{displaymath}

\begin{displaymath}
  \wpre{\while{e_1}{I}{\phi}{e_2}}{q} = I 
\end{displaymath}


\section{Functional interpretation}

%%% Local Variables: 
%%% mode: latex
%%% TeX-master: "theory"
%%% End: 
