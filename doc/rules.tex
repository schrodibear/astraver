
\section{Syntax}

\subsection{Abstract syntax}

\begin{displaymath}
  \begin{array}{rrl}
    \epsilon
      & ::= & (\{x,\ldots,x\}, \{x,\ldots,x\}) \\
    \tau
      & ::= & \beta \\
      &   | & \tref{\tau} \\
      &   | & \tarray{t}{\tau} \\
      &   | & (x:\tau)\rightarrow\kappa \\
    \kappa
      & ::= & (x:\tau, \epsilon, p, p) \\[1em]

    e
      & ::= & \prepost{p}{s}{p} \\[1em]
    s
      & ::= & x \\
      &   | & \access{x} \\
      &   | & \assign{x}{e} \\
      &   | & \taccess{x}{e} \\ % sugar?
      &   | & \tassign{x}{e}{e} \\ % sugar?
      &   | & \seq{e}{e} \\
      &   | & \plabel{x}{e} \\
      &   | & \assert{p}{e} \\
      &   | & \while{e}{p}{t}{p}{e} \\
      &   | & \ite{e}{e}{e} \\
      &   | & \fun{x}{\tau}{e} \\
      &   | & \app{e}{e} \\
      &   | & \rec{x}{\tau}{e} \\
      &   | & \letin{x}{e}{e}
  \end{array}
\end{displaymath}

\subsection{Syntactic sugar}

\section{Typing}

\section{Weakest preconditions}

\section{Functional interpretation}

%%% Local Variables: 
%%% mode: latex
%%% TeX-master: "theory"
%%% End: 
