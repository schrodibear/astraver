%%%%%%%%%%%%%%%%%%%%%%%%%%%%%%%%%%%%%%%%%%%%%%%%%%%%%%%%%%%%%%%%%%%%%%%%%%
%                                                                        %
%  The Why platform for program certification                            %
%                                                                        %
%  Copyright (C) 2002-2014                                               %
%                                                                        %
%    Jean-Christophe FILLIATRE, CNRS & Univ. Paris-sud                   %
%    Claude MARCHE, INRIA & Univ. Paris-sud                              %
%    Yannick MOY, Univ. Paris-sud                                        %
%    Romain BARDOU, Univ. Paris-sud                                      %
%                                                                        %
%  Secondary contributors:                                               %
%                                                                        %
%    Thierry HUBERT, Univ. Paris-sud  (former Caduceus front-end)        %
%    Nicolas ROUSSET, Univ. Paris-sud (on Jessie & Krakatoa)             %
%    Ali AYAD, CNRS & CEA Saclay      (floating-point support)           %
%    Sylvie BOLDO, INRIA              (floating-point support)           %
%    Jean-Francois COUCHOT, INRIA     (sort encodings, hyps pruning)     %
%    Mehdi DOGGUY, Univ. Paris-sud    (Why GUI)                          %
%                                                                        %
%  This software is free software; you can redistribute it and/or        %
%  modify it under the terms of the GNU Lesser General Public            %
%  License version 2.1, with the special exception on linking            %
%  described in file LICENSE.                                            %
%                                                                        %
%  This software is distributed in the hope that it will be useful,      %
%  but WITHOUT ANY WARRANTY; without even the implied warranty of        %
%  MERCHANTABILITY or FITNESS FOR A PARTICULAR PURPOSE.                  %
%                                                                        %
%%%%%%%%%%%%%%%%%%%%%%%%%%%%%%%%%%%%%%%%%%%%%%%%%%%%%%%%%%%%%%%%%%%%%%%%%%


\newtheorem{example}{Example}[chapter]

% common Why title page

\newcommand{\whytitlepage}[4]{%
\begin{titlepage}
\begin{center}
~\vfill
\rule\textwidth{0.1cm}\\[0.5cm]
\begin{Huge}\sffamily
#1 % title
\end{Huge}
\\[1cm]
\begin{Large}\sffamily
#2
\end{Large}
\\[0.1cm]
\rule\textwidth{0.1cm}\\[1cm]
Version #3\\[3cm]
#4
\vfill
\today\\
INRIA Team-Project \emph{Proval} \url{http://proval.lri.fr} \\
INRIA Saclay - \^Ile-de-France \& LRI, CNRS UMR 8623\\ 
4, rue Jacques Monod, 91893 Orsay cedex, France
\end{center}
\end{titlepage}}

\newcommand{\why}{\textsf{Why}}
\newcommand{\Why}{\why}
\newcommand{\java}{\textsc{Java}\index{Java@\textsf{Java}}}
\newcommand{\Java}{\java}
\newcommand{\krakatoa}{\textsf{Krakatoa}\index{Krakatoa@\textsf{Krakatoa}}}
\newcommand{\Krakatoa}{\krakatoa}
\newcommand{\caduceus}{\textsf{Caduceus}\index{Caduceus@\textsf{Caduceus}}}
\newcommand{\Caduceus}{\caduceus}
\newcommand{\coq}{\textsf{Coq}\index{Coq@\textsf{Coq}}}
\newcommand{\Coq}{\coq}
\newcommand{\pvs}{\textsf{PVS}\index{PVS@\textsf{PVS}}}

%

\newcommand{\kw}[1]{\ensuremath{\mathsf{#1}}}

% types
\newcommand{\bool}{\kw{bool}}
\newcommand{\unit}{\kw{unit}}
%\newcommand{\tref}[1]{\ensuremath{#1~\kw{ref}}}
\newcommand{\tref}[1]{\ensuremath{#1~\mathsf{ref}}}
\newcommand{\tarray}[2]{\ensuremath{\kw{array}~#1~\kw{of}~#2}}

% constructs
\newcommand{\prepost}[3]{\ensuremath{\{#1\}\,#2\,\{#3\}}}
\newcommand{\result}{\ensuremath{\mathit{result}}}

\newcommand{\void}{\kw{void}}
\newcommand{\access}[1]{\ensuremath{!#1}}
\newcommand{\assign}[2]{\ensuremath{#1~:=~#2}}
\newcommand{\pref}[1]{\ensuremath{\kw{ref}~#1}}
\newcommand{\taccess}[2]{\ensuremath{#1\texttt{[}#2\texttt{]}}}
\newcommand{\tassign}[3]{\ensuremath{#1\texttt{[}#2\texttt{]}~\texttt{:=}~#3}}
\newcommand{\faccess}[2]{\ensuremath{(\mathit{access}~#1~#2)}}
\newcommand{\fupdate}[3]{\ensuremath{(\mathit{update}~#1~#2~#3)}}
%\newcommand{\taccess}[2]{\ensuremath{#1[#2]}}
%\newcommand{\tassign}[3]{\ensuremath{#1[#2]~:=~#3}}
%\newcommand{\faccess}[2]{\ensuremath{(\mathit{access}~#1~#2)}}
%\newcommand{\fupdate}[3]{\ensuremath{(\mathit{update}~#1~#2~#3)}}
% \newcommand{\block}[1]{\ensuremath{\kw{begin}~#1~\kw{end}}}
\newcommand{\seq}[2]{\ensuremath{#1;~#2}}
%\newcommand{\plabel}[2]{\ensuremath{#1:#2}}
\newcommand{\plabel}[2]{\ensuremath{#1\texttt{:}#2}}
\newcommand{\assert}[2]{\ensuremath{\kw{assert}~\{#1\};~#2}}
\newcommand{\while}[4]{\ensuremath{\kw{while}~#1~\kw{do}~\{\kw{invariant}~#2~\kw{variant}~#3\}~#4~\kw{done}}}
\newcommand{\ite}[3]{\ensuremath{\kw{if}~#1~\kw{then}~#2~\kw{else}~#3}}
\newcommand{\fun}[3]{\ensuremath{\kw{fun}~#1:#2\rightarrow#3}}
\newcommand{\app}[2]{\ensuremath{(#1~#2)}}
\newcommand{\rec}[4]{\ensuremath{\kw{rec}~#1:#2~\{\kw{variant}~#3\}=#4}}
\newcommand{\letin}[3]{\ensuremath{\kw{let}~#1=#2~\kw{in}~#3}}
\newcommand{\raisex}[2]{\ensuremath{\kw{raise}~(#1~#2)}}
\newcommand{\exn}[1]{\ensuremath{\kw{Exn}~#1}}
\newcommand{\try}[2]{\ensuremath{\kw{try}~#1~\kw{with}~#2~\kw{end}}}
\newcommand{\coerce}[2]{\ensuremath{(#1:#2)}}

\newcommand{\statement}{\textit{statement}}
\newcommand{\program}{\textit{program}}
\newcommand{\expression}{\textit{expression}}
\newcommand{\predicate}{\textit{predicate}}

% inference rules
\newcommand{\espacev}{\rule{0in}{1em}}
\newcommand{\espacevn}{\rule[-0.4em]{0in}{1em}}
\newcommand{\irule}[2]
  {\frac{\espacevn\displaystyle#1}{\espacev\displaystyle#2}}
\newcommand{\typage}[3]{#1 \, \vdash \, #2 : #3}
\newcommand{\iname}[1]{\textsf{#1}}

\newcommand{\emptyef}{\bot}
\newcommand{\wf}[1]{#1~\kw{wf}}
\newcommand{\pur}[1]{#1~\kw{pure}}
\newcommand{\variant}[1]{#1~\kw{variant}}

\newcommand{\wpre}[2]{\ensuremath{\mathit{wp}(#1,#2)}}
\newcommand{\wprx}[3]{\ensuremath{\mathit{wp}(#1,#2,#3)}}

\newcommand{\barre}[1]{\ensuremath{\overline{#1}}}


% BNF grammar

\newif\ifspace
\newif\ifnewentry
\newcommand{\addspace}{\ifspace \; \spacefalse \fi}
\newcommand{\term}[1]{\addspace\hbox{\texttt{#1}} \spacetrue}
\newcommand{\nonterm}[1]{%
\addspace\hbox{\textsl{#1}\ifnewentry\index{#1@\textsl{#1}!non-terminal}\fi}\spacetrue}
\newcommand{\repetstar}{^*\spacetrue}
\newcommand{\repetplus}{^+\spacetrue}
\newcommand{\repetone}{^?\spacetrue}
\newcommand{\lparen}{\addspace(}
\newcommand{\rparen}{)}
\newcommand{\orelse}{\addspace\mid\spacetrue}
\newcommand{\sep}{ \\[2mm] \spacefalse\newentrytrue}
\newcommand{\newl}{ \\ & & \spacefalse}
\newcommand{\alt}{ \\ & \mid & \spacefalse}
\newcommand{\is}{ & ::= & \newentryfalse}
\newenvironment{syntax}{$$\begin{array}{rrll}\spacefalse}{\end{array}$$}
\newcommand{\synt}[1]{$\spacefalse#1$}
\newcommand{\emptystring}{\epsilon}
\newcommand{\below}{See\; below}

% backslash keywords

\newcommand{\ttkw}[1]{\texttt{#1}}
\newcommand{\bskw}[1]{\ttkw{\char'134 #1}}
\newcommand{\bkw}[1]{\ttkw{\textbf{#1}}}


%%% Local Variables: 
%%% mode: latex
%%% TeX-master: "doc"
%%% End: 
