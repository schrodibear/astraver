%%%%%%%%%%%%%%%%%%%%%%%%%%%%%%%%%%%%%%%%%%%%%%%%%%%%%%%%%%%%%%%%%%%%%%%%%%%%%%%%
%                                                                              %
%  The Why platform for program certification                                  %
%                                                                              %
%  Copyright (C) 2002-2010                                                     %
%                                                                              %
%    Yannick MOY, Univ. Paris-sud 11                                           %
%    Jean-Christophe FILLIATRE, CNRS                                           %
%    Claude MARCHE, INRIA & Univ. Paris-sud 11                                 %
%    Romain BARDOU, Univ. Paris-sud 11                                         %
%    Thierry HUBERT, Univ. Paris-sud 11                                        %
%                                                                              %
%  Secondary contributors:                                                     %
%                                                                              %
%    Nicolas ROUSSET, Univ. Paris-sud 11 (on Jessie & Krakatoa)                %
%    Ali AYAD, CNRS & CEA Saclay         (floating-point support)              %
%    Sylvie BOLDO, INRIA                 (floating-point support)              %
%    Jean-Francois COUCHOT, INRIA        (sort encodings, hypothesis pruning)  %
%    Mehdi DOGGUY, Univ. Paris-sud 11    (Why GUI)                             %
%                                                                              %
%  This software is free software; you can redistribute it and/or              %
%  modify it under the terms of the GNU Lesser General Public                  %
%  License version 2.1, with the special exception on linking                  %
%  described in file LICENSE.                                                  %
%                                                                              %
%  This software is distributed in the hope that it will be useful,            %
%  but WITHOUT ANY WARRANTY; without even the implied warranty of              %
%  MERCHANTABILITY or FITNESS FOR A PARTICULAR PURPOSE.                        %
%                                                                              %
%%%%%%%%%%%%%%%%%%%%%%%%%%%%%%%%%%%%%%%%%%%%%%%%%%%%%%%%%%%%%%%%%%%%%%%%%%%%%%%%


\section{Automatic invariants}

\caduceus{} automatically adds invariants into the program, to express
information that can be statically inferred from the program. 

For each structure declared, axioms and predicates are generated. For
each global variables declared, strong invariants are generated.

\subsection{validity of pointers}

for a struct s with fields (ty1 f1 ; ... ; tyn fn), a predicate
valid_s is generated :

valid_s(struct s x) := 
  conjonction sur i de :
   selon le cas sur tyi :
     struct t :  valid_t(x.fi)
     array(t,size) : valid_range(x.fi,0,size-1) &&
                     forall i, 0 <= i < size => valid_t(x.fi[i])
               (assuming valid_t is true for t a base type)
     default : true

and one axiom is generated :

valid_s_pointer(struct s *x) :=
   valid(x) => valid_s(*x)

% not needed
%valid_s_range(struct s *x,a,b) :=
%   valid(x) => valid_s(*x)

for each global variable x of type ty, a strong invariant is generated
:

selon le cas sur ty :

 type de base : rien
 struct s : valid(x)

 array(t,size) :

 array(array(...,array(t),size1,..,sizen) : 
    valid_range(x,0,sizen-1) &&
      (forall in, 0 <= in < sizen,
           valid_range(x[in],0,size{n-1}-1) &&
            (forall i_{n-1} , ....
                  ...
                  valid_range(x[in]...[i2],0,size1-1)))

  ie, recursively, valid_array(x,t,size) where

  valid_array(x,t,size) =
    switch t :
      array(t',size') : 
        valid_range(x,0,size-1) &&
        forall i, 0 <= i < size => valid_array(x[i],t',size')
      default : valid_range(x,0,size-1)


\subsection{separation}

for a struct s with fields (ty1 f1 ; ... ; tyn fn), a predicate
internal_separation_s is generated :

internal_separation_s(struct s x) := 
  each pair (p1,p2) of different pointers statically
  inside x are non-aliased (if they same type, if not this information
  is not added since it is supposed to be useless).

for each pair of struct s1 and s2 (even if s1=s2), a predicate
separation_s1_s2 is generated :

separation_s1_s2(struct s1 x,struct s2 y) := 
  each pointer p1 statically inside x and each
  pointer p2 statically inside y are non-aliased (if they same type),
  under the assumption x <> y if s1=s2

  

internal_separation_s(struct s x) := 
  each pair (p1,p2) of different pointers statically
  inside x are non-aliased (if they same type, if not this information
  is not added since it is supposed to be useless).







For each pair of structures named $s_1$ and $s_2$, an axiom 
$verb|separation_|s_1\verb|_|s_2$ is generated