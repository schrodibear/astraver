\documentclass[12pt,a4paper,twoside,openright]{report}

\usepackage{fullpage}
\usepackage{url}
\usepackage{makeidx}
\usepackage{alltt}
\input{./version.tex}

\newcommand{\caml}{\textsf{Caml}}
\newcommand{\pvs}{\textsf{PVS}\index{PVS@\textsf{PVS}}}
\newcommand{\coq}{\textsf{Coq}\index{Coq@\textsf{Coq}}}
\newcommand{\harvey}{\textsf{haRVey}\index{haRVey@\textsf{haRVey}}}
\newcommand{\simplify}{\textsf{Simplify}\index{Simplify@\textsf{Simplify}}}
\newcommand{\mizar}{\textsf{Mizar}\index{Mizar@\textsf{Mizar}}}
\newcommand{\hollight}{\textsf{HOL Light}\index{HOL Light@\textsf{HOL Light}}}
\newcommand{\krakatoa}{\textsf{Krakatoa}\index{Krakatoa@\textsf{Krakatoa}}}
\newcommand{\java}{\textsc{Java}\index{Java@\textsf{Java}}}
\newcommand{\jml}{\textsc{JML}\index{JML@\textsf{JML}}}
\newcommand{\why}{\textsf{Why}}
\newcommand{\caduceus}{\textsf{Caduceus}}
\newcommand{\te}[1]{\texttt{#1}}
\newcommand{\nt}[1]{$\langle$\textsl{#1}$\rangle$}
\newcommand{\indexnt}[1]{\index{#1@\textsl{#1}, grammar entry}}
\newcommand{\indextt}[1]{\index{#1@\texttt{#1}}}
\newcommand{\etoile}{$^{\star}$}
\newcommand{\etoilesep}[1]{$^{\star}_#1$}
\newcommand{\plus}{$^+$}
\newcommand{\plussep}[1]{$^+_#1$}
\newcommand{\caveat}{\paragraph{Caveat.}}
\newcommand{\caveats}{\paragraph{Caveats.}}
\newenvironment{code}{\begin{small}\begin{alltt}%
\begin{tabular}{|p{0.97\textwidth}|}\hline%
}{\\\hline\end{tabular}\end{alltt}\end{small}}
\def\result{\char'134 result}
\def\at{\char'134 at}
\def\forall{\char'134 forall}
\def\exists{\char'134 exists}
\def\valid{\char'134 valid}
\def\block{\char'134 block}
\def\base{\char'134 base}
\def\fresh{\char'134 fresh}
\def\old{\char'134 old}
\def\bs{\char'134}

\makeindex

\begin{document}
\sloppy

%%% coverpage
\thispagestyle{empty}
\begin{center}
~\\[3cm]
\rule\textwidth{0.1cm}\\[0.5cm]
{\Huge\sf The CADUCEUS verification tool \\[0.5em] for C programs}\\[1cm]
{\Large\sf Tutorial and Reference Manual}\\[0.1cm]
\rule\textwidth{0.1cm}\\[1cm]
Version \caduceusversion\\[3cm]
Jean-Christophe Filli\^atre and Claude March\'e
\vfill
\today\\
\end{center}


%%%%%%%%%%%%%%%%%%%%%%%%%%%%%%%%%%%%%%%%%%%%%%%%%%%%%%%%%%%%%%%%%%%%%%
\chapter*{Foreword}
\addcontentsline{toc}{chapter}{Foreword}

\caduceus\ is a verification tool for C programs. Programs are
specified using annotations in special comments. Proof obligations are
generated using the \why\ tool~\cite{Why} and then can be validated using
various proof assistants or decision procedures.

\medskip

This manual is organized as follows. Chapter~\ref{tutorial} is a
tutorial, introducing \caduceus{} features step by step, illustrated
with examples. Chapter~\ref{refman} is a reference manual, providing
the complete syntax of annotations, list of features, etc.

\subsection*{License}

The \caduceus\ certification tool is \copyright\ 2003-2004 Laboratoire de
Recherche en Informatique (\url{www.lri.fr}).
It is open source and freely available under the terms of the GNU
GENERAL PUBLIC LICENSE Version 2. See the files \texttt{COPYING} and
\texttt{GPL} in the distribution.

\subsection*{Authors}

\caduceus\ is developed by Jean-Christophe Filli\^atre and Claude March\'e.

\subsection*{Availability}

The \caduceus\ tool is available from
\url{http://why.lri.fr/caduceus/}, 
in source and binary formats, together with this documentation and
examples.


%%%%%%%%%%%%%%%%%%%%%%%%%%%%%%%%%%%%%%%%%%%%%%%%%%%%%%%%%%%%%%%%%%%%%%

\chapter{Tutorial}
\label{tutorial}

% This chapter gives an overview of \caduceus\ using trivial programs.

\section{Basic use of \caduceus}

To introduce how the \caduceus{} tool must be used in practice, let us
start with a trivial program: function \texttt{max} computing the
maximum of two integers: \indextt{ensures}
\begin{code}
/*@ ensures 
  @   \result >= x && \result >= y &&
  @   \forall int z; z >= x && z >= y => z >= \result 
  @*/
int max(int x, int y) \{
  if (x > y) return x; else return y;
\}
\end{code}
The specification of \texttt{max} is given in the source code right
before the function definition in a comment of the shape \texttt{/*@
  \dots */}.  Any \caduceus\ annotation is placed in such a comment
(or in a single-line comment \texttt{//@ \dots}).  Note that the
character \texttt{@} is considered as a blank inside annotations.

The specification is here a post-condition, introduced by the keyword
\texttt{ensures}. This is the predicate to be valid when the function
returns. The keyword \texttt{\result} denotes the value returned by
the function, thus of type \texttt{int} in this example. The
post-condition expresses that \texttt{\result} is the smallest integer
greater than or equal to \texttt{x} and \texttt{y} (there are plenty
other ways to specify the \texttt{max} function).

The proof obligations are generated in two steps. First, the caduceus
tool must be run on the source file \texttt{max.c} with
the command:
\begin{verbatim}
> caduceus max.c
\end{verbatim}
This generates several files, some of them in sub-directories. Among
those files, a file named \texttt{max.makefile} is generated, which
contains several commands to be run using the standard \texttt{make}
tool. 

The second step is then to generate proof obligations for some
prover. If the proofs are to be made with the \coq\ proof assistant, 
the command is:
\begin{verbatim}
>  make -f max.makefile coq
\end{verbatim}
This command will generate the proof obligations, together with some
Coq \emph{modeling} of the C program (file \verb|caduceus_why.v|):
\begin{verbatim}
why [...] why/caduceus_spec.why
why [...] why/max.why
\end{verbatim}
and then will try to compile the resulting Coq files:
\begin{verbatim}
coqdep -I coq coq/*.v > max.depend
coqc -I coq coq/caduceus_why.v
coqc -I coq coq/caduceus_tactics.v
coqc -I coq coq/caduceus_spec_why.v
coqc -I coq coq/max_why.v
\end{verbatim}
The last file contains two proof
obligations (two lemmas whose proofs have to be filled in); see the
\why\ manual for more details~\cite{Why}. On this example, the
compilation succeeds, because the default tactic given
(\texttt{intuition}) solves the obligations. In general, the proofs of
this file must be filled in manually, using \texttt{coqide} for instance.

As an alternative to \coq\, let us use the \simplify\ decision procedure.
The \texttt{make} command becomes:
\begin{verbatim}
> make -f max.makefile simplify
\end{verbatim}
The obligations are also discharged automatically in that case:
\begin{verbatim}
time make -f max.makefile clean do_simplify > simplify.log

real    0m0.163s
user    0m0.110s
sys     0m0.060s
valid obligations:       2
invalid obligations:       0
\end{verbatim}
When an obligation fails with \simplify, then no manual proof may be
attempted, but counter-examples given in file \texttt{simplify.log}
may be consulted, to discover missing pre-conditions for instance.

\subsection*{Other provers}
\subsubsection*{\textsf{CVC Lite}}
The use of \textsf{CVC Lite} is very similar to the use of
\simplify. Simply run
\begin{verbatim}
> make -f max.makefile cvcl
\end{verbatim}
and \textsf{CVC Lite} will be run on the proof obligations and results
displayed. (\textsf{CVC Lite} input files are generated withtin the
\texttt{cvcl/} subdirectory.)

\subsubsection*{\pvs} 
To use the \pvs\ proof assistant, first run
\begin{verbatim}
> make -f max.makefile pvs
\end{verbatim}
It generates a bunch of files within the \texttt{pvs/}
subdirectory. Proof obligations are contained in the
\url{pvs/max_why.pvs} file. 

\section{Pointer access and modification}

Let us consider now a small program involving a pointer, in a file
\verb|abs.c|: 
\begin{code}
//@ ensures *p >= 0
void abs1(int *p) \{
  if (*p < 0) *p = -*p;
\}
\end{code}
Running \caduceus{} and then \simplify\ results in:
\begin{verbatim}
> caduceus abs.c
generating Why code for function abs1
> make -f abs.makefile simplify
[...]
why [...] why/caduceus_spec.why
why [...] why/abs.why
[...]
time make -f abs.makefile clean do_simplify > simplify.log

real    0m0.244s
user    0m0.170s
sys     0m0.060s
valid obligations:       1
invalid obligations:       1
\end{verbatim}
so it that case, there are two proof obligations and only one of them
has been proved. To investigate the unproved one, we may look at file
\verb|simplify.log|:
\begin{verbatim}
> cat simplify.log
[...]
Simplify simplify/abs_why.sx.all
Counterexample:
  context:
    (AND
      (EQ null p)
      (EQ (ISpointer p) |@true|)
      (EQ (ISalloc alloc) |@true|)
    )

1: Invalid.

2: Valid.
\end{verbatim}
The counter example given contains \verb|(EQ null p)|, meaning that
this program may not run as expected if pointer \verb|p| is
null: indeed, a segmentation fault would occur in that case. If you try to
use \coq\ instead of \simplify, you will face the same problem when
trying to prove the first obligation: you will end up with
\begin{verbatim}
1 subgoal
p : pointer
alloc : alloc
______________________________________(1/1)
valid alloc p
\end{verbatim}
which is not provable.

To fix this problem, one should specify a pre-condition to function
\verb|abs|, which can be given using the \verb|\valid| predicate of
\caduceus{} as follows: \index{valid@\texttt{\bs{}valid}}\indextt{requires}
\begin{code}
/*@ requires \valid(p)
  @ ensures *p >= 0
  @*/
void abs2(int *p) \{
  if (*p < 0) *p = -*p;
\}
\end{code}
If you run \caduceus{} again on this version, you will first notice
that the first obligation is now automatically discharged by Why, and
Simplify will succeed to prove the second obligation. If you run Coq
instead, then this obligation will not be proved automatically: you
will have to make the proof by hand. After 
executing the \verb|intuition| tactic, the goal to solve has the
following shape:
\begin{verbatim}
...
result : bool
Post5 : if result then ... else ...
______________________________________(1/1)
if result
then ...
\end{verbatim}
It is indeed quite large, because it has two cases depending on the
value of \verb|result|, which is (see \verb|Post5|) the condition of
the \verb|if| in the C program. With Coq, the simplest thing to do
with such a goal shape is to make two cases, with
\begin{verbatim}
destruct result; intuition.
\end{verbatim}
instead of \verb|intuition| alone. This will then result in one
subgoal (the other case has been solved automatically):
\begin{verbatim}
p : pointer
...
intP : memory Z
...
result : Z
H : result = p # intP
intP0 : memory Z
H0 : intP0 = upd intP p (- result)
______________________________________(1/1)
p # intP0 >= 0
\end{verbatim}
\indextt{upd}\indextt{\#}\indextt{memory} This should be read as
follows: \verb|intP| and \verb|intP0| are \emph{memory states}
containing integers. \verb|p # intP| denotes the value stored in
\verb|intP| at ``address'' \verb|p|, i.e it is the logical
representation of the C expression \verb|*p|. \verb|upd| is the logical
function representing memory update: $(\verb|upd|~s~p~v)$ is the
memory state obtained from $s$ by putting value $v$ at ``address''
$p$. Addresses considered here must not be seen as integers like
in a real computer, but simply an abstract datatype \verb|pointer|: this
is why in the following we will refer to them as \emph{pointers} instead
of addresses.

So, what should be proved is that the value stored at pointer \verb|p|
in the memory state \verb|intP0| is non-negative. Using tactic
\verb|subst intP0|, we have to prove that
\begin{verbatim}
Post3 : caduceus_1 = p # intP
Post5 : caduceus_1 < 0
result : Z
H : result = p # intP
______________________________________(1/1)
p # (upd intP p (- result)) >= 0
\end{verbatim}
but \verb|p # (upd intP p (- result))| is simply \verb|-result| since
we access exactly to the value which has just been updated. This shape
$p~\verb|#|~(\verb|upd|~s~q~v)$ will indeed occur very often when making
proofs, and it should be rewritten into $v$ when $p=q$ and into
$p~\verb|#|~s$ otherwise. Since it occurs often, a tactic has been
made for that, called \verb|caduceus|. So here, the proof is simply
done using this tactic (all further reasoning is pure linear
arithmetic that is done automatically).

\section{Arrays}

We now illustrate how arrays are handled by \caduceus{}.

\subsection{Array indexing}

Let us first consider a trivial program which permutes to elements of
an array of integers: 
\begin{code}
void swap(int t[],int i,int j) \{
  int tmp = t[i];
  t[i] = t[j]; 
  t[j] = tmp;
\}
\end{code}
Running \caduceus{} on this program, without any annotation, leads to
two obligations which are indeed not provable: as before, \verb|t|
must not be null, but also the indexes \verb|i| and \verb|j| must be
valid. With \caduceus{}, these conditions may be required using
predicate
\verb|valid_index|\index{valid\_index@\texttt{\bs{}valid\_index}},
i.e. inserting the pre-condition
\begin{code}
//@ requires \valid_index(t,i) && \valid_index(t,j) 
\end{code}
As this, the obligations are now solved automatically both with
Simplify and Coq. However, one may want to specify the function in
more details: for example\index{old@\texttt{\bs{}old}}
\begin{code}
/*@ requires \valid_index(t,i) && \valid_index(t,j) 
  @ ensures t[i] == \old(t[j]) && t[j] == \old(t[i])
\end{code}
where in the post-condition, $\verb|\old|(e)$ means the value of expression
$e$ before execution of the function.

With Simplify, this new specification is still proved
automatically. With Coq, after \verb|intuition; subst; auto.| we have
2 subgoals corresponding to the two terms of the conjunction:
\indextt{shift}
\begin{verbatim} 
t : pointer
i : Z
j : Z
alloc : alloc
intP : memory Z
H : valid_index alloc t i
H0 : valid_index alloc t j
Pre1 : valid alloc (shift t i)
______________________________________(1/2)
shift t i
# (upd (upd intP (shift t i) (shift t j # intP)) (shift t j)
     (shift t i # intP)) = shift t j # intP


______________________________________(2/2)
shift t j
# (upd (upd intP (shift t i) (shift t j # intP)) (shift t j)
     (shift t i # intP)) = shift t i # intP
\end{verbatim}
We meet here for the first time the function \verb|shift|:
$(\verb|shift|~p~i)$ denotes in the logic the C expression $p+i$
(pointer arithmetic), hence $(\verb|shift|~t~i)~\#~s$ denotes
$t[i]$ in the memory state $s$. Dealing with pointer arithmetic in
general will be explained in Section~\ref{pointer-arith}. For the
moment, we only have to deal again with the shape
$p~\verb|#|~(\verb|upd|~s~q~v)$, by using the \verb|caduceus|
tactic. However, this does not work immediately, because indeed we do
not know whether $i$ and $j$ are different. So the Coq proof may be made by
a case reasoning such as:
\begin{verbatim}
assert (i=j \/ i<>j); [omega | intuition].
subst i; caduceus.
caduceus.
caduceus.
\end{verbatim}


\subsection{Arrays and loops}

Let us now consider a more complex example: searching an element in an
array, returning its index if it exists, and the first non-existent
index otherwise: 
\index{valid\_range@\texttt{\bs{}valid\_range}}
\index{forall@\texttt{\bs{}forall}}
\indextt{invariant}\indextt{variant}
\begin{code}
/*@ requires \valid_range(t,0,n-1)
  @ ensures 
  @   (0 <= \result < n => t[\result] == v) &&
  @   (\result == n => \forall int i; 0 <= i < n => t[i] != v) 
  @*/
int index(int t[], int n, int v) \{
  int i = 0;
  /*@ invariant 0 <= i && \forall int k; 0 <= k < i => t[k] != v
    @ variant n - i */ 
  while (i < n) \{
    if (t[i] == v) break;
    i++;
  \}
  return i;
\}
\end{code}
In the pre-condition, the new predicate $\verb|valid_range|(t,0,n-1)$ is used
to specify that all memory locations $t[i]$ for $0\leq i \leq n-1$ are
valid. Notice also the use of universal quantification
(\verb|\forall|) in the post-condition, together with the symbol
\verb|=>| to denote logical implication. 

A very important novelty is for the \verb|while| loop: it is annotated
with an invariant (a property which holds at 
the loop entrance and is preserved by the loop body) and a variant (a
quantity which decreases at each loop step for a well-founded
ordering, thus ensuring termination). The three kinds of C loops can
be annotated (\texttt{while}, \texttt{for}, and \texttt{do \dots\ while}).

A loop annotation is placed right before the loop; the keyword
\texttt{invariant} (resp. \texttt{variant}) introduces the invariant
(resp. the variant). The invariant is optional and must appear before
the variant when present.

With Simplify, the two proof obligations generated are solved
automatically. With Coq, the second one (which appears to be the
validity of the invariant when entering the loop) is solved by
\verb|intuition|, whereas the first one (the preservation of the
invariant) needs a little more reasoning:
\begin{verbatim}
destruct result0; intuition.
subst ; auto.
assert (k<i1 \/ k=i1) ; [omega | intuition ].
apply (H0 k); auto with *.
subst; auto.
subst; auto.
apply (H0 i0); auto with *.
\end{verbatim} 

Notice that both the validity of the post-condition and the
decreasingness of the variant have been discharged automatically by
Why.

\section{Structures}
\label{sec:purse}
Let us now consider a program using structures. We consider a simple
``electronic purse'' modelized by
\begin{code}
typedef struct \{
  int balance;
\} purse;

//@ predicate purse_inv(purse *p) \{ \valid(p) && p->balance >= 0 \}
\end{code}
The new thing here is this last \verb|predicate|\indextt{predicate}
declaration: it is simply a definition in the logic of a new predicate
\verb|purse_inv|, which applies to a (pointer to a) purse, stating
that it is valid and its balance is non-negative. One may now specify
two functions for crediting and withdrawing money has follows:
\begin{code}
/*@ requires purse_inv(p) && s >= 0
  @ ensures purse_inv(p) && p->balance == \old(p->balance) + s 
  @*/
void credit(purse *p,int s) \{
  p->balance = p->balance + s;
\}

/*@ requires purse_inv(p) && 0 <= s <= p->balance
  @ ensures purse_inv(p) && p->balance == \old(p->balance) - s
  @*/
void withdraw(purse *p,int s) \{
  p->balance = p->balance - s;
\}
\end{code}
Execution of \caduceus{} on this file produces indeed a warning, which
will be explained in the next section. For the moment, it can be
safely ignored.

The four obligations are solved automatically by Simplify. With Coq,
the first thing to do manually is to unfold the definition of
\verb|purse_inv|, and indeed the first obligation for \verb|credit|,
to show that access to \verb|p->balance| is valid, is solved by
\begin{verbatim}
unfold purse_inv; intuition.
\end{verbatim}
After the same tactic, the second obligation for \verb|credit| results
into two subgoals:
\begin{verbatim}
2 subgoals
p : pointer
s : Z
alloc : alloc
balance : memory Z
Pre3 : valid alloc p
H0 : s >= 0
H1 : valid alloc p
H2 : p # balance >= 0
caduceus_1 : Z
Post2 : caduceus_1 = p # balance
balance0 : memory Z
H : balance0 = upd balance p (caduceus_1 + s)
______________________________________(1/2)
p # balance0 >= 0


______________________________________(2/2)
p # balance0 = p # balance + s
\end{verbatim}
We meet again the notation $p~\#~s$ where $s$ is now \verb|balance| or
\verb|balance0|: these are also memory states with integer memory
cells, they represent the values of the \verb|balance| field for each
pointer to a purse. In general, there will be memory states whose
names corresponds to names of each fields of structures, this is a
representation of memory states already present in an early work on
general pointer programs by Burstall~\cite{burstall72}, and was emphasized
by Bornat~\cite{bornat00mpc}.

Indeed, the situation is very similar to arrays, we again meet the
shape $p~\verb|#|~(\verb|upd|~s~q~v)$, and the proofs are done by
\begin{verbatim}
subst; caduceus.
\end{verbatim}
for each subgoals. The situation is almost the same for
\verb|withdraw|.

\section{Calling functions}
\label{assigns}\indextt{assigns}

The next step in this tutorial is to deal with function call. Let us
use again our \verb|purse| structure, and consider a simple test:
\begin{code}
/*@ requires purse_inv(p1) && purse_inv(p2)
  @ ensures \result == 0
  @*/
int test1(purse *p1, purse *p2) \{
    p1->balance = 0;
    credit(p2,100);
    return p1->balance;
\}
\end{code}

This function requires three proof obligations, two of them being
solved by Simplify. Unfortunately, this time the counter-example
proposed by Simplify is not easy to interpret. Let's go to Coq proofs:
the first obligation is to show the validity of use of pointer
\verb|p1| in \verb|p1->balance = 0|, it is solved by 
\begin{verbatim}
unfold purse_inv; intuition.
\end{verbatim}
The second obligation is to show that the pre-condition to call to
\verb|credit(p2,100)| is valid, that is (see specification of
\verb|credit|) \verb|purse_inv(p2) && 100 >= 0|. After
\begin{verbatim}
unfold purse_inv; intuition.
subst.
\end{verbatim}
we have to prove
\begin{verbatim}
p1 : pointer
p2 : pointer
alloc : alloc
balance : memory Z
Pre9 : valid alloc p1
H1 : valid alloc p1
H2 : p1 # balance >= 0
H : valid alloc p2
H3 : p2 # balance >= 0
______________________________________(1/1)
p2 # (upd balance p1 0) >= 0
\end{verbatim}
but the \verb|caduceus| tactic does not solve the goal this time,
because indeed the tactic needs to find whether \verb|p1| is equal to
\verb|p2| or not. Although it would be provable by cases, at this
point one should understand that the case where \verb|p1=p2| is not
supposed to happen, and indeed the returned value of \verb|test1| would not be
0 in that case but 100. In other words, the third obligation, which is
the proof of the post-condition, will not be provable. This is an
instance of the so-called \emph{pointer aliasing} problem, and in fact
one should avoid this aliasing in the pre-condition of \verb|test1| by
requiring also \verb|p1 != p2|.\index{pointer
  aliasing}\index{aliasing} 

With that addition, the second obligation is provable in Coq with
\begin{verbatim}
unfold purse_inv; intuition.
subst; caduceus.
\end{verbatim}

But the third obligation is not provable, both with Simplify and
Coq. The problem now is that when proving post-condition of
\verb|test1|, we do not see anymore the implementation of
\verb|credit| but only its specification, and in this specification,
nothing is said about modification of memory locations other than
\verb|p->balance|. Indeed, this is why \caduceus{} issues a warning
on function \verb|credit|: the right thing to do is to add an 
\emph{\texttt{assigns} clause}\indextt{assigns} into the specification
of \verb|credit|. The program now looks like
\begin{code}
/*@ requires purse_inv(p) && s >= 0
  @ assigns p->balance
  @ ensures purse_inv(p) && p->balance == \old(p->balance) + s 
  @*/
void credit(purse *p,int s) \{
  p->balance = p->balance + s;
\}

/*@ requires purse_inv(p1) && purse_inv(p2) && p1 != p2
  @ ensures \result == 0
  @*/
int test1(purse *p1, purse *p2) \{
    p1->balance = 0;
    credit(p2,100);
    return p1->balance;
\}
\end{code}
The \emph{assigns} clause \verb|assigns p->balance| is very similar to
the ones of ESC/Java~\cite{?} and JML~\cite{JML}: it specifies that no
other memory locations than \verb|p->balance| may be modified by the
function.  Of course, that modification requires to make again the
verification of function \verb|credit|: the second obligation is a bit
more complicated, it is nevertheless solved by Simplify automatically,
and with Coq the additional assertion is solved by intuition so indeed
the same proof remains correct.

Let's now go back to function \verb|test1|: the three obligations are
now solved by Simplify, and for the Coq proof of the third obligation,
after \verb|unfold purse_inv; intuition|, we have to solve
\begin{verbatim}
p1 : pointer
p2 : pointer
alloc : alloc
balance : memory Z
Pre9 : valid alloc p1
H0 : p1 = p2 -> False
H : valid alloc p1
H3 : p1 # balance >= 0
H1 : valid alloc p2
H4 : p2 # balance >= 0
balance0 : memory Z
Post2 : balance0 = upd balance p1 0
H5 : 100 >= 0
H6 : valid alloc p2
H7 : p2 # balance0 >= 0
balance1 : memory Z
Pre7 : valid alloc p1
H8 : assigns alloc balance0 balance1 (pointer_loc p2)
H10 : p2 # balance1 = p2 # balance0 + 100
H2 : valid alloc p2
H11 : p2 # balance1 >= 0
result1 : Z
Post7 : result1 = p1 # balance1
______________________________________(1/1)
result1 = 0
\end{verbatim}
Now the important hypothesis we have is \verb|H8|: it specifies that
between memory states \verb|balance0| and \verb|balance1|, the only
memory location that has been assigned is the one pointed by
\verb|p2|. So, since we know that \verb|p1| and \verb|p2| are different
(\verb|H0|), we should have \verb|p1 # balance1 = p1 # balance0|. We proceed
by tactics
\begin{verbatim}
subst result1.
rewrite H8; intuition.
\end{verbatim}
resulting in subgoal
\begin{verbatim}
...
______________________________________(1/1)
p1 # balance0 = 0
\end{verbatim}
which is proved by \verb|subst; caduceus|.















\section{A complete example}

TODO







\section{Other features not already in the tutorial}

\subsection{Other annotations}

Beside loop annotations, there are three other kinds of inner annotations.

\paragraph{Intermediate assertions.}\indextt{assert}
An assertion can be placed at any statement point. A proof obligation
is generated to show its validity. Example:
\begin{code}

  \dots
  if (a > b) \{ int tmp = a; a = b; b = tmp; \}
  /*@ assert a <= b */
  \dots
\end{code}
Such an assertion can be seen as a cut in the proof. It is a property
available below this program point and, conversely, it is collected by
the upward weakest preconditions computation.

\paragraph{Labels.}\indextt{label}
Usual C labels can be used in annotations to refer to terms or
predicates at a particular program point. Example:
\begin{code}
  
  L:
  /*@ invariant i >= \at(i,L) */
  while (i++ < N) \dots
\end{code}
When there is no available label in the code, it is possible to
insert a ``logical'' label using the following annotation:
\begin{code}
  
  /*@ label L */
  /*@ invariant i >= \at(i,L) */
  while (i++ < N) \dots
\end{code}


\paragraph{Specifying pieces of code.}
It is possible to give a specification to an inner statement (which
may be a block or not). The syntax is the same as a function specification:
\begin{code}

  \dots
  /*@ ensures a <= b */
  if (a > b) \{ int tmp = a; a = b; b = tmp; \}
  \dots
\end{code}
Contrary to the \texttt{assert} annotation, the specified
statement is seen as a black box, as if it would be a
function call.
In particular, the weakest precondition computation
will not enter this statement.

Inner specifications may be used to help automatic decision procedures
or to reduce the size of proof obligations (there would be more
numerous but smaller proof obligations).

\subsection{Logical declarations}
\indextt{logic}\indextt{axiom}

Analogously to the \texttt{predicate} declaration seen in
Section~\ref{sec:purse}, it possible to introduce logical functions,
and axioms on them:
\begin{code}
/*@ predicate is_min(int t[],int n,int min) \{
  @     (\forall int i; 0 <= i < n => min <= t[i]) &&
  @     (\exists int i; 0 <= i < n && min == t[i]) 
  @ \}
  @*/

/*@ requires n > 0 && \valid_range(t,0,n)
  @ ensures is_min(t,n,\result) 
  @*/
int min(int t[],int n) \{
  int i;
  int tmp = t[0];  
  /*@ invariant 1 <= i <= n && is_min(t,i,tmp)
    @ variant n-i
    @*/
  for (i=1 ; i < n; i++) \{
     if (t[i] < tmp) tmp = t[i];
  \}
  return tmp;
\}

/*@ logic int min(int t[],int n) reads t[*] */
/*@ logic int max(int t[],int n) reads t[*] */

/*@ axiom min_is_min:
  @    \forall int t[]; \forall int n; n > 0 => is_min(t,n,min(t,n))
  @*/

/*@ axiom is_min_is_min:
  @    \forall int t[]; \forall int n; n > 0 => \forall int m;
  @         is_min(t,n,m) => m == min(t,n)
  @*/

/*@ requires \valid_range(t,0,n) && n > 0
  @ ensures 
  @     min(t,n) <= \result
  @*/
int average(int t[],int n) \{
  int i;
  int sum = 0;  
  /*@ invariant 0 <= i <= n && i * min(t,i) <= sum
    @ variant n-i
    @*/
  for (i=0 ; i < n; i++) \{
     sum += t[i];
  \}
  return sum / n;
\}
\end{code}

Missing: \texttt{predicate} not defined by using \texttt{reads}.

\subsection{Invariant declarations}
\indextt{invariant}

Global invariants may be stated: they are assertions that must hold at
entrance and at exit of functions. Example:

\begin{code}
struct \{
  int x;
  int y;
\} s;
/*@ invariant s_intervals: 0 <= s.x && s.x <= s.y && s.y <= 100 */

int c = 12;
/*@ invariant const_c : c==12 */
\end{code}

For example, with the following function, invariant
\texttt{s\_intervals} is both assumed as pre-condition and required as
post-condition, and \texttt{const\_c} is assumed (but not required
since \texttt{c} is not modified).
\begin{code}
/*@ requires n>=0 @*/
void f(int n) \{
  int t = s.x+n;
  if (t <= s.y - 20) s.x = t + c ;
\}
\end{code}



%%%%%%%%%%%%%%%%%%%%%%%%%%%%%%%%%%%%%%%%%%%%%%%%%%%%%%%%%%%%%%%%%%%%%%
\chapter{Reference manual}
\label{refman}

\caduceus\ is invoked as a batch compiler:
\begin{center}
  \texttt{caduceus} [\textit{options}] \textit{file}\texttt{.c}
\end{center}

\section{Command line options}
\label{usage}\index{Option, of the command line}

\begin{description}
  \item[Generic options:] ~\par
  \item[\texttt{--help}]: ~\par    
    Give usage and exit. 
  \item[\texttt{--version}]: ~\par    
    Give \caduceus\ version number and exit. 
  \item[\texttt{-v}]: ~\par 
    Verbose mode. 
  \item[\texttt{-q}]: ~\par  
    Quiet mode (default).
  \item[\texttt{--werror}]: ~\par 
    Turn warnings as errors. \caduceus\ will stop on the first warning.
  \item[\texttt{-d}]: ~\par 
    Debugging mode. In this mode, \caduceus\ details everything it does,
    printing a lot of material on error output. Not of use for the
    casual user.

  \item[Overall options:] ~\par
  \item[\texttt{-parse-only}]: ~\par  
    Stop after parsing.
  \item[\texttt{-type-only}]: ~\par  
    Stop after type-checking.

  \item[Pre-processing options:] ~\par
  \item[\texttt{-no-cpp}]: ~\par  
    Turn off the pre-processing.
  \item[\texttt{-cpp} \textit{command}]: ~\par  
    Set the pre-processor. It must be an executable taking the file
    to be pre-processed on its command line and printing its output
    on the standard output. 
    The default value is \texttt{gcc -C -E}.
  \item[\texttt{-E}]: ~\par  
    Stop after pre-processing and dump the pre-processed file on
    standard output (mainly useful for debugging).

  \item[Proof obligations generation:] ~\par
  \item[\texttt{-why-opt} \textit{string}]: ~\par
    Pass options to \why\ (to be used in the generated \texttt{Makefile}).
  \item[\texttt{-s}]: ~\par
    Use a separate file for each C function.
    
    The default behaviour is to generate a single \why\ file 
    for each C file and therefore to get all proof obligations for
    each C file into the same prover file (e.g. all \coq\ obligations
    in \texttt{coq/foo\_why.v} for a C file \texttt{foo.c}).
    With this option \texttt{-s}, a separate \why\ file is generated
    for each C function and therefore a separate prover file
    (e.g. the verification of a function \texttt{f} from
    \texttt{foo.c} will generate proof obligations into the \coq\ file
    \texttt{coq/foo\_\_f\_why.v}).
  \item[\texttt{-verify} \textit{f,g,h,\dots}]: ~\par
    Specify the sets of functions to verify. Implies \texttt{-s}.

  \item[Prover specific options:] ~\par
  \item[\texttt{-coq-tactic} \textit{tactic}]: ~\par
    Pass a default Coq tactic to \why\ (in the generated \texttt{Makefile}).

\end{description}

\section{Input Files Syntax}
\label{syntax}\index{Syntax (of input files)}

C files conform to the usual ANSI C
syntax and annotations are inserted in the source as comments.

\subsection{Lexical conventions}
\label{lexical:c}\index{Lexical conventions!C programs}

\subsubsection{Code}

The lexical conventions conform to the ANSI C standard (see
for instance~\cite{KR88}).

\subsubsection{Annotations}
\index{Annotations}

Within annotations the lexical conventions are the same as C ones,
except that:
\begin{itemize}
\item comments are enclosed by \texttt{(*} and \texttt{*)} (so that
  you can put comments inside annotations without messing traditional
  C compilers)
\item the set of keywords is different: \par
  \begin{center}
  \begin{tabular}{l@{\qquad}l@{\qquad}l@{\qquad}l@{\qquad}l}
  \verb!\forall! & \verb!\exists! & \verb!int! & \verb!float! &
  \verb!decreases! \\
  \verb!\true! & \verb!\false! & \verb!if! & \verb!then! & \verb!else! \\
  \verb!invariant! & \verb!variant! & \verb!for! & \verb!label! & 
  \verb!assert! \\ 
  \verb!requires! & \verb!ensures! & \verb!assigns! & \verb!logic! & 
  \verb!axiom! \\
  \verb!predicate! & \verb!\result! & \verb!\old! & 
  \verb!\block_length! & \verb!\null! \\
  \verb!reads! & \verb!\valid! & \verb!\valid_index! &
  \verb!\valid_range! & \verb!\fresh! \\
  \verb!\base_addr! & \verb!\nothing!
  \end{tabular}
  \end{center}
\end{itemize}

\subsection{Syntax}

Annotations are inserted into programs as special comments of the shape
\texttt{/*@ ... */} or one-line comments of the shape \texttt{//@ ...}
Figure~\ref{fig:cfiles} gives the syntax for the additional C
constructs containing annotations; some are C declarations
(non-terminal \nt{declaration}) and some are C statements (non-terminal
\nt{statement}).
Figure~\ref{fig:logic} gives the syntax for the
annotations. 

\begin{figure}[htbp]
\begin{center}
\hrulefill\\
\begin{tabular}{lrl}
  \nt{c\_file}
    & $::=$ & \nt{declaration}\etoile\ \\
  \\[0.1em]

  \nt{declaration}
    & $::=$ & \nt{spec} \nt{type} \nt{identifier} \te{(} 
              \nt{parameter}\etoilesep{\te{,}} \te{)} 
              \te{;} \\
      & $|$ & \nt{spec} \nt{type} \nt{identifier} \te{(} 
              \nt{parameter}\etoilesep{\te{,}} \te{)} 
              \nt{block} \te{;} \\
      & $|$ & \te{/*@} \te{logic} \nt{logic\_type} \nt{identifier} \te{(} 
              \nt{logic\_parameter}\etoilesep{\te{,}} \te{)} \\
              && $[$ 
              \te{reads} \nt{location}\plussep{\te{,}} $]$ \te{*/} \\
      & $|$ & \te{/*@} \te{predicate} \nt{identifier} \te{(} 
              \nt{logic\_parameter}\etoilesep{\te{,}} \te{)} \\
           && $[$ \te{\{} \nt{predicate} \te{\}} $|$ 
              \te{reads} \nt{location}\plussep{\te{,}} $]$ \te{*/} \\
      & $|$ & \te{/*@} \te{axiom} \nt{identifier} \te{:} 
              \nt{predicate} \te{*/} \\
      & $|$ & \te{/*@} \te{invariant} \nt{identifier} \te{:} 
              \nt{predicate} \te{*/} \\
  \\[0.1em]

  \nt{assigned\_locs}
    & $::=$ & \nt{location}\plussep{\te{,}} $|$ \verb!\nothing! \\
  \\[0.1em]

  \nt{spec}
    & $::=$ & \te{/*@} $[$ \te{requires} \nt{predicate} $]$ 
              $[$ \te{assigns} \nt{assigned\_locs} $]$ \\
           && $[$ \te{ensures} \nt{predicate} $]$ 
              $[$ \te{decreases} \nt{variant} $]$ \te{*/} \\
                  \indextt{requires}\indextt{assigns}
                  \indextt{ensures}\indextt{decreases}
  \\[0.1em]

  \nt{statement}
    & $::=$ & \nt{loop\_annot} \te{while} \te{(} \nt{expr} \te{)}
              \nt{statement} \\
    &   $|$ & \nt{loop\_annot} \te{do} \nt{statement} 
              \te{while} \te{(} \nt{expr} \te{)} \\
    &   $|$ & \nt{loop\_annot} \te{for} \te{(} \nt{statement} \te{;} \nt{statement} \te{;}
              $[$ \nt{expr} $]$ \te{)} \\
           && \nt{statement} \\
    &   $|$ & \te{/*@} \te{assert} \nt{predicate} \te{*/} \\ \indextt{assert}
    &   $|$ & \te{/*@} \te{label} \nt{identifier} \te{*/} \\ \indextt{label}
    &   $|$ & \nt{spec} \nt{statement} \\
  \\[0.1em]

  \nt{loop\_annot}
    & $::=$ & \te{/*@} $[$ \te{invariant} \nt{predicate} $]$
              $[$ \te{loop\_assigns} \nt{assigned\_locs} $]$ \\
           && $[$ \te{variant} \nt{variant} $]$ \te{*/}  \\
              \indextt{invariant}\indextt{variant}
  \nt{variant} 
    & $::=$ & \nt{term} $[$ \te{for} \nt{identifier} $]$ \\

  \\[0.1em]
  
  \nt{logic\_type}
    & $::=$ & \te{int} $|$ \te{float} $|$ \nt{logic\_type} \te{[]} 
            $|$ \nt{identifier} \\
  \nt{logic\_parameter}
    & $::=$ & \nt{logic\_type} \nt{identifier} \\

  \\[0.1em]

  \nt{location}
    & $::=$ & \nt{identifier} \\
    &   $|$ & \nt{term} \te{.} \nt{identifier} \\
    &   $|$ & \nt{term} \te{->} \nt{identifier} \\
    &   $|$ & \te{*} \nt{term} \\
    &   $|$ & \nt{term} \te{[} \nt{term} \te{]} \\
    &   $|$ & \nt{term} \te{[} \te{*} \te{]} \\
    &   $|$ & \nt{term} \te{[} \nt{term} \te{..} \nt{term} \te{]} \\
  \\[0.1em]
\end{tabular}\\
\hrulefill
\caption{Annotations within C constructs}
\label{fig:cfiles}
\end{center}           
\end{figure}


\begin{figure}[htbp]
\begin{center}
\hrulefill\\
\begin{tabular}{lrl}
  \nt{term}\indexnt{term}
    & $::=$ & \nt{constant} \\
      & $|$ & \nt{term} \nt{arith\_op} \nt{term} \\
      & $|$ & \te{-} \nt{term} $|$ \te{+} \nt{term}  \\
      & $|$ & \te{*} \nt{term} \\
      & $|$ & \nt{term} \te{->} \nt{identifier} \\
      & $|$ & \nt{term} \te{.} \nt{identifier} \\
      & $|$ & \nt{identifier} \te{(} \nt{term}\plussep{\te{,}} \te{)} \\
      & $|$ & \nt{term} \te{[} \nt{term} \te{]} \\
%      & $|$ & \nt{term} \te{?} \nt{term} \te{:} \nt{term} \\
      & $|$ & \te{(} \nt{term} \te{)} \\
      & $|$ & \verb!\old! \te{(} \nt{term} \te{)} \\
      & $|$ & \verb!\at! \te{(} \nt{term} \te{,} \nt{identifier} \te{)} \\
      & $|$ & \verb!\block_length! \te{(} \nt{term} \te{)} \\
      & $|$ & \verb!\base_addr! \te{(} \nt{term} \te{)} \\
      & $|$ & \verb!\result! \\
      & $|$ & \verb!\null! \\
  \\[0.1em]

  \nt{constant}\indexnt{constant}
    & $::=$ & \nt{integer-constant} $|$ \nt{floating-point-constant} \\
  \\[0.1em]

  \nt{arith\_op}\indexnt{arith\_op}
    & $::=$ & \te{+} $|$ \te{-} $|$ \te{*} $|$ \te{/} $|$ \te{\%} \\
  \\[0.1em]

  \nt{predicate}\indexnt{predicate}
    & $::=$ & \verb!\true! \\
      & $|$ & \verb!\false! \\
      & $|$ & \nt{identifier} \\
      & $|$ & \nt{identifier} \te{(} \nt{term}\plussep{\te{,}} \te{)} \\
      & $|$ & \nt{term} \nt{relation} \nt{term} 
              $[$ \nt{relation} \nt{term} $]$ \\
      & $|$ & \nt{predicate} \te{=>} \nt{predicate} \\
      & $|$ & \nt{predicate} \te{<=>} \nt{predicate} \\
      & $|$ & \nt{predicate} \te{||} \nt{predicate} \\
      & $|$ & \nt{predicate} \te{\&\&} \nt{predicate} \\
      & $|$ & \te{!} \nt{predicate} \\
      & $|$ & \te{if} \nt{term} \te{then} \nt{predicate} 
              \te{else} \nt{predicate} \\
      & $|$ & \verb!\forall! \nt{logic\_parameter}\plussep{\te{,}} \te{;} 
              \nt{predicate} \\
      & $|$ & \verb!\exists! \nt{logic\_parameter}\plussep{\te{,}} \te{;} 
              \nt{predicate} \\
      & $|$ & \te{(} \nt{predicate} \te{)} \\
      & $|$ & \verb!\old! \te{(} \nt{predicate} \te{)} \\
      & $|$ & \verb!\at! \te{(} \nt{predicate} \te{,} \nt{identifier} \te{)} \\
      & $|$ & \verb!\valid! \te{(} \nt{term} \te{)}
              \index{valid@\texttt{\bs{}valid}} \\
      & $|$ & \verb!\valid_index! \te{(} \nt{term} \te{,} \nt{term} \te{)}
              \index{valid\_index@\texttt{\bs{}valid\_index}} \\
      & $|$ & \verb!\valid_range! \te{(} \nt{term} \te{,} \nt{term} 
              \te{,} \nt{term} \te{)}
              \index{valid\_range@\texttt{\bs{}valid\_range}} \\
      & $|$ & \verb!\fresh! \te{(} \nt{term} \te{)}
              \index{fresh@\texttt{\bs{}fresh}} \\
  \\[0.1em]

  \nt{relation}\indexnt{relation}
    & $::=$ & \te{==} $|$ \te{!=} $|$ 
              \te{<} $|$ \te{<=} $|$ \te{>} $|$ \te{>=}
\end{tabular}\\
\hrulefill
\caption{Syntax of annotations}
\label{fig:logic}
\end{center}            
\end{figure}






\section{Memory model and pointer arithmetic
\label{pointer-arith}}
\index{Pointer arithmetic}

The C language makes almost no distinction between arrays and
pointers, and \caduceus\ identifies both notions on the model side.
A pointer \texttt{p} is either the \texttt{null} pointer, or
designates a particular element of an allocated memory block. In this
case, \texttt{p} is a pair of a base address, \texttt{base\_addr(p)},
and an offset inside the block, \texttt{offset(p)}. Pointer arithmetic
is only a matter of changing offsets inside a given block. If
\texttt{i} is an integer then the pointer \texttt{p+i} has the same
base address as \texttt{p} and offset \texttt{offset(p)+i}.
This situation is illustrated below:
\newcommand{\carre}[1]{\multicolumn{#1}{|c|}{}}
\begin{center}
  \begin{tabular}{rlllllll}
    &\qquad\qquad\qquad\qquad&&&p&\qquad\qquad\qquad&p&\hspace{-1em}+i=shift(p,i)
    \\
    &\multicolumn{3}{c}{\leftarrowfill offset(p)\rightarrowfill}&
      $\downarrow$&&$\downarrow$
    \\\cline{2-8}
    base\_addr(p)&\carre{3}&\carre{1}&\carre{1}&\carre{1}&\carre{1}
    \\\cline{2-8}\\[-0.5em]
    &\multicolumn{7}{c}{\leftarrowfill block\_length(a,p)\rightarrowfill}
  \end{tabular}
\end{center}
A memory block is constituted of zero, one or several elements; the
length of the block \texttt{p} points at is
\texttt{block\_length(a,p)}, where \texttt{a} is the current
\emph{allocation map}, of type \verb|alloc|. This map is precisely
telling if a given base address corresponds to an allocated block and,
if so, its length. The map \texttt{a} changes whenever memory is
allocated.

Pointer arithmetic may produce pointers pointing outside blocks; there
is nothing wrong with this (no proof obligation generated). But
whenever a pointer is dereferenced a proof obligation is generated
expressing the validity of this access:
\begin{displaymath}
  \mathtt{valid}(a,p) \equiv p\not=\mathtt{null} ~ \land ~ 
    0 \le \mathtt{offset}(p) < \mathtt{block\_length}(a,p)
\end{displaymath}
Additional predicates are introduced to express that access to
\texttt{p+i} and to the whole range \texttt{p[i\dots j]} is valid:
\begin{displaymath}
  \mathtt{valid\_index}(a,p,i) \equiv p\not=\mathtt{null} ~ \land ~ 
    0 \le \mathtt{offset}(p)+i < \mathtt{block\_length}(a,p)
\end{displaymath}
\begin{displaymath}
  \mathtt{valid\_range}(a,p,i,j) \equiv 
  \begin{array}[t]{l}
    p\not=\mathtt{null} ~ \land ~ 
    0 \le \mathtt{offset}(p)+i ~\land~ i\le j ~\land~ \\
    \mathtt{offset}(p)+j < \mathtt{block\_length}(a,p)
  \end{array}
\end{displaymath}
Note that two different blocks are truly \emph{separated} i.e. there
is no way to access and/or modify any element of a block using a
pointer related to another block. There is no way to
express that two blocks allocated with two different calls to
\texttt{malloc} are contiguous in memory. In particular, a C program
realizing a buffer overflow cannot be proved correct.

\medskip

As specified in the ANSI C standard, pointer subtraction and pointer
comparison is restricted to pointers with a same base address.
When it is so, pointer subtraction (resp. comparison) corresponds to
the subtraction (resp. comparison) of offsets.

\medskip

The following table summarizes the various pointer notions:
\begin{center}
  \begin{tabular}{l|l|l}
    C syntax & annotation syntax & logic \\\hline
    \texttt{p+i} & \texttt{p+i} & \texttt{shift(p,i)} \\
    \texttt{p-q} & \texttt{p-q} & \texttt{sub\_pointer(p,q)} \\
    \texttt{p<q}, \texttt{p<=p}, \texttt{p>q}, \texttt{p>=q} & 
    \texttt{p<q}, \texttt{p<=p}, \texttt{p>q}, \texttt{p>=q} &
    \{\texttt{lt},\texttt{le},\texttt{gt},\texttt{ge}\}\texttt{\_pointer(p,q)}
    \\
        & \texttt{\valid(p)} & \texttt{valid(a,p)} \\
        & \texttt{\valid\_index(p,i)} & \texttt{valid\_index(a,p,i)} \\
        & \texttt{\valid\_range(p,i,j)} & \texttt{valid\_range(a,p,i,j)} \\
        & \texttt{\block\_length(p)} & \texttt{block\_length(a,p)} \\
        & \texttt{\base\_addr(p)} & \texttt{base\_addr(p)}
  \end{tabular}
\end{center}


\section{Unsupported features}

If a feature is not supported by \caduceus, it is reported on the
error output and the corresponding function is skipped:
\begin{verbatim}
unsupported feature (goto); skipping function f
\end{verbatim}
It means that no \why\ code is produced for function \texttt{f}, and
thus no proof obligations, but \texttt{f} may still be called in other
parts of the code and its specification is visible.

\subsection{Features not supported in current version}
The following features are not yet supported by \caduceus\ but will be
so in future versions of the tool:
\begin{itemize}
\item \texttt{switch} constructs
\item string literals (e.g. \texttt{"hello world"})
\item arithmetic overflow (currently all integers types are mapped to
  arbitrary precision integers on the prover side)
\item \texttt{sizeof}
\item standard library functions: \texttt{malloc}, \texttt{memcpy},
  \texttt{strcmp}, etc.
\item more support for \verb|&| operator.
\end{itemize}

\subsection{Other unsupported features}
The following features are not planed be supported by \caduceus:
\begin{itemize}
\item arbitrary \texttt{goto} (some simple \texttt{goto}'s will be supported)
\item function pointer (apart from trivial situations where we know at
  compile-time which function is pointed at)
\item pointer cast (e.g. casting an \texttt{int*} to a \texttt{void*},
  passing it to a function which casts it back to \texttt{int*})
\item any non-ANSI feature
\end{itemize}



\nocite{*}
\bibliographystyle{plain}
\bibliography{./biblio}


%\cleardoublepage
\tableofcontents

%\cleardoublepage
\addcontentsline{toc}{chapter}{Index}
\printindex

\end{document}
