\documentclass[a4paper,12pt]{report}

\usepackage{fullpage}
\usepackage{url}
\usepackage{makeidx}
\input{./version.tex}

\newcommand{\caml}{\textsf{Caml}}
\newcommand{\pvs}{\textsf{PVS}\index{PVS@\textsf{PVS}}}
\newcommand{\coq}{\textsf{Coq}\index{Coq@\textsf{Coq}}}
\newcommand{\harvey}{\textsf{haRVey}\index{haRVey@\textsf{haRVey}}}
\newcommand{\simplify}{\textsf{Simplify}\index{Simplify@\textsf{Simplify}}}
\newcommand{\mizar}{\textsf{Mizar}\index{Mizar@\textsf{Mizar}}}
\newcommand{\hollight}{\textsf{HOL Light}\index{HOL Light@\textsf{HOL Light}}}
\newcommand{\krakatoa}{\textsf{Krakatoa}\index{Krakatoa@\textsf{Krakatoa}}}
\newcommand{\java}{\textsc{Java}\index{Java@\textsf{Java}}}
\newcommand{\jml}{\textsc{JML}\index{JML@\textsf{JML}}}
\newcommand{\why}{\textsf{Why}}
\newcommand{\caduceus}{\textsf{Caduceus}}
\newcommand{\te}[1]{\texttt{#1}}
\newcommand{\nt}[1]{$\langle$\textsl{#1}$\rangle$}
\newcommand{\indexnt}[1]{\index{#1@\textsl{#1}, grammar entry}}
\newcommand{\indextt}[1]{\index{#1@\texttt{#1}}}
\newcommand{\etoile}{$^{\star}$}
\newcommand{\etoilesep}[1]{$^{\star}_#1$}
\newcommand{\plus}{$^+$}
\newcommand{\plussep}[1]{$^+_#1$}
\newcommand{\caveat}{\paragraph{Caveat.}}
\newcommand{\caveats}{\paragraph{Caveats.}}

\makeindex

\begin{document}

%%% coverpage
\thispagestyle{empty}
\begin{center}
~\\[3cm]
\rule\textwidth{0.1cm}\\[0.5cm]
{\Huge\sf The CADUCEUS verification tool \\[0.5em] for C programs}\\[1cm]
{\Large\sf Tutorial and Reference Manual}\\[0.1cm]
\rule\textwidth{0.1cm}\\[1cm]
Version \caduceusversion\\[3cm]
Jean-Christophe Filli\^atre and Claude March\'e
\vfill
\today\\
\end{center}


\tableofcontents

%%%%%%%%%%%%%%%%%%%%%%%%%%%%%%%%%%%%%%%%%%%%%%%%%%%%%%%%%%%%%%%%%%%%%%
\chapter*{Foreword}
\addcontentsline{toc}{chapter}{Foreword}

TODO

\medskip

This manual is organized as follows. Chapter~\ref{tutorial} gives an
overview of \caduceus, illustrating all features with one-line examples.
Chapter~\ref{refman} is a reference manual.


\subsection*{License}

The \caduceus\ certification tool is \copyright\ 2002 Laboratoire de
Recherche en Informatique (\url{www.lri.fr}).
It is open source and freely available under the terms of the GNU
GENERAL PUBLIC LICENSE Version 2. See the files \texttt{COPYING} and
\texttt{GPL} in the distribution.


\subsection*{Availability}

The \caduceus\ tool is available from \url{http://why.lri.fr/}, in source
and binary formats, together with this documentation and many
examples.


%%%%%%%%%%%%%%%%%%%%%%%%%%%%%%%%%%%%%%%%%%%%%%%%%%%%%%%%%%%%%%%%%%%%%%
\chapter{Tutorial}
\label{tutorial}

% ANSI C file where annotations are inserted as special comments

\section{Verifying C programs}
\label{tutorial:C}\index{C program}

\why\ also handles C programs.
Contrary to ML input files, C input files conform to the usual ANSI C
syntax~\cite{KR88} and annotations are inserted in the source as
special comments of the shape 
\begin{verbatim}
     /*@ ... */
\end{verbatim}
On a C file \texttt{foo.c}, \why\ is invoked as for ML files:
\begin{verbatim}
     why foo.c
\end{verbatim}
and obligations are generated in file \texttt{foo\_why.v} (assuming prover
\coq\ is selected).

\subsection{Input files}
\index{C program!files}

C files contain variable and function declarations, and function
definitions. 
A variable declaration has no particular annotation:
\begin{verbatim}
     int x;
     float t[];
     int* r;
\end{verbatim}
Such a declaration corresponds to a \texttt{parameter} declaration in
an ML input file.

A function declaration can be given a specification:
\begin{verbatim}
     int f(int x);
     int g(int x) /*@ pre x <> 0 */;
     int h(int x) /*@ writes r post x = r */;
\end{verbatim}
The specification is made of an optional precondition (following
the keyword \texttt{pre}), an effect (in the same syntax as in ML
programs), and an optional postcondition (following the keyword
\texttt{post}). 

A function definition is given (optional) pre- and postcondition as
comments preceding and following the function body:
\begin{verbatim}
     void f() /*@ x = 0 */ { y = x++; } /*@ x = 1 and y = 0 */
     void g(int* x) { *x = 0; } /*@ x = 0 */
\end{verbatim}
When the function is recursive, the variant is given right after the
precondition, inside the same comment:
\begin{verbatim}
     int f(int x) /*@ x >= 0 variant x */ {
       if (x == 0) return 0;
       return f(x - 1);
     } /*@ result = 0 */
\end{verbatim}

Finally, \why\ declarations can be inserted in the middle of C
declarations, with comments of the shape \texttt{/*W}\dots\texttt{*/}:
\begin{verbatim}
     int x;
     /*W logic f: int -> int */
     int y;
\end{verbatim}

\subsection{Annotations}
\index{C program!annotations}

\paragraph{Loops annotations.}
\indextt{invariant}\indextt{variant}
Inside C code, annotations must be inserted to give loops invariant and
variant. \texttt{for} loops are given an annotation right before the
loop body:
\begin{verbatim}
     for (i = 0; i < 10; ++i)
       /*@ invariant x = i and i <= 10 variant 10-i */
       { x = x + 1; }
\end{verbatim}
Similarly for \texttt{while} loops:
\begin{verbatim}
     while (n >= 0) 
       /*@ invariant 1 <= n variant n */ { 
       if (n == 1) { n++; break; }
       n--;
     }
\end{verbatim}
On the contrary, \texttt{do} loops are annotated after the loop body:
\begin{verbatim}
     do {
       x = x + 1;
       i = i - 1;
     }
     /*@ invariant x = 10 - i and i >= 0 variant i */
     while (i > 0);
\end{verbatim}

\paragraph{Other annotations.}
\indextt{assert}\indextt{label}
Assertions can be inserted within statements, with the syntax:
\begin{verbatim}
     x = 2 * x;
     /*@ assert even(x) */;
     x = x + 1;
\end{verbatim}
Labels are inserted within statements with a similar syntax:
\begin{verbatim}
     x = 2 * x;
     /*@ label L */;
     x = x + 1;
\end{verbatim}
or using C labels directly:
\begin{verbatim}
     x = 2 * x;
   L:
     x = x + 1;
\end{verbatim}

\subsection{C fragment currently covered}
\index{C program!unsupported constructs}

Section~\ref{syntax:C} gives the syntactic fragment of ANSI C
currently supported in \why. Additionally, some semantic restrictions
are added.
\begin{description}
\item[No pointer arithmetic.] Pointers are partly supported. They can
  be used to do reference-passing, as in:
\begin{verbatim}
     void g(int* x) { *x = 0; } /*@ x = 0 */
     void f() { int i = 1; g(&i); }
\end{verbatim}
  but pointer arithmetic is not allowed, as in:
\begin{verbatim}
     for (p = a; c = *p; p++) ...
\end{verbatim}

\item[Types.] Currently, the types \texttt{char},
  \texttt{short}, \texttt{int} and \texttt{long} are all identified
  (and modeled as unbound integers as in ML). Similarly, types
  \texttt{float} and \texttt{double} are identified (and modeled as
  reals as in ML).

\end{description}

% Note that \caml\ code can also be generated from C input files, giving
% an ML translation of the C code. (This is not of great use, though,
% since the C input code is already executable.)



\chapter{Reference manual}
\label{refman}

\section{Command line options}
\label{usage}\index{Option, of the command line}


\section{C programs}
\label{syntax:C}\index{C program!syntax}

Contrary to ML input files, C input files conform to the usual ANSI C
syntax and annotations are inserted in the source as comments.

\subsection{Lexical conventions}
\label{lexical:c}\index{Lexical conventions!C programs}

The lexical conventions conform to the ANSI C standard (see
for instance~\cite{KR88}).

\subsection{Annotations}
\index{C program!annotations}

Annotations are inserted as special comments of the shape
\texttt{/*@ ... */} or \texttt{/*W ... */}.
Note that within annotations the syntax in use is the one given in
Section~\ref{syntax:logic}, with its own lexical conventions,
different from the C ones. In particular, comments are enclosed by
\texttt{(*} and \texttt{*)}.

\subsection{Syntax}

The syntax of C files is a fragment of the ANSI C syntax
and is given in Figures~\ref{fig:cfiles:a} and~\ref{fig:cfiles:b}.

\begin{figure}[htbp]
\begin{center}
\hrulefill\\
\begin{tabular}{lrl}
  \nt{c\_file}
    & $::=$ & \nt{c\_declaration}\etoile\ \\
  \\[0.1em]

  \nt{c\_declaration}
    & $::=$ & \nt{c\_type} $[$ \te{*} $]$ \nt{identifier} \te{;} \\
      & $|$ & \nt{c\_type} \nt{identifier} \te{[]} \te{;} \\
      & $|$ & \nt{c\_type} \nt{identifier} \te{(} 
              \nt{parameter}\etoilesep{\te{,}} \te{)} 
              $[$ \nt{c\_spec} $]$ \te{;} \\
      & $|$ & \nt{c\_type} \nt{identifier} \te{(} 
              \nt{parameter}\etoilesep{\te{,}} \te{)} 
              \nt{annotated\_block} \te{;} \\
      & $|$ & \te{/*W} \te{logic} \nt{identifier} \te{:} 
              \nt{logic\_type} \te{*/} \\
  \\[0.1em]

  \nt{c\_type}
    & $::=$ & \te{void} $|$ \te{char} $|$ \te{short} $|$ \te{int} 
              $|$ \te{long} $|$ \te{float} $|$ \te{double} 
              $|$ \nt{identifier} \\
  \\[0.1em]

  \nt{parameter}
    & $::=$ & \nt{c\_type} $[$ \te{*} $]$ \nt{identifier} 
          $|$ \nt{c\_type} \nt{identifier} \te{[]} \\
  \\[0.1em]

  \nt{c\_spec}
    & $::=$ & \te{/*@} $[$ \te{pre} \nt{precondition} $]$ \nt{effects}
              $[$ \te{post} \nt{postcondition} $]$ \te{*/}  \\
  \\[0.1em]

  \nt{c\_expr}
    & $::=$ & \nt{integer-constant} \\
    &   $|$ & \nt{identifier} \\
    &   $|$ & \nt{identifier} \te{[} \nt{c\_expr} \te{]} \\
    &   $|$ & \nt{c\_expr} \te{,} \nt{c\_expr} \\
    &   $|$ & \nt{c\_expr} \nt{assignment\_operator} \nt{c\_expr} \\
    &   $|$ & \nt{c\_expr} \nt{binary\_operator} \nt{c\_expr} \\
    &   $|$ & \nt{c\_expr} \nt{postfix\_operator} \\
    &   $|$ & \nt{prefix\_operator} \nt{c\_expr} \\
    &   $|$ & \nt{identifier} \te{(} \nt{c\_expr}\etoilesep{\te{,}} \te{)} \\
    &   $|$ & \nt{c\_expr} \te{?} \nt{c\_expr} \te{:} \nt{c\_expr} \\
  \\[0.1em]

  \nt{assignment\_operator}
    & $::=$ & \te{=} $|$ \te{*=} $|$ \te{/=} $|$ \te{\%=} $|$ \te{+=} 
          $|$ \te{-=} $|$ \te{<<=} $|$ \te{>>=} $|$ \te{\&=} $|$ \te{\^{}=}
          $|$ \te{|=}
  \\[0.1em]

  \nt{prefix\_operator}
    & $::=$ & \te{++} $|$ \te{--} $|$ \te{-} $|$ \te{!} $|$ \te{\&}
          $|$ \te{*}
  \\[0.1em]

  \nt{postfix\_operator}
    & $::=$ & \te{++} $|$ \te{--}
  \\[0.1em]

  \nt{binary\_operator}
    & $::=$ & \te{+} $|$ \te{-} $|$ \te{*} $|$ \te{/} $|$ \te{\%} $|$ \te{<} 
          $|$ \te{>} $|$ \te{<=} $|$ \te{>=} $|$ \te{==} $|$ \te{!=} 
          $|$ \te{\&} $|$ \te{\^{}} $|$ \te{|} $|$ \te{\&\&} $|$ \te{||} \\
  \\[0.1em]

\end{tabular}\\
\hrulefill
\caption{Syntax of C files: declarations and expressions}
\label{fig:cfiles:a}
\end{center}           
\end{figure}

\begin{figure}[htbp]
\begin{center}
\hrulefill\\
\begin{tabular}{lrl}
  \nt{statement}
    & $::=$ & \nt{c\_expr} \\
    &   $|$ & \te{if} \te{(} \nt{c\_expr} \te{)} \nt{statement} 
              $[$ \te{else} \nt{statement} $]$ \\
    &   $|$ & \te{while} \te{(} \nt{c\_expr} \te{)}
              \nt{c\_loop\_annot} \nt{statement} \\
    &   $|$ & \te{do} \nt{statement} \nt{c\_loop\_annot} 
              \te{while} \te{(} \nt{c\_expr} \te{)} \\
    &   $|$ & \te{for} \te{(} \nt{statement} \te{;} \nt{statement} \te{;}
              $[$ \nt{c\_expr} $]$ \te{)} \\
           && \nt{c\_loop\_annot} \nt{statement} \\
    &   $|$ & \nt{identifier} \te{:} \nt{statement} \\
    &   $|$ & \te{/*@} \te{assert} \nt{assertion} \te{*/} \\ \indextt{assert}
    &   $|$ & \te{/*@} \te{label} \nt{identifier} \te{*/} \\ \indextt{label}
    &   $|$ & \nt{block} \\
  \\[0.1em]

  \nt{block}
    & $::=$ & \te{\{} \nt{c\_declaration}\etoile\ 
              \nt{statement}\etoilesep{\te{;}} \te{\}}\\
  \\[0.1em]

  \nt{annotated\_block}
    & $::=$ & $[$ \nt{c\_pre} $]$ \nt{block} $[$ \nt{c\_post} $]$ \\
  \\[0.1em]

  \nt{c\_pre}
    & $::=$ & \te{/*@} $[$ \nt{precondition} $]$ $[$ \te{variant} \nt{wf\_arg}
              $]$ \te{*/}  \\
  \nt{c\_post}
    & $::=$ & \te{/*@} $[$ \nt{postcondition} $]$ \te{*/}  \\
  \nt{c\_loop\_annot}
    & $::=$ & \te{/*@} \te{invariant} \nt{assertion} 
              \te{variant} \nt{wf\_arg} \te{*/}  \\
  \\[0.1em]

\end{tabular}\\
\hrulefill
\caption{Syntax of C files (cont'd): statements and annotations}
\label{fig:cfiles:b}
\end{center}           
\end{figure}


\nocite{*}
\bibliographystyle{plain}
\bibliography{./biblio}


\newpage
\addcontentsline{toc}{chapter}{Index}
\printindex

\end{document}
