\documentclass[a4paper,12pt]{report}

\usepackage{fullpage}
\usepackage{url}
\usepackage{makeidx}
\usepackage{alltt}
\input{./version.tex}

\newcommand{\caml}{\textsf{Caml}}
\newcommand{\pvs}{\textsf{PVS}\index{PVS@\textsf{PVS}}}
\newcommand{\coq}{\textsf{Coq}\index{Coq@\textsf{Coq}}}
\newcommand{\harvey}{\textsf{haRVey}\index{haRVey@\textsf{haRVey}}}
\newcommand{\simplify}{\textsf{Simplify}\index{Simplify@\textsf{Simplify}}}
\newcommand{\mizar}{\textsf{Mizar}\index{Mizar@\textsf{Mizar}}}
\newcommand{\hollight}{\textsf{HOL Light}\index{HOL Light@\textsf{HOL Light}}}
\newcommand{\krakatoa}{\textsf{Krakatoa}\index{Krakatoa@\textsf{Krakatoa}}}
\newcommand{\java}{\textsc{Java}\index{Java@\textsf{Java}}}
\newcommand{\jml}{\textsc{JML}\index{JML@\textsf{JML}}}
\newcommand{\why}{\textsf{Why}}
\newcommand{\caduceus}{\textsf{Caduceus}}
\newcommand{\te}[1]{\texttt{#1}}
\newcommand{\nt}[1]{$\langle$\textsl{#1}$\rangle$}
\newcommand{\indexnt}[1]{\index{#1@\textsl{#1}, grammar entry}}
\newcommand{\indextt}[1]{\index{#1@\texttt{#1}}}
\newcommand{\etoile}{$^{\star}$}
\newcommand{\etoilesep}[1]{$^{\star}_#1$}
\newcommand{\plus}{$^+$}
\newcommand{\plussep}[1]{$^+_#1$}
\newcommand{\caveat}{\paragraph{Caveat.}}
\newcommand{\caveats}{\paragraph{Caveats.}}
\newenvironment{code}{\begin{small}\begin{alltt}%
\begin{tabular}{|p{0.97\textwidth}|}\hline%
}{\\\hline\end{tabular}\end{alltt}\end{small}}
\def\result{\char'134 result}
\def\forall{\char'134 forall}
\def\old{\char'134 old}
\def\bs{\char'134}

\makeindex

\begin{document}

%%% coverpage
\thispagestyle{empty}
\begin{center}
~\\[3cm]
\rule\textwidth{0.1cm}\\[0.5cm]
{\Huge\sf The CADUCEUS verification tool \\[0.5em] for C programs}\\[1cm]
{\Large\sf Tutorial and Reference Manual}\\[0.1cm]
\rule\textwidth{0.1cm}\\[1cm]
Version \caduceusversion\\[3cm]
Jean-Christophe Filli\^atre and Claude March\'e
\vfill
\today\\
\end{center}


\tableofcontents

%%%%%%%%%%%%%%%%%%%%%%%%%%%%%%%%%%%%%%%%%%%%%%%%%%%%%%%%%%%%%%%%%%%%%%
\chapter*{Foreword}
\addcontentsline{toc}{chapter}{Foreword}

\caduceus\ is a verification tool for C programs. Programs are
specified using annotations in special comments. Proof obligations are
generated using the \why\ tool~\cite{Why} and then can be validated using
various proof assistants or decision procedures.

\medskip

This manual is organized as follows. Chapter~\ref{tutorial} gives an
overview of \caduceus, illustrating all features with tiny examples.
Chapter~\ref{refman} is a reference manual.


\subsection*{License}

The \caduceus\ certification tool is \copyright\ 2003 Laboratoire de
Recherche en Informatique (\url{www.lri.fr}).
It is open source and freely available under the terms of the GNU
GENERAL PUBLIC LICENSE Version 2. See the files \texttt{COPYING} and
\texttt{GPL} in the distribution.


\subsection*{Authors}

\caduceus\ is developed by Jean-Christophe Filli\^atre and Claude March\'e.

\subsection*{Availability}

The \caduceus\ tool is available from
\url{http://why.lri.fr/caduceus/}, 
in source and binary formats, together with this documentation and
examples.


%%%%%%%%%%%%%%%%%%%%%%%%%%%%%%%%%%%%%%%%%%%%%%%%%%%%%%%%%%%%%%%%%%%%%%
\chapter{Tutorial}
\label{tutorial}

This chapter gives an overview of \caduceus\ using trivial programs.

\section{Basics}

Let us start with a trivial function \texttt{max} computing the maximum of two
integers:
\begin{code}
  
/*@ ensures 
  @   \result >= x && \result >= y &&
  @   \forall int z; z >= x && z >= y => z >= \result 
  @*/
int max(int x, int y) \{
  if (x > y) return x; else return y;
\}
\end{code}
The specification is inserted in the source code right before the
function definition in a comment of the shape \texttt{/*@ \dots */}.
Any \caduceus\ annotation is placed in such a comment (or in
a single-line comment \texttt{//@ \dots}).
Note that the character \texttt{@} is considered as a blank inside
annotations.

The specification is here a post-condition, introduced by the keyword
\texttt{ensures}. This is the predicate to be valid when the function
returns. The keyword \texttt{\result} denotes the value returned by
the function, thus of type \texttt{int} in this example. The
post-condition expresses that \texttt{\result} is the smallest integer
greater or equal than \texttt{x} and \texttt{y} (there are plenty
other ways to specify the \texttt{max} function).

The proof obligations are generated in two steps. First, a \why\
program \texttt{max.why} is built from the source \texttt{max.c} with
the command:
\begin{verbatim}
    caduceus max.c
\end{verbatim}
Then the proof obligations are generated from \texttt{max.why}
depending on the selected proof tool. If the proofs are to be made
with the \coq\ proof assistant, the command is:
\begin{verbatim}
    why --coq caduceus.why max.why
\end{verbatim}
The file \texttt{caduceus.why} was generated by the call to
\texttt{caduceus}; it contains all the \why\ declarations which
are necessary to compile \texttt{max.why} (the model).
Two \coq\ files \texttt{caduceus\_why.v} and \texttt{max\_why.v} are
produced. The latter contains two proof
obligations (two lemmas which proofs have to be filled in); see the
\why\ manual for more details~\cite{Why}.

As an alternative to \coq\, let us use the \simplify\ decision procedure.
The \why\ command becomes:
\begin{verbatim}
    why --simplify caduceus.why max.why
\end{verbatim}
and two files \texttt{caduceus\_why.sx} and \texttt{max\_why.sx} are
produced. Then they are passed to \simplify\ which dischargs the two
obligations automatically:
\begin{verbatim}
    cat caduceus_why.sx pointer_why.sx | Simplify
    >       1: Valid.
    >       2: Valid.
\end{verbatim}

\subsection{Loops annotations}
\indextt{invariant}\indextt{variant}

A loop can be annotated with an invariant (a property which holds at
the loop entrance and is preserved by the loop body) and a variant (a
quantity which decreases at each loop step for a well-founded
ordering, thus ensuring termination). The three kinds of C loops can
be annotated (\texttt{while}, \texttt{for}, and \texttt{do \dots\ while}).

A loop annotation is placed right before the loop; the keyword
\texttt{invariant} (resp. \texttt{variant}) introduces the invariant
(resp. the variant). The invariant is optional and must appear before
the variant when present.

Here is an example with a trivial loop:
\begin{code}

  int i = 0;
  int s = 0;
  /*@ invariant s == i variant 10-i */
  while (i++ < 10) s++;
\end{code}

\subsection{Other annotations}

Beside loop annotations, there are three other kinds of inner annotations.

\paragraph{Intermediate assertions.}\indextt{assert}

\paragraph{Labels.}\indextt{label}

\paragraph{Specifying pieces of code.}

\section{Pointers and arrays}
\index{valid@\texttt{\bs{}valid}}
\index{valid\_index@\texttt{\bs{}valid\_index}}
\index{valid\_range@\texttt{\bs{}valid\_range}}

\section{Calling functions}


\section{The \texttt{assigns} clause}
\label{assigns}\indextt{assigns}



\section{Logical declarations}
\indextt{predicate}\indextt{logic}\indextt{axiom}


\section{A complete example}


%%%%%%%%%%%%%%%%%%%%%%%%%%%%%%%%%%%%%%%%%%%%%%%%%%%%%%%%%%%%%%%%%%%%%%
\chapter{Reference manual}
\label{refman}

\caduceus\ is invoked as a batch compiler, given a list of input files:
\begin{center}
  \texttt{caduceus} [\textit{options}] \textit{file}$_1$\texttt{.c} $\cdots$ \textit{file}$_n$\texttt{.c}
\end{center}

\section{Command line options}
\label{usage}\index{Option, of the command line}

\begin{description}
  \item[Generic options:] ~\par
  \item[\texttt{--help}]: ~\par    
    Give usage and exit. 
  \item[\texttt{--version}]: ~\par    
    Give \caduceus\ version number and exit. 
  \item[\texttt{-v}]: ~\par 
    Verbose mode. 
  \item[\texttt{-q}]: ~\par  
    Quiet mode (default).
  \item[\texttt{--werror}]: ~\par 
    Turn warnings as errors. \caduceus\ will stop on the first warning.
  \item[\texttt{-d}]: ~\par 
    Debugging mode. In this mode, \caduceus\ details everything it does,
    printing a lot of material on error output. Not of use for the
    casual user.

  \item[Overall options:] ~\par
  \item[\texttt{-parse-only}]: ~\par  
    Stop after parsing.
  \item[\texttt{-type-only}]: ~\par  
    Stop after type-checking.

  \item[Pre-processing options:] ~\par
  \item[\texttt{-no-cpp}]: ~\par  
    Turn off the pre-processing.
  \item[\texttt{-cpp} \textit{command}]: ~\par  
    Set the pre-processor. It must be an executable taking the file
    to be pre-processed on its command line and printing its output
    on the standard output. 
    The default value is \texttt{gcc -C -E}.
  \item[\texttt{-E}]: ~\par  
    Stop after pre-processing and dump the pre-processed file on
    standard output (mainly useful for debugging).

\end{description}

\section{Input Files Syntax}
\label{syntax}\index{Syntax (of input files)}

C files conform to the usual ANSI C
syntax and annotations are inserted in the source as comments.

\subsection{Lexical conventions}
\label{lexical:c}\index{Lexical conventions!C programs}

\subsubsection{Code}

The lexical conventions conform to the ANSI C standard (see
for instance~\cite{KR88}).

\subsubsection{Annotations}
\index{Annotations}

Within annotations the lexical conventions are the same as C ones,
except that:
\begin{itemize}
\item comments are enclosed by \texttt{(*} and \texttt{*)} (so that
  you can put comments inside annotations without messing traditional
  C compilers)
\item the set of keywords is different: \par
  \begin{center}
  \begin{tabular}{l@{\qquad}l@{\qquad}l@{\qquad}l@{\qquad}l}
  \verb!\forall! & \verb!\exists! & \verb!int! & \verb!float! &
  \verb!decreases! \\
  \verb!\true! & \verb!\false! & \verb!if! & \verb!then! & \verb!else! \\
  \verb!invariant! & \verb!variant! & \verb!for! & \verb!label! & 
  \verb!assert! \\ 
  \verb!requires! & \verb!ensures! & \verb!assigns! & \verb!logic! & 
  \verb!axiom! \\
  \verb!predicate! & \verb!\result! & \verb!\old! & 
  \verb!\length! & \verb!\null! \\
  \verb!reads! & \verb!\valid! & \verb!\valid_index! & \verb!\valid_range!
  \end{tabular}
  \end{center}
\end{itemize}

\subsection{Syntax}

Annotations are inserted into programs as special comments of the shape
\texttt{/*@ ... */} or one-line comments of the shape \texttt{//@ ...}
Figure~\ref{fig:cfiles} gives the syntax for the additional C
constructs containing annotations; some are C declarations
(non-terminal \nt{declaration}) and some are C statements (non-terminal
\nt{statement}).
Figure~\ref{fig:logic} gives the syntax for the
annotations. 

\begin{figure}[htbp]
\begin{center}
\hrulefill\\
\begin{tabular}{lrl}
  \nt{c\_file}
    & $::=$ & \nt{declaration}\etoile\ \\
  \\[0.1em]

  \nt{declaration}
    & $::=$ & \nt{spec} \nt{type} \nt{identifier} \te{(} 
              \nt{parameter}\etoilesep{\te{,}} \te{)} 
              \te{;} \\
      & $|$ & \nt{spec} \nt{type} \nt{identifier} \te{(} 
              \nt{parameter}\etoilesep{\te{,}} \te{)} 
              \nt{block} \te{;} \\
      & $|$ & \te{/*@} \te{logic} \nt{logic\_type} \nt{identifier} \te{(} 
              \nt{logic\_parameter}\etoilesep{\te{,}} \te{)} \\
              && $[$ 
              \te{reads} \nt{location}\plussep{\te{,}} $]$ \te{*/} \\
      & $|$ & \te{/*@} \te{predicate} \nt{identifier} \te{(} 
              \nt{logic\_parameter}\etoilesep{\te{,}} \te{)} \\
           && $[$ \te{\{} \nt{predicate} \te{\}} $|$ 
              \te{reads} \nt{location}\plussep{\te{,}} $]$ \te{*/} \\
      & $|$ & \te{/*@} \te{axiom} \nt{identifier} \te{:} 
              \nt{predicate} \te{*/} \\
  \\[0.1em]

  \nt{spec}
    & $::=$ & \te{/*@} $[$ \te{requires} \nt{predicate} $]$ 
              $[$ \te{assigns} \nt{location}\plussep{\te{,}} $]$ \\
           && $[$ \te{ensures} \nt{predicate} $]$ 
              $[$ \te{decreases} \nt{variant} $]$ \te{*/} \\
                  \indextt{requires}\indextt{assigns}
                  \indextt{ensures}\indextt{decreases}
  \\[0.1em]

  \nt{statement}
    & $::=$ & \nt{loop\_annot} \te{while} \te{(} \nt{expr} \te{)}
              \nt{statement} \\
    &   $|$ & \nt{loop\_annot} \te{do} \nt{statement} 
              \te{while} \te{(} \nt{expr} \te{)} \\
    &   $|$ & \nt{loop\_annot} \te{for} \te{(} \nt{statement} \te{;} \nt{statement} \te{;}
              $[$ \nt{expr} $]$ \te{)} \\
           && \nt{statement} \\
    &   $|$ & \te{/*@} \te{assert} \nt{predicate} \te{*/} \\ \indextt{assert}
    &   $|$ & \te{/*@} \te{label} \nt{identifier} \te{*/} \\ \indextt{label}
    &   $|$ & \nt{spec} \nt{statement} \\
  \\[0.1em]

  \nt{loop\_annot}
    & $::=$ & \te{/*@} $[$ \te{invariant} \nt{predicate} $]$
              $[$ \te{variant} \nt{variant} $]$ \te{*/}  \\
              \indextt{invariant}\indextt{variant}
  \nt{variant} 
    & $::=$ & \nt{term} $[$ \te{for} \nt{identifier} $]$ \\

  \\[0.1em]
  
  \nt{logic\_type}
    & $::=$ & \te{int} $|$ \te{float} $|$ \nt{logic\_type} \te{[]} 
            $|$ \nt{identifier} \\
  \nt{logic\_parameter}
    & $::=$ & \nt{logic\_type} \nt{identifier} \\

  \\[0.1em]

  \nt{location}
    & $::=$ & \nt{identifier} \\
    &   $|$ & \nt{term} \te{.} \nt{identifier} \\
    &   $|$ & \nt{term} \te{->} \nt{identifier} \\
    &   $|$ & \te{*} \nt{term} \\
    &   $|$ & \nt{term} \te{[} \nt{term} \te{]} \\
    &   $|$ & \nt{term} \te{[} \te{*} \te{]} \\
    &   $|$ & \nt{term} \te{[} \nt{term} \te{..} \nt{term} \te{]} \\
  \\[0.1em]
\end{tabular}\\
\hrulefill
\caption{Annotations within C constructs}
\label{fig:cfiles}
\end{center}           
\end{figure}


\begin{figure}[htbp]
\begin{center}
\hrulefill\\
\begin{tabular}{lrl}
  \nt{term}\indexnt{term}
    & $::=$ & \nt{constant} \\
      & $|$ & \nt{term} \nt{arith\_op} \nt{term} \\
      & $|$ & \te{-} \nt{term} $|$ \te{+} \nt{term}  \\
      & $|$ & \te{*} \nt{term} \\
      & $|$ & \nt{term} \te{->} \nt{identifier} \\
      & $|$ & \nt{term} \te{.} \nt{identifier} \\
      & $|$ & \nt{identifier} \te{(} \nt{term}\plussep{\te{,}} \te{)} \\
      & $|$ & \nt{term} \te{[} \nt{term} \te{]} \\
      & $|$ & \nt{term} \te{?} \nt{term} \te{:} \nt{term} \\
      & $|$ & \te{(} \nt{term} \te{)} \\
      & $|$ & \verb!\old! \te{(} \nt{term} \te{)} \\
      & $|$ & \verb!\at! \te{(} \nt{term} \te{,} \nt{identifier} \te{)} \\
      & $|$ & \verb!\length! \te{(} \nt{term} \te{)} \\
      & $|$ & \verb!\result! \\
      & $|$ & \verb!\null! \\
  \\[0.1em]

  \nt{constant}\indexnt{constant}
    & $::=$ & \nt{integer-constant} \\
      & $|$ & \nt{floating-point-constant} \\
  \\[0.1em]

  \nt{arith\_op}\indexnt{arith\_op}
    & $::=$ & \te{+} $|$ \te{-} $|$ \te{*} $|$ \te{/} $|$ \te{\%} \\
  \\[0.1em]

  \nt{predicate}\indexnt{predicate}
    & $::=$ & \verb!\true! \\
      & $|$ & \verb!\false! \\
      & $|$ & \nt{identifier} \\
      & $|$ & \nt{identifier} \te{(} \nt{term}\plussep{\te{,}} \te{)} \\
      & $|$ & \nt{term} \nt{relation} \nt{term} 
              $[$ \nt{relation} \nt{term} $]$ \\
      & $|$ & \nt{predicate} \te{=>} \nt{predicate} \\
      & $|$ & \nt{predicate} \te{||} \nt{predicate} \\
      & $|$ & \nt{predicate} \te{\&\&} \nt{predicate} \\
      & $|$ & \te{!} \nt{predicate} \\
      & $|$ & \te{if} \nt{term} \te{then} \nt{predicate} 
              \te{else} \nt{predicate} \\
      & $|$ & \verb!\forall! \nt{logic\_parameter}\etoilesep{\te{,}} \te{;} 
              \nt{predicate} \\
      & $|$ & \verb!\exists! \nt{logic\_parameter}\etoilesep{\te{,}} \te{;} 
              \nt{predicate} \\
      & $|$ & \te{(} \nt{predicate} \te{)} \\
      & $|$ & \verb!\old! \te{(} \nt{predicate} \te{)} \\
      & $|$ & \verb!\at! \te{(} \nt{predicate} \te{,} \nt{identifier} \te{)} \\
      & $|$ & \verb!\valid! \te{(} \nt{term} \te{)}
              \index{valid@\texttt{\bs{}valid}} \\
      & $|$ & \verb!\valid_index! \te{(} \nt{term} \te{,} \nt{term} \te{)}
              \index{valid\_index@\texttt{\bs{}valid\_index}} \\
      & $|$ & \verb!\valid_range! \te{(} \nt{term} \te{,} \nt{term} 
              \te{,} \nt{term} \te{)}
              \index{valid\_range@\texttt{\bs{}valid\_range}} \\
  \\[0.1em]

  \nt{relation}\indexnt{relation}
    & $::=$ & \te{==} $|$ \te{!=} $|$ 
              \te{<} $|$ \te{<=} $|$ \te{>} $|$ \te{>=}
\end{tabular}\\
\hrulefill
\caption{Syntax of annotations}
\label{fig:logic}
\end{center}            
\end{figure}



\nocite{*}
\bibliographystyle{plain}
\bibliography{./biblio}


\newpage
\addcontentsline{toc}{chapter}{Index}
\printindex

\end{document}
