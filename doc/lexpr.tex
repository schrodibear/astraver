%%%%%%%%%%%%%%%%%%%%%%%%%%%%%%%%%%%%%%%%%%%%%%%%%%%%%%%%%%%%%%%%%%%%%%%%%%
%                                                                        %
%  The Why platform for program certification                            %
%                                                                        %
%  Copyright (C) 2002-2011                                               %
%                                                                        %
%    Jean-Christophe FILLIATRE, CNRS & Univ. Paris-sud 11                %
%    Claude MARCHE, INRIA & Univ. Paris-sud 11                           %
%    Yannick MOY, Univ. Paris-sud 11                                     %
%    Romain BARDOU, Univ. Paris-sud 11                                   %
%                                                                        %
%  Secondary contributors:                                               %
%                                                                        %
%    Thierry HUBERT, Univ. Paris-sud 11  (former Caduceus front-end)     %
%    Nicolas ROUSSET, Univ. Paris-sud 11 (on Jessie & Krakatoa)          %
%    Ali AYAD, CNRS & CEA Saclay         (floating-point support)        %
%    Sylvie BOLDO, INRIA                 (floating-point support)        %
%    Jean-Francois COUCHOT, INRIA        (sort encodings, hyps pruning)  %
%    Mehdi DOGGUY, Univ. Paris-sud 11    (Why GUI)                       %
%                                                                        %
%  This software is free software; you can redistribute it and/or        %
%  modify it under the terms of the GNU Lesser General Public            %
%  License version 2.1, with the special exception on linking            %
%  described in file LICENSE.                                            %
%                                                                        %
%  This software is distributed in the hope that it will be useful,      %
%  but WITHOUT ANY WARRANTY; without even the implied warranty of        %
%  MERCHANTABILITY or FITNESS FOR A PARTICULAR PURPOSE.                  %
%                                                                        %
%%%%%%%%%%%%%%%%%%%%%%%%%%%%%%%%%%%%%%%%%%%%%%%%%%%%%%%%%%%%%%%%%%%%%%%%%%

%       | "\let" id "=" lexpr ";" lexpr ; local binding
%       | id ":" lexpr ; syntactic naming

\begin{syntax}
  rel-op ::= "==" | "!=" | "<=" | ">=" | ">" | "<"
       \
  bin-op ::= "+" | "-" | "*" | "/" | "%" ;
       | "&&" | "||" |   ; boolean operations
       | "&" | "|" ; bitwise operations
       | "==>" ; boolean implication
       | "<==>" ; boolean equivalence
       \
  unary-op ::= "+" | "-" ; unary plus and minus
       | "!" ; boolean negation
       | "~" ;  bitwise complementation
       \
  lexpr ::= "true" | "false" ;
       | integer ; integer constants
       | real ; real constants
       | id ; variables
       | unary-op lexpr ;
       | lexpr bin-op lexpr ;
       | lexpr (rel-op lexpr)+ ; comparisons, see remark below
       | lexpr "[" lexpr "]" ; array access
       | lexpr "." id  ; object field access
       | "(" type-expr ")" lexpr  ; cast
       | id "(" lexpr ("," lexpr)* ")" ; function application
       | "(" lexpr ")" ; parentheses
       | lexpr "?" lexpr ":" lexpr ;
       | "\forall" binders ";" lexpr ; universal quantification
       | "\exists" binders ";" lexpr ; existential quantification
       \
  binders ::= type-expr variable-ident+ ;
  ("," variable-ident+)*
  \
  type-expr ::= logic-type-expr | Java-type-expr
  \
  logic-type-expr ::= built-in-logic-type ;
  | id ; logic type identifier
  \
  built-in-logic-type ::= "integer" | "real" 
  \
  variable-ident ::= id 
  | variable-ident "[]"
\end{syntax}

%%% Local Variables:
%%% mode: latex
%%% TeX-master: "main"
%%% End:
